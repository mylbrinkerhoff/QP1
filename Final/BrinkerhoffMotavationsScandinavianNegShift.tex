% !TEX TS-program = xelatex
% !TEX encoding = UTF-8 Unicode		

\documentclass[12pt, letterpaper]{article}

%%BIBLIOGRAPHY- This uses biber/biblatex to generate bibliographies according to the 
%%Unified Style Sheet for Linguistics
\usepackage[main=american, german]{babel}% Recommended
\usepackage{csquotes}% Recommended
\usepackage[backend=biber,
		style=unified,
		maxcitenames=3,
		maxbibnames=99,
		natbib,
		url=false]{biblatex}
\addbibresource{QP1.bib}
% \addbibresource{link_desktop.bib}
\setcounter{biburlnumpenalty}{100}  % allow URL breaks at numbers
%\setcounter{biburlucpenalty}{100}   % allow URL breaks at uppercase letters
%\setcounter{biburllcpenalty}{100}   % allow URL breaks at lowercase letters

%%TYPOLOGY
\usepackage[svgnames]{xcolor} % Specify colors by their 'svgnames', for a full list of all colors available see here: http://www.latextemplates.com/svgnames-colors
%\usepackage[compact]{titlesec}
%\titleformat{\section}[runin]{\normalfont\bfseries}{\thesection.}{.5em}{}[.]
%\titleformat{\subsection}[runin]{\normalfont\scshape}{\thesubsection}{.5em}{}[.]
\usepackage[hmargin=1in,vmargin=1in]{geometry}  %Margins          
\usepackage{graphicx}	%Inserting graphics, pictures, images 		
\usepackage{stackengine} %Package to allow text above or below other text, Also helpful for HG weights 
\usepackage{fontspec} %Selection of fonts must be ran in XeLaTeX
\usepackage{amsfonts, amssymb, amsmath}
\usepackage{mathtools} % also loads amsmath
\usepackage{stmaryrd} % nice denotation brackets and a few other flourishes
\usepackage{setspace} %Linespacing
\usepackage{multicol} %Multicolumn text
\usepackage{enumitem} %Allows for continuous numbering of lists over examples, etc.
\usepackage{multirow} %Useful for combining cells in tablesbrew 
\usepackage{hanging}
\usepackage{datetime2}
\usepackage{fancyhdr} %Allows for the 
\pagestyle{fancy}
\fancyhead[L]{\textit{Qualifying Paper}} 
\fancyhead[R]{\textit{\today}} 
\fancyfoot[L,R]{} 
\fancyfoot[C]{\thepage} 
\renewcommand{\headrulewidth}{0.4pt}
\setlength{\headheight}{14.5pt} % ...at least 14.49998pt
% \usepackage{fourier} % This allows for the use of certain wingdings like bombs, frowns, etc.
% \usepackage{fourier-orns} %More useful symbols like bombs and jolly-roger, mostly for OT
\usepackage[colorlinks,allcolors={black},urlcolor={blue}]{hyperref} %allows for hyperlinks and pdf bookmarks
% \usepackage{url} %allows for urls
% \def\UrlBreaks{\do\/\do-} %allows for urls to be broken up
\usepackage[normalem]{ulem} %strike out text. Handy for syntax
\usepackage{tcolorbox}


%%FONTS
\setmainfont{Libertinus Serif}
\setsansfont{Libertinus Sans}
\setmonofont[Scale=MatchLowercase]{Libertinus Mono}

%%PACKAGES FOR LINGUISTICS
%\usepackage{OTtablx} %Generating tableaux with using TIPA
% \usepackage[noipa]{OTtablx} % Use this one generating tableaux without using TIPA
% \usepackage[notipa]{ot-tableau} % Another tableau drawing packing use for posters.
% \usepackage{linguex} % Linguistic examples
% \usepackage{langsci-linguex} % Linguistic examples
\usepackage{langsci-gb4e} % Language Science Press' modification of gb4e
\usepackage{langsci-avm} % Language Science Press' AVM package
\usepackage{tikz} % Drawing Hasse diagrams
% \usepackage{pst-asr} % Drawing autosegmental features
\usepackage{pstricks} % required for pst-asr, OTtablx, pst-jtree.
\usepackage{pst-jtree} 	% Syntax tree drawing software 
\usepackage{tikz-qtree}	% Another syntax tree drawing software. Uses bracket notation.
\usepackage[linguistics]{forest}	% Another syntax tree drawing software. Uses bracket notation.
% \forestset{ default preamble={for tree={s sep=10mm, inner sep=0, l=0} }
% \usepackage{ling-macros} % Various linguistic macros. Does not work with linguex.
% \usepackage{covington} % Another linguistic examples package.
\usepackage{leipzig} %	Offers support for Leipzig Glossing Rules

%%TITLE INFORMATION
\title{Motivations for Scandinavian Negative Indefinite Shift}
\author{Mykel Loren Brinkerhoff}
\date{\today}

%%MACROS
\newcommand{\sub}[1]{\textsubscript{#1}}
\newcommand{\supr}[1]{\textsuperscript{#1}}

%%TIKZ PREAMBLE STUFF 
% \tikzset{exarrows/.style={semithick,
% 						arrows={-Stealth[scale=1, scale length=1,
% 							 			scale width=1]}}}


\begin{document}
	
% %If using linguex, need the following commands to get correct LSA style spacing
% % these have to be after  \begin{document}
% 	\setlength{\Extopsep}{6pt}
% 	\setlength{\Exlabelsep}{9pt}		%effect of 0.4in indent from left text edge
% %
	
%% Line spacing setting. Comment out the line spacing you do not need. Comment out all if you want single spacing.
	% \doublespacing
	\onehalfspacing
	
\begin{center}
	{\Large \textbf{Motivations for Scandinavian Negative Indefinite Shift}}
	\vspace{6pt}

	Mykel Loren Brinkerhoff
	\vspace{6pt}
\end{center}
% \maketitle
%\maketitleinst
% \thispagestyle{plain}

% \tableofcontents

%------------------------------------
\section{Introduction} \label{sec:INTRO}
%------------------------------------
Negative Shifting (NegShift) is a process in the Scandinavian languages where a negative indefinite expression (NI) obligatorily shifts to a position outside of the VP. The Danish examples in (\ref{ex:PRIMA}) show that the NI pronoun \textit{ingenting} `nothing' and the complex DP \textit{ingen bøger} `no books' both shift across the verb into a position that is higher than the VP, but lower than certain high adverbials. 
	\ea NegShift of pronouns and complex DPs\label{ex:PRIMA}
		\ea {
		\gll Manden havde måske \textit{ingenting} [\textsubscript{VP} sagt t\textsubscript{o} ].\\
		man-the had probably nothing ~ said\\} \jambox*{Da}
		\glt `The man hadn't said anything.'
		\ex 
		\gll Jeg har \textit{ingen} \textit{bøger} [\textsubscript{VP} lånt børnene t\textsubscript{o}.]\\
		I have no books ~ lent children-the\\
		\glt `I haven't lent the children any books.'
		\z
	\z 

NegShift bears some resemblance to the more widely studied Scandinavian Object Shift (OS), which causes a weak pronoun to shift to a position outside of the VP but only when there is no phonologically visible category in  the VP \citep{holmbergWordOrderSyntactic1986,holmbergRemarksHolmbergGeneralization1999}.
	\ea {
	\gll Jag kyssade\textsubscript{v} henne\textsubscript{o} inte [\textsubscript{VP} t\textsubscript{v} t\textsubscript{o} ] \\
	I kiss.\Pst{} her \Neg{}\\} \jambox*{Sw}
	\glt `I didn't kiss her.'   
	\z  
Interestingly, several authors have claimed that Scandinavian OS is conditioned by prosodic factors (see \cite{erteschik-shirSoundPatternsSyntax2005,erteschik-shirScandinavianObjectShift2017,erteschik-shirVariationMainlandScandinavian2020,brinkerhoffMATCHINGPhrasesNorwegian2021} for some recent accounts). However, there are others that claim that OS is best accounted for as syntactic movement to satisfy PF, information structure, or some other syntactic requirement (\cite{holmbergRemarksHolmbergGeneralization1999,thrainssonObjectShiftScrambling2001,foxCyclicLinearizationSyntactic2005,bentzenObjectShiftSpoken2013}, among many others). 

However, it is clear that not all instances of NegShift directly correlate to the accounts of OS. One of the chief reasons for this difference is due to the wider range of material that is allowed to undergo NegShift, which includes both pronouns and full DPs, whereas only prosodically weak object pronouns are allowed to undergo OS. Further discussion about the similarities and differences between OS and NegShift is found in Section \ref{sec:DISTRIBUTION}. Even though there are differences, the one thing that could unite NegShift and OS is their shared movement of pronouns. There have been several claims that NegShift has a preference for shifting ``lighter" NIs over ``heavier" ones \citep{christensenInterfacesNegationSyntax2005,penkaNegativeIndefinites2011}. This preference for ``lighter'' NIs has the effect that speakers prefer shifting pronouns and small DPs over more complex DPs, see Section \ref{sec:PROSODY} for a more detailed discussion. 

These observations about movement and the status of ``lighter'' NIs being more grammatical in NegShift raises several questions: (i) what are the different triggers for NegShift; (ii) in what ways can prosody interact with syntactic movement; and (iii) does prosody play a role in NegShift in Scandinavian languages? I propose that the answers and solutions to these questions lie in \posscitet{zeijlstraSyntacticallyComplexStatus2011} account of the compositional nature of NIs with the addition of specific prosodic restrictions that exist for the \emph{Mittelfeld}. These \emph{Mittelfeld} restrictions constrain the prosodic size of elements so only prosodic units of a word or smaller are allowed to occupy the \emph{Mittelfeld}. 

The rest of this paper is organized as follows. Section \ref{sec:DISTRIBUTION} explains the basic patterns of both OS and NegShift. Section \ref{sec:NEG-OS} makes a detailed comparison between OS and NegShift in three different criteria: (i) Holmberg's Generalization, (ii) landing sites, and (iii) how OS and NegShift interact. These criteria demonstrate that OS and NegShift are not identical. Section \ref{sec:DERIVING} demonstrates how \posscitet{zeijlstraSyntacticallyComplexStatus2011} account holds for NegShift with the addition of the Light Mittelfeld Condition, discussed in Section \ref{sec:PROSODY}, which stipulates that only objects of a certain prosodic size are allowed to occupy the \emph{Mittelfeld}. Additional evidence for the Light Mittelfeld Condition is found in the size of material that is allowed to undergo particle shift in the different Scandinavian languages. Section \ref{sec:CONCLUSION} concludes this paper.

%------------------------------------
\section{Distributional properties of NegShift versus OS} \label{sec:DISTRIBUTION}
%------------------------------------

As mentioned above there are certain patterns that NegShift and OS both share and differ in. Both OS and NegShift involve the movement of elements from their base position to a position that is to the left of the VP, as seen by the movement across negation/adverbials in the case of OS, (\ref{ex:OS}), and across the verb in the case of NegShift, (\ref{ex:NS}). Additionally, they are similar in that they operate on pronouns. 
	\ea Distributional similarities between OS and NegShift. 
		\ea \label{ex:OS} {
		\gll Jag kyssade\textsubscript{v} \emph{henne\textsubscript{o}} inte [\textsubscript{VP} t\textsubscript{v} t\textsubscript{o} ] \\
		I kiss.\Pst{} her \Neg{}\\} \jambox*{Sw}
		\glt `I didn't kiss her.'
		\ex \label{ex:NS}
		\gll Jag har \emph{ingen\textsubscript{o}} [\textsubscript{VP} kyssat t\textsubscript{o} ]\\
		I have no-one ~ kiss.\Pst{}.\Ptcp{} \\
		\glt `I haven't kissed anyone.'
		\z 
	\z 

However, two main differences exist between OS and NegShift. First; NegShift applies to full negative DPs such as \textit{inga böcker} `no books' in addition to pronouns. Second; NegShift \emph{is not} subject to Holmberg's Generalization but is instead subject to an ``Anti-Holmberg Effect" where it can shift across phonological material, (\ref{ex:NS}), whereas OS is subject to Holmberg's Generalization. 

Holmberg's Generalization states that ``[OS] cannot apply across a phonologically visible category asymmetrically c-commanding the object position except adjuncts" \citep[15]{holmbergRemarksHolmbergGeneralization1999}. This means that OS can only occur if there is no phonological material, except adjuncts, between its base position and the position to which it shifts, as seen in the following example when compared to (\ref{ex:OS}). 
	\ea[*]  
		{\gll Jag har henne\textsubscript{o} [\textsubscript{VP} kyssat t\textsubscript{o} ]\\
				I have her ~ kiss.\Pst{}.\Ptcp{} \\}
		\glt intended: `I kissed her.' \label{ex:OS-Ungrammatical}
	\z 
In contrast to OS, NegShift only applies if the verb has not moved out of the VP, as seen in (\ref{ex:NS}) and (\ref{ex:ShiftingAcross}). 
	\ea \label{ex:ShiftingAcross}
		\ea
		{\gll Jens har \textit{ingen} \emph{hunder} [\sub{VP} \textbf{sluppet} t\sub{o}	ud].\\
			Jens has no dogs ~ let ~ out\\}\jambox*{Da}
		\glt `I haven't seen anybody.'
		\ex[*] {Jens har [\sub{VP} \textbf{sluppet} \textit{ingen hunder} ud].}
		% \ex \label{ex:String} 
		% {\gll Jag har \textit{ingen} \emph{bøger} [\sub{VP} \textbf{lånt} børnene t\sub{o}  ].\\
		% I have no books ~ lent children-the \\}\jambox*{WJ/Scan1}
		% \glt `I haven't lent the children any books.'
		\z
	\z 

Evidence collected and reported in \citet{engelsMicrovariationObjectPositions2011,engelsScandinavianNegativeIndefinites2012} shows that NegShift is in fact more complicated and subject to greater variability than was previously thought. \citeauthor{engelsScandinavianNegativeIndefinites2012} found that NegShift was permissible from a greater number of contexts and was more likely to occur depending on the variety and register that was being used, which is summarized in Table \ref{tab:Distribution} taken from \citet{engelsScandinavianNegativeIndefinites2012}. In Table \ref{tab:Distribution}, ✔︎ indicates that NegShift occurs, * indicates that NegShift cannot occur, and ? means that there was idiosyncratic variation between speakers.
\begin{table}[!ht]
	\centering
	\caption{Distribution of NegShift across Scandinavian languages. (WJ = West Jutlandic, Ic = Icelandic, Fa = Faroese, DaL = Danish Linguists, SwL = Swedish Linguists, Scan1 = literary/formal Mainland Scandinavian, Scan2 = colloquial Mainland Scandinavian and Norwegian)}
	\label{tab:Distribution}
\begin{tabular}{llccccccccc}
	\hline 
	NegShift across &  & WJ1 & WJ2 & Ic & Fa & DaL1 & DaL2 & SwL & Scan1 & Scan2 \\ 
	\hline 
	String-vacuous &  & ✔︎ & ✔︎ & ✔︎ & ✔︎ & ✔︎ & ✔︎ & ✔︎ & ✔︎ & ✔︎ \\ 
	Verb &  & ✔︎ & ✔︎ & ✔︎ & ✔︎ & ✔︎ & ✔︎ & ✔︎ & ✔︎ & * \\ 
	IO & verb in situ & ✔︎ & ✔︎ & ✔︎ & ✔︎ & ✔︎ & ✔︎ & ✔︎ & ✔︎ & * \\ 
	& verb moved & * & * & * & * & * & * & * & * & * \\ 
	Preposition & verb in situ & ✔︎ & ✔︎ & ✔︎ & ✔︎ & ? & ? & * & * & * \\ 
	& verb moved & ✔︎ & ✔︎ & ? & * & * & * & * & * & * \\ 
	Infinitive & verb in situ & ✔︎ & ✔︎ & ✔︎ & ✔︎ & ✔︎ & * & ? & * & * \\ 
	& verb moved & ✔︎ & * & * & ✔︎ & * & * & * & * & * \\ 
	\hline 
\end{tabular} 
\end{table}


Of crucial interest to our discussion are the rows showing the behavior of NegShift with respect to crossing a verb and crossing an indirect object.  The reason these rows are of interest lies in how NegShift behaves with these syntactic objects when compared with OS. This comparison is carried out in the following section.

%------------------------------------
\section{Comparison of NegShift and OS} \label{sec:NEG-OS}
%------------------------------------
We can compare certain patterns that NegShift has against patterns that OS has. In so doing we can determine whether they are governed by the same or different factors. If OS and NegShift are derived by the same trigger then we expect them to behave the similarly to one another. If they are not similar then this suggests that they are not derived by the same trigger. We will see that NegShift does not behave the same as OS.

There are three different metrics that we can use to determine the governing factors. These factors will examine: (i) how obedient to Holmberg's Generalization are NegShift and OS; (ii) where the loci of movement are; and (iii) how NegShift and OS interact with one another.   

I show that these two phenomena are in fact different with NegShift being governed and derived by syntactic factors and not prosodic as described for OS in \citet{erteschik-shirSoundPatternsSyntax2005,erteschik-shirScandinavianObjectShift2017,erteschik-shirVariationMainlandScandinavian2020,brinkerhoffMATCHINGPhrasesNorwegian2021}.  

%------------------------------------
\subsection{Homlberg's Generalization} \label{sec:HG}
%------------------------------------

The most defining characteristic for OS is its adherence to Holmberg's Generalization, which is defined formally in (\ref{ex:HG}).  
\ea \label{ex:HG} {Holmberg's Generalization:\\
Object Shift cannot apply across a phonologically visible category asymmetrically c-commanding the object position except adjuncts} \jambox*{\citep[15]{holmbergRemarksHolmbergGeneralization1999}}
\z
This generalization captures the fact that OS is only possible if the verb has evacuated the VP and there is no phonologically visible category present, except adverbials. 

\ea Examples of OS 
 	\ea {
	\gll Peter så\textsubscript{v} \textit{ham\textsubscript{o}} ikke [\textsubscript{VP} t\textsubscript{v} \textbf{t\textsubscript{o}}]\\
    Peter see.\Pst{} him not \\}\jambox*{(Across negation)}
    \glt `Peter didn't see him.' \label{ex:classic}
    \ex {
    \gll Peter så\textsubscript{v} \textit{ham\textsubscript{o}} ofte [\textsubscript{VP} t\textsubscript{v} \textbf{t\textsubscript{o}}]\\
    Peter see.\Pst{} him often \\}\jambox*{(Across adverbials)} 
    \glt `Peter often saw him.' \label{ex:oft}
	\z 
\z 
If, however, there is anything in the VP then OS is blocked and the object pronoun must remain in situ. Note that there is some variability when it comes to how OS interacts with verbal particles, which remain low. Verbal particles will be discussed in more detail in Section \ref{sec:Particles}.

\ea Blocking by verb
	\ea {
	\gll Peter har ikke [\sub{VP} \textbf{set} \textit{ham} ].\\
	Peter has not ~ seen him\\}\jambox*{Da}
	\glt `Peter didn't see him'
	\ex[*]{ 
	\gll Peter har \textit{ham\sub{o}} ikke [\sub{VP} \textbf{set} \textbf{t\sub{o}}] \\
	Peter has him not ~ seen \\}
	\glt Intended: `Peter didn't see him'
	\z

\ex Blocking by object
	\ea {
	\gll Peter lånte\sub{v} ikke [\sub{VP}  \textbf{Tore\sub{io}} t\sub{v} \textit{dem\sub{o}}]\\
    Peter lend.\textsc{past} not ~ Tore ~ them \\}\jambox*{No}
	\glt `Peter didn't lend them to Thor.'
	\ex[*]{
	\gll Peter lånte\sub{v} \textit{dem\sub{o}} ikke [\sub{VP}  \textbf{Tore\sub{io}} t\sub{v} \textbf{t\sub{o}}] \\
    Peter lend.\textsc{past} them not ~ Tore \\}
	\glt Intended: `Peter didn't lend them to Thor.'
	\z 
\z 


As mentioned at the end of Section \ref{sec:DISTRIBUTION}, NegShift seems to be the polar opposite of OS and generally requires the verb to have remained in situ, which is evidenced by the overwhelming acceptability of the NegShift when the verb remains in situ. 

\ea Examples of NegShift with the verb in situ. 
	\ea {
	\gll Manden havde måske \textit{ingenting}\sub{o} [\sub{VP} \textbf{sagt} t\textsubscript{o}].\\
	man-the had probably nothing {} said\\}\jambox*{Da}
	\glt `The man hadn't said anything'
	\ex[*] {Manden havde måske [\sub{VP} \textbf{sagt} \textit{ingenting}\sub{o}].}
	\z  
\z 

NegShift is also insensitive to any phonologically visible category, which results in NegShift occurring regardless of the amount of phonological material present in the VP. We see this with NegShift being acceptable when it crosses an indirect object. 

\ea NegShift across indirect objects. 
	\ea {
	\gll Jeg har \textit{ingen} \textit{bøger} [\sub{VP} \textbf{lånt} \textbf{børnene} t\sub{o}].\\
	I have no books {} lent children-the\\}\jambox*{WJ/Scan1}
	\glt `I haven't lent the children any books.'
	\ex[*] {Jeg har [\sub{VP} \textbf{lånt} \textbf{børnene} \textit{ingen bøger}]}
	\z 
\z 

We see that NegShift does not obey Holmberg's Generalization, whereas OS does. In fact, NegShift's generalization seems to be the exact opposite of the Holmberg's Generalization. This behavior that NegShift exhibits in requiring verbs to remain in situ and being able to cross phonologically visible material has lead some authors to call this the \emph{Anti-Holmberg Effect} \citep{foxCyclicLinearizationSyntactic2005,engelsScandinavianNegativeIndefinites2012}. 

%------------------------------------
\subsection{Landing sites} \label{sec:LANDING}
%------------------------------------

We further observe a difference between OS and NegShift as it relates to where the moved elements land. In terms of OS, it is always located to a position higher than all of the adverbials.  
\ea {
	\gll Jeg lånte \emph{ham\sub{IO}} faktisk\sub{Adv} aldri\sub{Adv} [\sub{VP} t\sub{IO} bøgerne].\\
	I lent him actually never ~ ~ books-the\\}\jambox*{No}
	\glt `I actually never lent him the books' 
\z 

This movement always results in the verb and weak pronoun being adjacent, after linearization. This behavior lead \citet{erteschik-shirSoundPatternsSyntax2005} to claim that the phonologically weak pronouns incorporate into the verb prior to the verb movement to C, indicated by the dashed line in (\ref{ex:incorporation}). This incorporation causes the weak object pronoun to appear in a position higher than the adverbials, which reside in a hierarchy of functional heads between TP and \emph{v}P \citep{cinqueAdverbsFunctionalHeads1999}. 
\ea \label{ex:incorporation}
\resizebox*{0.8\linewidth}{!}{
\begin{forest}
for tree={s sep=5mm, inner sep=0, l=0}
	[CP [DP\\\emph{Jeg}] 
		[C´ [C [T [\emph{v} [V\\\emph{så{=}ham},name=C ] [\emph{v}]] [T]] [C] ]
			[TP [<DP>\\\emph{Jeg}]
				[T´ [<T> [\emph{v} [V\\\emph{så{=}ham},name=T ] [\emph{v}]] [T]]
					[EpistP [Epist\\\emph{faktisk}]
						[NegP [Neg\\\emph{aldri}]
							[\emph{v}P [<DP>\\\emph{Jeg}]
								[\emph{v´} [<\emph{v}> [V\\\emph{så{=}ham},name=v ] [\emph{v}]]
									[VP [<V>\\\emph{så},name=verb]
										[<DP>\\\emph{ham},name=pro]
									]
								]
							]
						]
					]
				]
			]	
		]
	]
\draw[dashed,<->] (pro) to[out=south,in=south east] (verb);
\draw[->] (verb) to[out=south, in=south east] (v);
\draw[->] (v) to[out=south, in=south east] (T);
\draw[->] (T) to[out=south, in=south east] (C);
\end{forest}}
\z 

More recent accounts \citep[e.g.,][]{erteschik-shirVariationMainlandScandinavian2020,brinkerhoffMATCHINGPhrasesNorwegian2021} assume that OS is entirely conditioned on phonological phrasing. \citet{brinkerhoffMATCHINGPhrasesNorwegian2021} argue that OS is the result of pressures from \textsc{Match}-style prosodic constraints \citep{selkirkClauseIntonationalPhrase2009,selkirkSyntaxPhonologyInterface2011} that prevent the pronoun from forming a phonological phrase with the adverb. This results in the pronoun leaving the adverb's phonological phrase and forming a phonological phrase with the verb and subject where it is then allowed to incorporate into the verb because of its status as a prosodically weak element, i.e., a clitic.\footnote{I am considering weak pronouns and clitics from \citet{cardinalettiTypologyStructuralDeficiency1999} as prosodic clitics, which has to do with their status as prosodically deficient elements (i.e., lacking stress). Two reasons why we should treat the object pronouns as clitics comes from: (i) syllabification of the clitic as part of the verb and (ii) the tonal accent of the verb overrides the tonal accent of the pronoun. See \citet{myrbergProsodicWordSwedish2013,myrbergProsodicHierarchySwedish2015,riadPhonologySwedish2014} for a detailed discussion into the facts about syllabification and word accents. } 
\newline
\ea Prosodic structure according to \citet{brinkerhoffMATCHINGPhrasesNorwegian2021}.\\
( (jeg\sub{\textsc{cl}}=så\sub{$\omega$}=ham\sub{\textsc{cl}})\sub{$\phi$} (faktisk\sub{$\omega$} aldri\sub{$\omega$})\sub{$\phi$} )\sub{$\iota$}
\z 

However, this is not the position that is observed for negation and NegShift in these languages. \citet{nilsenAdverbsAshift1997} shows that negation resides in a position between `high' adverbials, such as \emph{faktisk} `actually', and `low' adverbials, such as \emph{lenger} `any longer' and \emph{alltid} `always'. In the following examples from \citet{nilsenAdverbsAshift1997}, we see in (\ref{ex:low_adverbials}) that negation has to be above certain `low' adverbials.

\ea \label{ex:low_adverbials} Examples of the Neg above `low' adverbials. 
	\ea {
	\gll Jon aksepterer \emph{ikke} lenger\sub{\textsc{adv}} alltid\sub{\textsc{adv}} vår invitasjon.\\ 
	Jon accepts \Neg{} any.longer always our invitation\\}\jambox*{No}
	\glt `Jon no longer always accepts our invitation'
	\ex[*] {Jon aksepterer \emph{ikke}\sub{\textsc{neg}} alltid\sub{\textsc{adv}} lenger\sub{\textsc{adv}}  vår invitasjon. }
	\ex[*] {Jon aksepterer alltid\sub{\textsc{adv}} \emph{ikke}\sub{\textsc{neg}} lenger\sub{\textsc{adv}}  vår invitasjon.}
	\z 
\z
In \ref{ex:high_adverbials}, we see that negation needs to be in a position lower than `high' adverbials.
\ea \label{ex:high_adverbials} Examples of Neg below `high' adverbials
	\ea {
	\gll Per kommer {ærlig talt} heldigvis \emph{ikke} tilbake.\\
	Per comes frankly fortunately \Neg{} back.\\}\jambox*{No}
	\glt `Frankly, Per is fortunately not comming back.'
	\ex[*] {Per kommer heldigvis \emph{ikke} {ærlig talt} tilbake.}
	\ex[*] {Per kommer heldigvis {ærlig talt} \emph{ikke} tilbake.}
	\ex[*] {Per kommer {ærlig talt} \emph{ikke} heldigvis tilbake.}
	\z 
\z 
Finally, we see that negation needs to reside between the `high' and `low' adverbials. 
\ea Example of Neg between `high' and `low' adverbials\\
{\gll Per går klokelig vanligvis ikke lenger hjem før {klokka 8}.\\
Per goes wisely usually not any.longer home before {8 o'clock}\\}\jambox*{No}
\z 
Based on these observations, \citet{nilsenAdverbsAshift1997} convincingly shows how negation interacts with other adverbs in Norwegian. He also shows how this is in line with the Cinquean hierarchy of adverbs \citep{cinqueAdverbsFunctionalHeads1999}, suggesting that this position is identical with NegP.  

\citet{svenoniusStrainsNegationNorwegian2002} shows that the same positional restrictions for where the negative particle is located also exist for NegShifted objects. In (\ref{ex:negshift_lowadverbs}), we see that when a NI object, in this instance the negative PP \emph{på intet tidspunkt} `at no point in time', is shifted it must occur higher than `low' adverbs into NegP.  
\ea \label{ex:negshift_lowadverbs}
	\ea 
	{\gll Fænsene har \emph{på} \emph{intet} \emph{tidspunkt} \emph{tidlig} slått av TV’en.\\
	the.fans have on no time early turned off the.TV\\}\jambox*{No}
	\glt `The fans have at no time turned the TV off early’ 
	\ex[*] {
	\gll Fænsene har \emph{tidlig} \emph{på} \emph{intet} \emph{tidspunkt} slått av TV’en.\\
	the.fans have early on no time turned off the.TV\\
	}
	\z  
\z
The evidence that NegShift targets NegP comes from a combination of these positional facts and the fact that these shifted elements don't scope from the position of sentential negation. \citet{svenoniusStrainsNegationNorwegian2002} shows that if the NI remains in situ then it only was narrow scope and does not establish sentential negation. We see this in the following sentences where the NI is only establishing the negation of the NP and not the negation of the whole sentence. We observe this with using the `neither' tag in (\ref{ex:NEITHER}). The `neither' tag is a method of testing whether or not sentiental negation is occurring \citep{klimaNegationEnglish1964}. If the resulting sentence with the `neither' tag – or in our Norwegian case \emph{heller ikke} – is grammatical, `neither' shows that the original sentence without `neither' was not negated. If, however, the sentence is ungrammatical with the addition of `neither', then something is already supply negation to the sentence. Because (\ref{ex:NEITHER}) is ungrammatical with the `neither' tag, then something is already providing negation to the sentence, in this case it is the PP  \emph{ingen poeng}. 

\ea 
	\ea[*] {{
	\gll De har gitt Norge ingen poeng, og det har heller ikke irene.\\
	they have given Norway no points and that have either not Irish-the\\}\jambox*{No}}
	\glt (intended: ‘They have given Norway no points, and neither have the Irish) \label{ex:NEITHER}
	\ex 
	\gll De har gitt Norge ingen poeng, og det har også irerne.\\
	they have given Norway no points, and that have also Irish-the\\
	\glt `They have given Norway no points, and so have the Irish'
	\z 
\z  


We have seen that the final positions that NegShift and OS occupy are located in different positions. When OS occurs it is always to a position higher than all adverbials and always adjacent to the verb \citep{holmbergWordOrderSyntactic1986,holmbergRemarksHolmbergGeneralization1999,erteschik-shirSoundPatternsSyntax2005}. When NegShift occurs it is to a position that is between `high' and `low' adverbs, exactly in the position that is predicted for negation in Cinque's hierarchy of adverbs \citep{nilsenAdverbsAshift1997,svenoniusStrainsNegationNorwegian2002}. 

If NegShift and OS were governed by the same principles then it would stand to reason that they would both be moving to identical positions. We fail to observe NegShift and OS moving to identical locations. Instead what we observe is OS moving to a position higher than the adverbials and NegShift moving to a position that lies between the `high' and `low' adverbials, exactly to the position that negation takes, i.e. NegP. This difference between locations shows that OS and NegShift are not the same because the different positions that the moved elements take.

%------------------------------------
\subsection{NegShift and OS interactions} \label{sec:INTERACTION}
%------------------------------------

This observation that OS is higher than the adverbials and NegShift is to NegP in the middle of the adverbs predicts that when we have both OS and NegShift then we should always have NegShift lower than OS. In fact this is exactly what we see.
\begin{table}[!ht]
\centering
\caption{Summary of OS and NegShift accoriding to \citet{christensenInterfacesNegationSyntax2005}.}
\label{tab:OSNEGS}
\begin{tabular}{lccccc}
\hline
IO-DO & Ic & Da/Fa & No/Sw & No\textsuperscript{+}/Sw\textsuperscript{+} & FS  \\
\hline 
Pron-Pron	&	+ +	&	+ +	&	\% \%	&	§ §	&	- -	\\
Pron-NegQP	&	+ +	&	+ +	&	+ +	&	+ +	&	- -	\\
NegQP-Pron	&	+ -	&	+ -	&	+ -	&	+ -	&	- -	\\
Pron-DP	&	+ \%	&	+ -	&	\% -	&	\% -	&	- -	\\
DP-Pron	&	\% -	&	- -	&	- -	&	- -	&	- -	\\
DP-DP	&	\% \%	&	- -	&	- -	&	- -	&	- -	\\
DP-NegQP	&	\% \%	&	- -	&	- -	&	- -	&	- -	\\
NegQP-DP	&	+ -	&	+ -	&	+ -	&	+ -	&	- -	\\
\hline 
\end{tabular}\\
(KEY: + = obligatory, - = blocked, \% = optional, § = optional and `non-parallel’)
\end{table}

\citet{christensenInterfacesNegationSyntax2005} reports on this behavior between NegShift and sentential objects and summarizes his findings in Table \ref{tab:OSNEGS}. In this table No\textsuperscript{+}/Sw\textsuperscript{+} represent some varieties of Norwegian and Swedish in contrast to more standard Norwegian (No) and Swedish (Sw), FS represent the Swedish variety which is spoken by Swedes in Finland.

The sections on this table that are most interesting are those involving what \citeauthor{christensenInterfacesNegationSyntax2005} calls Negative Quantifier Phrases (equivalent to NIs). However, this does conflate NI determiners and NI pronouns into a single category. According to \citeauthor{christensenInterfacesNegationSyntax2005}, when the IO is a pronoun and the DO is a NegQP both obligatorily shift when the verb has been able to swift to C, (\ref{ex:HER}), otherwise only the NegQP shifts, (\ref{ex:BOOKS}).
	\ea 
		\ea \label{ex:HER}
		{\gll Jeg lånte \textit{hende(IO)} faktisk \textit{ingen} \textit{bøger(DO)}\\
		I lent her actually no books\\}\jambox*{Da}
		\glt `I actually didn't lend her any books'
		\ex \label{ex:BOOKS}
		\gll Jeg har \textit{ingen} \textit{bøger(DO)} lånt \textit{hende(IO)}\\
		I have no books lend her\\
		\glt `I didn't lend her any books'
		\z 
	\z 
If, however, the IO is a NegQP and the DO is a pronoun then the pronoun is blocked from shifting, producing a freezing effect on OS. 
	\ea Freezing effects on OS
		\ea 
		{\gll Jeg lånte faktisk \textit{ingen(IO)} \textit{den(DO)}\\
		I lent actually no-one it\\}\jambox*{Da}
		\glt `I actually lent it to no-one.'
		\ex[*] {Jeg lånte \textit{den(DO)} faktisk \textit{ingen(IO)}} 
		\z 
	\z

One explanation for this freezing effect is found in \posscitet{foxCyclicLinearizationSyntactic2005} account of ordering restrictions where the verb first moves to the edge of \emph{v}P which establishes the ordering that must occur through the remainder of the derivation. Another possibility has to due with Holmberg's Generalization. The fact that NegShift blocks OS from occurring suggests that whatever position the NI is occupies it must be in a phonologically visible position that asymmetrically c-commands the object pronoun. This question is actually a very important point for the prosodic nature of the shifting. OS takes a weak pronominal object and is subject to Holmberg's Generalization, which prevents OS if there is a phonologically visible category in the VP. The question arises why are there freezing effects for OS when NegShift has occurred? One way to understand this freezing effect is that it is not actually sensitive to the VP edge but instead sensitive to phase boundaries \citep{chomskyMinimalistInquiriesFramework2000,chomskyDerivationPhase2001,chomskyPhases2008}. It is commonly assumed that adverbs belong to the \emph{v}P phase and if we also believe that adverbs form this Cinquean hierarchy of functional heads then these phrases that form the functional heads also must belong to the same phase as the material residing in \emph{v}P. This results in having the moved NI asymmetrically c-commanding the weak pronominal object (see discussion in \cite{brinkerhoffMATCHINGPhrasesNorwegian2021} about discussion why adverbs do not present problems for OS). We will see in Section \ref{sec:PROSODY} that this notion of a phase boundary encompassing both the NI in NegP and its base generated position in \emph{v}P will be important for explaining the prosodic restrictions on NegShift that we do.

An additional case involving OS and NegShift is when both the indirect and direct objects are allowed to shift. \citet[417f]{broekhuisUnificationObjectShift2020} observes that weak pronominal object shift behaves differently than full DP objects in what loci they are allowed to inhabit. In the case of weak pronominals they are required to appear outside of the \textit{v}P if there is no intervening phonological material (i.e., Holmberg's Generalization \cite{holmbergWordOrderSyntactic1986,holmbergRemarksHolmbergGeneralization1999}). \citeauthor{broekhuisUnificationObjectShift2020} observes that when OS and NegShift are both allowed to occur they are subject to certain ordering restrictions. Citing examples from \citet[163ff]{christensenInterfacesNegationSyntax2005}, Broekhuis shows the following pair of examples:
	\ea 
		\ea \label{ex:NegShift}
		{\gll Jeg har <ingen bøger> lånt hende <*ingen bøger>.\\
		I have no books lent her\\}\jambox*{Da}
		\glt `I haven't lent her any books
		\ex \label{ex:NegOS}
		\gll Jeg lånte \textit{henda} fraktisch \textit{ingen} \textit{bøger}.\\
		I lent her actually no books\\
		\glt `I didn't actually lend her any books.'
		\z 
	\z
In (\ref{ex:NegShift}), we see that when we have a negative object it shifts to a position higher than the \textit{v}P. If it were to remain in-situ it would be ungrammatical and would require the use of \textit{ikke} `not' and the negative polarity item \textit{nogen} `any'.
	\ea
	{\gll Jeg har \textit{ikke} lånt hende \textit{nogen} bøger.\\
	I have not lent her any books.\\}\jambox*{Da}
	\glt `I haven't lent her any books.'
	\z
However, when the main verb has risen to C as in (\ref{ex:NegOS}) then the weak pronominal moves to a position higher than the adverb \textit{fraktisk} `actually'. Additionally, this results in OS > NegShift, which \citeauthor{broekhuisUnificationObjectShift2020} reports to be a universal. This does help us see that even though these two phenomena appear to be similar they are in fact slightly different, due to the differences in the where the two different movement operations' targets are.

%------------------------------------
\subsection{Interim Summary} \label{sec:SUMMARY}
%------------------------------------

We have seen that each of the different criteria that I laid out at the start of this section clearly show that the behavior of NegShift and OS are not the same in spite of superficial similarities in behavior. The question now arises as to how exactly does NegShift get derived if not by the same mechanisms as OS. Starting in Section \ref{sec:DERIVING}, I will lay out how exactly NegShift is derived. I show that \posscitet{zeijlstraSyntacticallyComplexStatus2011} account provides the clearest and simplest explanation of how NegShift is derived. I also show that \posscitet{zeijlstraSyntacticallyComplexStatus2011} account needs to be modified to take into account certain weight restrictions that NegShift exhibits.

%------------------------------------
\section{Deriving NegShift} \label{sec:DERIVING}
%------------------------------------

\citet{iatridouNegativeDPsAMovement2011} show that NIs exhibit split-scope, which is evident when modals and other auxiliaries are present and the negation scopes higher than the modal/auxiliary's scope where the indefinite resides. We can see this in the following example from \citet{iatridouNegativeDPsAMovement2011}. 

\ea There has to be no doctor present for the nurses to administer the medicine.\\
Neg-split: It isn’t required that a doctor be present
\z 

In this example the negation is allowed to scope higher than the auxiliary \emph{has to} while \emph{doctor} resides within the scope of this auxiliary. This is what allows for the reading that it isn't necessary for a doctor to be present for nurses to administer medicine. 

\citet{zeijlstraSyntacticallyComplexStatus2011} takes these observations in English, as well as this same behavior in Dutch and German, and provides an analysis of the split-scope interpretation that exists for negative indefinites in these languages.  \citeauthor{zeijlstraSyntacticallyComplexStatus2011} assumes that this behavior is the result of the compositional status of negative indefinites. He claims that NIs are composed of a negative operator and an indefinite component, (\ref{ex:NIComposition}).
\ea \label{ex:NIComposition} NI composition according to \citet{zeijlstraSyntacticallyComplexStatus2011}.\\
$\begin{bmatrix}
\begin{forest}
	[	[$Op_\neg$]
		[$\exists$]
	]
\end{forest}
\end{bmatrix}$ $\Leftrightarrow$  /ingen/
\z 
In order for the sentence to receive negation this $Op_\neg$ needs to be located within NegP otherwise the sentence fails to receive negation. The fact that NIs have split-scope where negation scopes wide suggests that this $Op_\neg$ ultimately does reside in NegP. \citeauthor{zeijlstraSyntacticallyComplexStatus2011} accounts for this behavior by drawing on the copy-theory of movement \citep{chomskyMinimalistProgramLinguistic1993}. 

Following \citeauthor{chomskyMinimalistProgramLinguistic1993}, movement is not a unique operation that literally moves one element to a position higher in the syntax. Instead, movement is the product of copying a constituent and then remerging it into the syntactic structure. This then results in two or more instances of the constituent, as seen in the syntactic structure for (\ref{ex:Copy}). 

We see as we proceed along the derivation that the verb is first copied and then merged with \emph{v} leaving behind a complete copy of itself in its base-generated position. This operation of copying and then remerging is then continued until we arrive at spell-out. This movement is always driven by the need to value features. In the case of the the NI moving to Spec,NegP it is to value the [{$u$\textsc{Neg}}] feature in NegP with the $Op_\neg$ which bears an interpretable negation feature. After moving to value the feature at PF spell out only the highest copy will be pronounced, this is represented by the lower copies having a strike through them. 
% \pagebreak
\ea \label{ex:Copy}
\resizebox*{0.7\linewidth}{!}{
\begin{forest}
for tree={s sep=5mm, inner sep=0, l=0}
	[CP [DP\\\emph{Jeg},name=Sloci] 
		[C´ [C [T\\\emph{har},name=AUXtgt ] [C] ] 
			[TP [<DP>\\\sout{\emph{Jeg}},name=Stgt ] 
				[T´ [<T>\\\sout{\emph{har}},name=AUXsrc ] 
					[NegP [DP [\emph{ingen bøger}\\{[\textsc{Neg}]}, roof,name=NItgt] ]
						[Neg´ [Neg\\\emph{Op$_\neg$}\\{[\sout{$u$\textsc{Neg}}]}] 
							[\emph{v}P [<DP>\\\sout{\emph{Jeg}},name=Ssrc] 
								[\emph{v´} [\emph{v} [V\\\emph{lånt},name=tgt] [\emph{v}] ] 
									[VP [DP\\\emph{børnene}] 
										[V´ [<V>\\\emph{\sout{lånt}},name=src] 
											[<DP> [\sout{\emph{ingen bøger}}\\{[\textsc{Neg}]}, roof,name=NIsrc] ]
										]	
									] 
								] 
							] 
						] 
					] 
				] 
			] 
		] 
	] 
\draw[->] (NIsrc) to[out=south, in=south] (NItgt);
\end{forest}}
\z 

If we assume that \citet{zeijlstraSyntacticallyComplexStatus2011} is correct in that negative indefinites exist in their base-generated position and in this higher copy because of feature-checking and for interpretation at LF then we expect that the higher copy is the only one that is ever pronounced.

%------------------------------------
\subsection{Alternative Account} \label{sec:alternative}
%------------------------------------

An alternative to the theory presented above comes from Cyclic Linearization, which is a theory that was developed by \citet{foxCyclicLinearizationSyntactic2005} as a way to account for OS and Holmberg's Generalization. This theory works by stipulating that spell-out of the morphosyntax is cyclic and order preserving, which means that as you spell-out each successive spell-out domain (i.e., phases) you need to ensure that whatever linearization existed when that domain was spelled-out persist at the next spell-out domain's ordering restrictions. This ensures that at the end of the derivation the verb and the object pronoun have the correct linearization where the object pronouns are always to the right of the verb. This theory also had the benefit of accounting for when OS was allowed or not allowed to occur based on whether or not these ordering restrictions between spell-out domains are the same or different.  

In order to account for OS, \citeauthor{foxCyclicLinearizationSyntactic2005} propose that during the spell-out of the VP domain the V is the leftmost element in its domain\footnote{The position of the V at the left-edge of the phase could be due to the movement of V to \textit{v} in which case it is actually the \textit{v}P that acts as the spell-out domain not the VP.} and at which point the ordering restrictions are in place which state that the V must precede the O. At this point the V moves to T and then to C which results in the object being free to move to its higher position because the order that existed at the VP domain continues to hold at the CP spell-out domain. 
\ea OS and string-vacuous Neg-Shift
% \vspace{6pt}
 	\ea%
 	\tikzstyle{every picture}+=[remember picture, inner sep=0pt, baseline, anchor=base]%
 	 ~[\sub{CP} S \tikz\node(B2){V}; … [\sub{NegP} \tikz\node(A2){O}; adv [\sub{VP} \tikz\node(B1){t\sub{v}}; \tikz\node(A1){t\sub{o}}; ]]]
 	 \begin{tikzpicture}[overlay]
	\draw [->] (A1.south) -- ++(south:1.5ex) -| (A2.south); 
	\draw [->, dashed] (B1.north) -- ++(north:1.5ex) -| (B2.north); 
	\end{tikzpicture}
	\ex VP Ordering: \textbf{V>O}\\
		CP Ordering: S>V, \textbf{V>O}, O>adv, adv>VP
	\z
\z

% \ea%
% \tikzstyle{every picture}+=[remember picture, inner sep=0pt, baseline, anchor=base]%
% {}[\textsubscript{CP} \tikz\node(wh2){Which}; did John say [\textsubscript{CP} PRO to move \tikz\node(wh1){\textunderscore}; ]]? 
% \begin{tikzpicture}[overlay]
% \draw [->] (wh1.south) -- ++(south:1.5ex) -| (wh2.south); 
% \draw [->, dashed] (wh1.north) -- ++(north:1.5ex) -| (wh2.north); 
% \end{tikzpicture}
% \z 

This proposal was extended by \citet{foxCyclicLinearizationSyntactic2005} and \citet{engelsMicrovariationObjectPositions2011,engelsScandinavianNegativeIndefinites2012} to account for quantifier movement (QM), of which NegShift is claimed to be a subset under their analyses. QM is subject to an ``Anti-Holmberg Effect'' or an ``Inverse Holmberg Effect''. As previously discussed above Holmberg's Generalization stipulates that OS can only apply if the verb has undergone movement from V-to-T-to-C. The Anti-Holmberg Effect explains that only when the verb remains in situ can we have QM, which is the result of the ordering operations between the different phases being in agreement. 

In the case of NegShift, where it is able to shift across various phonological material as opposed to OS which cannot shift across intervening phonological material, it is proposed that the NI first moves to the left edge of the VP before spell-out of that domain. Once that domain is spelled-out the NI is free to shift to its position outside of the VP, in the case of (\ref{ex:n}) this is to NegP.
	\ea NegShift when V is in-situ. \label{ex:n}
		\ea%
 	\tikzstyle{every picture}+=[remember picture, inner sep=0pt, baseline, anchor=base]%
 	 ~[\sub{CP} S aux … [\sub{NegP} \tikz\node(A3){O}; [\sub{VP} \tikz\node(A2){t\sub{o}};  V \tikz\node(A1){t\sub{o}}; ]]]
 	 \begin{tikzpicture}[overlay]
	\draw [->] (A1.south) -- ++(south:1.5ex) -| (A2.south); 
	\draw [->] (A2.south) -- ++(south:1.5ex) -| (A3.south); 
	\end{tikzpicture}
		\ex VP Ordering: \textbf{O>V}\\
		CP Ordering: S>V, aux>O, O>adv, adv>VP → \textbf{O>V}
		\z 
	\z

This account does seem to explain the basic pattern of shifting found for both OS and NegShift. However, one of the biggest issues for this account ultimately has to do with the requirement for NegShift and other quantifier movement to first move to Spec,VP before moving to its final position outside of the verb phrase. The reason that this `micro' movement is problematic is because there is no direct evidence that this movement occurs, other than being motivated by the theory. 

% [TODO:// ADD INFORMATION ABOUT BLEAMAN]
%------------------------------------
\subsection{Prosodic factors of NegShift} \label{sec:Confound}
%------------------------------------

In addition to the standard case of NegShift where you shift a pronoun or a NI DP, there appears to be size restrictions in place which prevent movement of the NI if it is too large. This is particularly true for the Danish examples in (\ref{ex:NIinteractions}). 
	\ea\label{ex:NIinteractions}
		\ea
		{\gll Jeg har \textit{intet}\textsubscript{o} hørt t\textsubscript{o}.\\
		I have nothing heard\\}}\jambox*{Da}
		\glt  `I havn't heard anything.'\label{ex:NIPro}
		\ex 
		\gll Jeg har [\textit{intet} \textit{nyt}]\textsubscript{o} hørt t\textsubscript{o}.\\
		I have nothing new heard\\
		\glt `I haven't heard anything new'\label{ex:NIDP}
		\ex[*] {
		\gll Jeg har [\textit{intet} \textit{nyt} \textit{i} \textit{sagen}]\textsubscript{o} hørt t\textsubscript{o}.\\
		I have nothing new in case-\textsc{det} heard\\
		\glt `I haven't heard anything new about the case.'\label{ex:NIHeavy}
		}
		\ex[*] {
		\gll Jeg har [\textit{intet} \textit{nyt} \textit{i} \textit{sagen} \textit{om} \textit{de} \textit{stjålne} \textit{malerier}]\textsubscript{o} hørt t\textsubscript{o}.\\
		I have nothing new in case-\textsc{det} about the stolen paintings heard\\
		}
		\glt `I haven't heard anything new in the case about the stolen paintings.'\label{ex:NISuper}
		\z 
	\z
Both (\ref{ex:NIPro}) and (\ref{ex:NIDP}) are grammatical because a pronoun (\emph{intet}) and simple DP (\emph{intet nyt}) have undergone NegShift. However, once the NI gets sufficiently heavy (e.g., the addition of the PPs \emph{i sagen} and \emph{i sagen om de stjålne malerier}) in (\ref{ex:NIHeavy}) and (\ref{ex:NISuper}) NegShift becomes ungrammatical. 

These facts require refinement of the proposal presented above. If we had, for example, the sentence \emph{Jeg har intet hørt i sagen om de stjålne malerier} `I haven't heard anything new in the case about the stolen painting' and tried to derive this tree we would have no way to explain this behavior. Following \citet{zeijlstraSyntacticallyComplexStatus2011} theory, we would assume that the DP containing the NI and its complement would be fully copied from its base-position and merged into NegP as in the tree in (\ref{ex:tree}). This would result in two copies of the NI expression. 
% \pagebreak
\ea	\label{ex:tree} \resizebox*{\linewidth}{!}{
	\begin{forest}
	for tree={s sep=10mm, inner sep=0, l=0}
	[CP [DP\\\emph{Jeg}] 
		[C´ [C [T\\\emph{har} ] [C] ] 
			[TP [<DP>\\\sout{\emph{Jeg}} ] 
				[T´ [<T>\\\sout{\emph{har}} ] 
					[NegP [DP [\emph{intet nyt i sagen om de stjålne malerier}, roof] ]
						[Neg´ [Neg] 
							[\emph{v}P [<DP>\\\sout{\emph{Jeg}}] 
								[\emph{v´} [\emph{v} [V\\\emph{hørt},name=tgt] [\emph{v}] ] 
									[VP [<V>\\\emph{\sout{hørt}},name=src] 
										[DP [\emph{intet nyt i sagen om de stjålne malerier}, roof] ]
									] 
								] 
							] 
						] 
					] 
				] 
			] 
		] 
	] 
	\end{forest}	
	}
\z 
At this point during PF, part of the higher copy is deleted leaving only the NI pronoun. In the lower copy that same pronoun is deleted. 
\ea \label{ex:tree-problem}\resizebox*{\linewidth}{!}{
\begin{forest}
for tree={s sep=10mm, inner sep=0, l=0}
	[CP [DP\\\emph{Jeg}] 
		[C´ [C [T\\\emph{har} ] [C] ] 
			[TP [<DP>\\\sout{\emph{Jeg}} ] 
				[T´ [<T>\\\sout{\emph{har}} ] 
					[NegP [DP [\emph{intet nyt \sout{i sagen om de stjålne malerier}}, roof] ]
						[Neg´ [Neg] 
							[\emph{v}P [<DP>\\\sout{\emph{Jeg}}] 
								[\emph{v´} [\emph{v} [V\\\emph{hørt},name=tgt] [\emph{v}] ] 
									[VP [<V>\\\emph{\sout{hørt}},name=src] 
										[DP [\emph{\sout{intet nyt} i sagen om de stjålne malerier}, roof] ]
									] 
								] 
							] 
						] 
					] 
				] 
			] 
		] 
	] 
\end{forest}	
}
\z 
If \citet{chomskyMinimalistProgramLinguistic1993} is correct and only the higher copy should be pronounced, then the question arises as to why only a part of the copy is deleted. According to \citet{fanselowRemarksEconomyPronunciation2001,fanselowDistributedDeletion2002} the motivation for when partial deletion is possible has its motivations in PF. They argue that this comes from constraints on whether the different formal features need a ``proper phonetic realization".\footnote{\citet{fanselowRemarksEconomyPronunciation2001,fanselowDistributedDeletion2002} do not go into detail about what is forcing which and where features receive their ``proper phonetic realization''. They say that the notion of partial deletion is analogous to partial reconstruction at LF and ``involve[s] (partial) reconstruction of phonetic material in the overt component''.} If this is prosodically motivated what drives this deletion?

%------------------------------------
\section{Prosodic restrictions on NegShift} \label{sec:PROSODY}
%------------------------------------

As noted earlier not all NegShift is treated equal. \citet[65f]{christensenInterfacesNegationSyntax2005}, speaking on Danish, claims that the ``weight" of the NI plays a crucial factor in whether or not NegShift occurs. 
	\ea \label{ex:weight}
		\ea \label{ex:weight-pronoun}
		{\gll Jeg har \textit{intet}\textsubscript{o} hørt t\textsubscript{o}.\\
		I have nothing heard\\}\jambox*{Da}
		\glt  `I havn't heard anything.'
		\ex \label{ex:weight-simple}
		\gll Jeg har [\textit{intet} \textit{nyt}]\textsubscript{o} hørt t\textsubscript{o}.\\
		I have nothing new heard\\
		\glt `I haven't heard anything new'
		\ex[*] {
		\gll Jeg har [\textit{intet} \textit{nyt} \textit{i} \textit{sagen}]\textsubscript{o} hørt t\textsubscript{o}.\\
		I have nothing new in case-\textsc{det} heard\\}
		\glt `I haven't heard anything new about the case.' \label{ex:weight-heavy}
		\ex[*] {
		\gll Jeg har [\textit{intet} \textit{nyt} \textit{i} \textit{sagen} \textit{om} \textit{de} \textit{stjålne} \textit{malerier}]\textsubscript{o} hørt t\textsubscript{o}.\\
		I have nothing new in case-\textsc{det} about the stolen paintings heard\\}
		\glt `I haven't heard anything new in the case about the stolen paintings.' \label{ex:weight-superheavy}
		\z 
	\z
We observe in (\ref{ex:weight}) that if the NI is either a pronoun (\ref{ex:weight-pronoun}) or a simple DP (\ref{ex:weight-simple}), consisting of just the NI determiner and noun, then Danish treats such constructions as grammatical. If, however, the NI is larger then a simple DP, as in (\ref{ex:weight-heavy}) and (\ref{ex:weight-superheavy}), it suddenly becomes ungrammatical. 

In those instances where the NI is too large there are two potential repair strategies. One option is to strand the PP which results in moving just the pronoun (\ref{ex:split}) or using the negative particle \textit{ikke} in NegP and an NPI in the lower position (\ref{ex:NPI}).\footnote{I have not found any evidence that \emph{intet nyt} is allowed to shift while stranding the PP. I found one example that shows the negative particle and an NPI. 
\ea
\gll Selvom der ikke er noget nyt i sagen om det stjålne Nolde-maleri, tror de stadig på miraklet i Ølstrup.\\
though there not is anything new in case-the about the stolen {Nolde painting} believe they still in miracle-the in Ølstrup\\
\glt `Although there is nothing new in the case of the stolen Nolde painting, they still believe in the miracle in Ølstrup.' \hfill (\href{https://www.tvmidtvest.dk/midt-og-vestjylland/overblik-storste-kunsttyverier-i-danmark}{Overblik: Største kunsttyverier i Danmark})
\z}
	\ea 
		\ea {Jeg har \textit{intet}\textsubscript{i} hørt t\textsubscript{i} [\textsubscript{PP} i sagen om de stjålne malerier ]. \label{ex:split}}\jambox*{Da}
		\ex Jeg har \textit{ikke} hørt [ \textit{noget} i sagen om de stjålne malerier ]. \label{ex:NPI}
		\z 
	\z  

Restrictions on the weight of the NI has also been observed for Swedish \citep{penkaNegativeIndefinites2011}. In (\ref{ex:swedish-weight}), which comes from \citet{penkaNegativeIndefinites2011}, when a pronoun is moved it is fully grammatical. When we, however, move a simple DP it is still grammatical but is dispreferred or degraded, indicated by the question mark. 
	\ea \label{ex:swedish-weight}
		\ea 
		{\gll Men mänskligheten har \textit{ingenting}\textsubscript{o} lärt sig t\textsubscript{o}.\\
		but mankind-the have nothing taught themselves\\}\jambox*{Sw}
		\glt `But mankind haven't taught themselves anything.'
		\ex[?] {
		\gll Vi hade \textit{inga} \textit{grottor}\textsubscript{o} undersökt t\textsubscript{o}.\\
		we have no caves explored\\}
		\glt `We haven't explored any caves.'
		\z 
	\z 

The question arises as to how exactly does the grammar account for these variations based on the NI's weight. The fact that the `weight' of the NI somehow conditions the grammaticality of the utterance suggests that prosody might be constraining the size of the moved material. One reason why to think that this is prosody has to do with the fact that only items that would normally form a prosodic word undergo NegShift. 

Assuming that Match Theory \citep{selkirkClauseIntonationalPhrase2009,selkirkSyntaxPhonologyInterface2011} is a correct way of understanding the relationship between syntax and prosody, there are several predictions that this theory makes when mapping the syntax to the prosody. Under Match Theory an XP corresponds to a phonological phrase unless that XP is both syntactically minimal and maximal, which is where the syntactic phrase consists of just its head. When this occurs it forms a prosodic word, which has to do with the non-branching nature of the of the XP and will only ever result in a prosodic word  \citep{bennettLightestRightApparently2016}. Additionally, if the XP is a CP then it forms a intonational phrase.
\begin{table}[!h]
\caption{Syntax-prosody category mappings (modified from \cite{tylerSimplifyingMATCHWORD2019})}
\label{tab:Mappings}
\centering
\begin{tabular}{ll}
\hline
\textbf{Syntactic}&\textbf{Prosodic}\\
\hline
CP & Intonational Phrase ($\iota$)\\
XP & Phonological Phrase ($\phi$) \\
X & Prosodic Word ($\omega$)\\
\hline
\end{tabular}
\end{table}  

Additionally, it is well documented that functional heads often cliticize into a prosodic head which is able to bear stress (\cite{zwickyClitics1977,selkirkProsodicStructureIts1981,zwickyCliticizationVsInflection1983,inkelasProsodicConstituencyLexicon1990} among others). Often this means that determiners and their NP complement form a single prosodic word. This means that our cases of pronouns and simple DPs correspond to prosodic words, explicitly maximal prosodic words ($\omega_{max}$) based on the fact that the determiners and the noun receive a single tonal accent in Swedish and Norwegian, which according to \posscitet{myrbergProsodicWordSwedish2013,myrbergProsodicHierarchySwedish2015,riadPhonologySwedish2014} claim is strictly the domain of $\omega_{max}$. Even though most Danish varieties lack tonal accents, they do have a something that patterns similarly to these tonal accents in the other Scandinavian languages. Stød is a suprasegmental unit found in Danish, which is a type of creakiness or glottal closure \citep{basbollPhonologyDanish2005}. Recent research shows that its distribution is restricted to only having one stød per maximal prosodic word \citep{kalivodaProsodicRecursionPseudocyclicity2018}. I show, in the rest of this section, that this weight restriction is regulated by a prosodic constraint on the size of material that is allowed to occupy the \emph{Mittelfeld} during PF, namely prosodic words. 

%------------------------------------
\subsection{PF deletion} \label{sec:NEXT}
%------------------------------------

Assuming that the copy theory of movement is correct then multiple copies of the NI will be present during PF spell-out \citep{chomskyMinimalistProgramLinguistic1993}. Under the standard view of this theory only the highest copy is pronounced in its entirety and the lower copy is deleted. \citet{kandybowiczGrammarRepetitionNupe2008} makes the observation that multiple copies that are generated by the narrow syntax can be \emph{phonologically} realized when pronouncing both copies results in avoiding an identifiable PF well-formedness condition. Similarly, I argue that PF can also dictate the amount of material that is deleted in those copies. PF being able to delete is not new and has been argued for previously \citep{chomskyTheoryPrinciplesParameters1993,merchantSyntaxSilenceSluicing2001,foxSuccessiveCyclicMovementIsland2003,ottDeletionClausalEllipsis2016}. 

If PF is responsible for deleting the material then the crucial question is how are the two copies coordinated. We will see that in regards to NegShift this has to do with both copies belonging to the same phase and the difference in what type of prosodic constituent is allowed to occupy the \emph{Mittelfeld} by the different Scandinavian languages. It is assumed under phase theory \citep{chomskyDerivationPhase2001,chomskyPhases2008} that material is spelled-out at certain stages of the derivation instead of waiting until the complete tree is ready to be spelled-out at PF. Additionally, it is assumed that once a phase is spelled-out that the material in those phases become unaccessible for further manipulation by the syntax in what is known as the Phase Impenetrability Condition (PIC). 

Assuming that phases play a crucial role in accounting for the NegShift there are two potential answers that arise: (i) phases do not play a role in determining phonological/prosodic behavior during PF, or (ii) Phases do play a role in determining phonological/prosodic behavior during PF.

Recent evidence from \citet{weberPhaseBasedConstraints2021} suggests that phases do in fact play a role in determining phonological behavior. The crucial evidence for \citeauthor{weberPhaseBasedConstraints2021} comes from vowel epenthesis in Blackfoot verbs. Verbs in Blackfoot are highly complex allowing for a large number of affixes and a single verb often corresponding to a CP. Citing internal vowel epenthesis, which only occurs if the root of the verb will begin with a stop, \citeauthor{weberPhaseBasedConstraints2021} shows that the location where epenthesis occurs corresponds with a phase boundary in the syntax. Assuming \citeauthor{weberPhaseBasedConstraints2021} and the PIC are correct, then it bears to reason that in order for the phonology to interact with two copies at the same time, the base generated position and the landing site both must belong to the same phase. This requires that NegP and the rest of the Cinquean hierarchy of adverbials \citep{cinqueAdverbsFunctionalHeads1999} belong to the same phase as \emph{v}P.

In the case of NegShift, there are restrictions on \emph{Mittelfeld} well-formedness, which I will call the \textsc{Light Mittelfeld Condition} (LMC), more specifically what is allowed to move into the \emph{Mittelfeld}. It was observed above that only a limited amount of structure is allowed and a wide degree of variation is permitted in the \emph{Mittelfeld} (see \cite{haiderMittelfeldPhenomenaScrambling2017}). I argue that the largest unit that is allowed to remain in the \emph{Mittelfeld} in Scandinavian is a maximal prosodic word (ω\sub{max}).

Evidence for this comes from the size of the material that is allowed to ``shift" in these languages. As observed for Danish only a pronoun or simple DP, which both correspond to maximal prosodic words, is allowed to occupy the \emph{Mittelfeld} when NegShift occurs. Additionally, we see that if the NI would form a phonological phrase, because it contains a PP complement, in the \emph{Mittelfeld} it is treated as ungrammatical. 
\ea 
	\ea 
		\gll Jeg har \textit{intet}\textsubscript{o} hørt t\textsubscript{o}.\\
		I have nothing heard\\
		\glt  `I havn't heard anything.'
	\ex 
		\gll Jeg har [\textit{intet} \textit{nyt}]\textsubscript{o} hørt t\textsubscript{o}.\\
		I have nothing new heard\\
		\glt `I haven't heard anything new'
	\ex[*] {
		\gll Jeg har [\textit{intet} \textit{nyt} \textit{i} \textit{sagen}]\textsubscript{o} hørt t\textsubscript{o}.\\
		I have nothing new in case-\textsc{det} heard\\}
		\glt `I haven't heard anything new about the case.'
	\ex[*] {
		\gll Jeg har [\textit{intet} \textit{nyt} \textit{i} \textit{sagen} \textit{om} \textit{de} \textit{stjålne} \textit{malerier}]\textsubscript{o} hørt t\textsubscript{o}.\\
		I have nothing new in case-\textsc{det} about the stolen paintings heard\\}
		\glt `I haven't heard anything new in the case about the stolen paintings.'
	\z  	
\z 

This difference between Danish, which allows simple DPs, and Swedish, which tolerates full DPs but prefers pronouns, suggests the LMC in Swedish will delete copies until they are just the NI pronoun. This potentially comes down to Swedish being a tonal language and how tonal accents are assigned to maximal prosodic words \citep{myrbergProsodicWordSwedish2013,myrbergProsodicHierarchySwedish2015,riadPhonologySwedish2014} and Danish not being a tonal language. However, this is not the case and has to do with the status of allowing compounds in the \emph{Mittelfeld}.

LMC allowing for the deletion of the offending material would explain what the correct size of the material being pronounced in the \emph{Mittelfeld} needs to be. Once the LMC has done its work the standard notion of PF spell-out will preserve the portion of the shifted NI that remains in the high copy while deleting that portion in the lower copy. This works if the LMC operates left-to-right after linearization has taken place but prior to full pronunciation. The difference that we observe with Norwegian's lack of NegShift is the result of one of two possibilities. First; we instead have covert movement in LF, which is due to the [$u$\textsc{Neg}] feature being weak and will get valued during LF, similar to what \citet{zeijlstraSyntacticallyComplexStatus2011} proposes for NIs in German, Dutch, and English. The second possibility is that there are even tighter prosodic restrictions on what is allowed to move into the \emph{Mittelfeld}, which results in no NI being able to move into the \emph{Mittelfeld}. Further research is needed to determine which of these principles are at work. 

This results in a three-way system in Scandinavian languages: (i) those that delete until a simple DP is left, (ii) those that delete until a pronoun is left, and (iii) those that lack overt movement in the syntax. There seems to be some differences in behavior between the tonal and atonal Scandinavian languages. According to \citet{thrainssonFaroeseOverviewReference2004,thrainssonSyntaxIcelandic2010} Faroese and Icelandic pattern the same as Danish in this regard. This further suggests that there is something unique about being a tonal language that limits the acceptability of NegShifting.

This is due to the fact that, in Swedish, NI determiners and NPs form single maximal prosodic words, as evidenced by the tonal melodies on the resulting prosodic word. In Swedish, when the NI and the NP coalesce, tonal accent 2 emerges. This accent corresponds with word compounds.\footnote{See discussion in \cite{myrbergProsodicWordSwedish2013,myrbergProsodicHierarchySwedish2015} for more information about word compounds in Swedish} This indicates that, in Swedish, these structures are not composed of a clitic and its host, but rather the structure in (\ref{ex:ProsodicStructure}), in which two minimal prosodic words combine to create a maximal prosodic word. Evidence for this comes from the fact that both the NI and the NP bear stress, which is a characteristic restricted to the minimal prosodic word \citep{myrbergProsodicWordSwedish2013,myrbergProsodicHierarchySwedish2015}. 
\ea \label{ex:ProsodicStructure}
\begin{forest}
	[$\omega_{max}$
		[$\omega_{min}$\\\emph{inga}] [$\omega_{min}$\\\emph{grottor}]
	]
\end{forest} 
\z 
It seems, then, that Swedish prefers having only minimal prosodic words, rather than maximal prosodic words in the \emph{Mittelfeld}. The reason that NI pronouns do not present a problem in Swedish is because they are both maximal and \emph{minimal} prosodic words, which means that they do not contain any branching prosodic structure. This difference in behavior is the result of what is considered too much prosodic structure for the LMC to tolerate moving into the \emph{Mittelfeld}. If the prosodic structure contains any branching, then it is treated as a heavier element than one that doesn't contain any branching structure. However, this is only true for prosodic words because any prosodic structure higher than a prosodic word is not tolerated. This is evidenced by the fact that NIs that contain additional PP complements are treated as ungrammatical because these would correspond to phonological phrases. 

These facts about NegShift are summarized in Table \ref{tab:Paradigm}. It will be noted that if you allow full NI DPs then you also allow pronouns and complete deletion, which results in a negation particle and an NPI. If you allow NI pronouns then you allow a negation particle and an NPI. This is suggestive of a conditional hierarchy. 

\begin{table}[!ht]
	\centering
	\caption{Scandinavian acceptance of NegShift or NPI}
	\label{tab:Paradigm}
\begin{tabular}{lccc}
	\hline 
	& Full DPs & Pronouns & NPI\\
	\hline
	Icelandic & ✔︎ & ✔︎ & ✔︎ \\
	Faroese & ✔︎ & ✔︎ & ✔︎ \\
	Danish & ✔︎ & ✔︎ & ✔︎ \\
	Swedish & * & ✔︎ & ✔︎ \\
	Norwegian & * & * & ✔︎ \\
	\hline 
\end{tabular} 
\end{table}

Another possibility instead of the LMC controlling the size of material is that prosody is responsible for moving the NI to NegP. One such system comes from \citet{zubizarretaProsodyFocusWord1998}. In \citet{zubizarretaProsodyFocusWord1998}, she argues that certain word order restrictions that exist comes down to prosodic restrictions that directly manipulate the syntax. In her system there are prosodic features that the syntax will manipulate in order to insure that the the syntactic object obey certain prosodic factors. This view assumes that there is no distinct PF and LF branches and all that exists is the syntax. This means that any displacement can be caused by syntactic, semantic, or prosodic factors. 

One of the chief issues with this system, besides abandoning the Y-model, has to do with the phonologically/prosodically derived movement. The \citeauthor{zubizarretaProsodyFocusWord1998} system makes the prediction that any sort of later phonological restriction will directly operate on the narrow syntax in order to satisfy those restrictions. This means that such syntactic movements and operations should not produce interpretational effects. Instead what we see is that NegShift produces differences in interpretation and NegShift exhibits split-scope \citep{iatridouNegativeDPsAMovement2011,zeijlstraSyntacticallyComplexStatus2011} discussed above. This is something that the \citeauthor{zubizarretaProsodyFocusWord1998} system cannot account for. 

%------------------------------------
\subsection{Scandinavian particle shift} \label{sec:Particles}
%------------------------------------

If all the evidence we had for the LMC comes from the weight restrictions found in NegShift then it would be easy to dismiss such possibilities. However, independent evidence for weight restrictions on shifting full DPs into the \emph{Mittelfeld} is observed in particle shifting in Scandinavian languages. Following \citet[2]{holmbergRemarksHolmbergGeneralization1999} and \citet{faarlundSyntaxMainlandScandinavian2019} there is a difference in behavior between the different Scandinavian languages with what is allowed to shift across a verbal particle. Danish objects always precedes the verb particle. Norweigan, Icelandic, and Faroese are like English by shifting a pronoun across a particle and optionally for DPs. Swedish does not allow anything to shift across the particles (see \cite{erteschik-shirVariationMainlandScandinavian2020} for a potential explanation for why).
\ea \gllll Jeg skrev (nummeret/det) op (*nummeret/*det). \hfill (Danish)\\
		Jeg skrev (nummeret/det) opp (nummeret/*det). \hfill (Norwegian)\\
		Jag skrev (*numret/*det) upp (numret/det). \hfill (Swedish)\\
		I wrote (number-the/it) up (number-the/it)\\
\glt `I wrote the number/it down.'
\z 
It is interesting to note that this behavior, with respect to particle shift, is remarkable similar to what is allowed to undergo NegShift. In both cases we are dealing with pronouns and small DPs, again consisting of a determiner and a NP. In Danish, NegShift and particle shift both prefer moving pronouns and these small DPs and allowing them to occupy the \emph{Mittelfeld}. There is a difference, however, between Norwegian and Swedish. One would expect, based on the behavior of NegShift, that Swedish would prefer to undergo particle shift in the same fashion as Norwegian. This behavior, however, is not a problem as there are independent prosodic facts that contribute to why Swedish does not undergo particle shift. The reason has to do with particles forming tonal accent units (i.e., $\omega_{max}$) with whatever material is lower \citep[see]{erteschik-shirVariationMainlandScandinavian2020}. The fact that we still see a pattern that is reminiscent of Swedish NegShift and Norwegian NegShift is what is important to show that prosodically light elements are preferred in the \emph{Mittelfeld}, exactly as predicted by the LMC. 

Another piece of evidence in favor for the LMC comes from how Danish places restriction on the verbal complement if it is too ``heavy" \citep[44f]{mullerDanishHeadDrivenPhraseInpreparation}. If the complement is larger then a simple DP particle shifting is blocked, exactly as was observed with Danish NegShift.\footnote{Examples are from \emph{KorpusDK} as reported by \citet{mullerDanishHeadDrivenPhraseInpreparation}.} 
	\ea \gll {[…]} så må partiet melde [holdninger] [ud], {[…]}\\
	~ then must party.\Def{} make stances out\\
	\glt `{[…]} then the party must make its stances clear, {[…]}'
	\ex \gll Den danske regering bør snart melde [ud], [at den støtter de amerikanske planer]\\
	the Danish government must soon make out that it supports the American plans\\
	\glt `The Danish government must soon make clear that it supports the American plans.'
	\z 
This is further evidence that the LMC is an active constraint in these languages. Prosody, in the form of the \textsc{Light Mittelfeld Condition}, plays an active role in regulating the size of the material in the \emph{Mittelfeld}.

%------------------------------------
\section{Conclusion} \label{sec:CONCLUSION}
%------------------------------------

I have shown that NegShift is easily derived by assuming \posscitet{zeijlstraSyntacticallyComplexStatus2011} compositional nature of NIs. This system was able to account for the basic pattern when we had an NI pronoun or DP and account for the semantic effects of split-scope. This system was better able to account for the patterns than other theories such as Cyclic Linearization. 

This system needed to be refined in order to account for the the fact that there are restrictions on the ``weight'' of the shifted element. This was accounted for by stipulating the existence of the LMC, which prevents objects larger than a prosodic word from occupying the \emph{Mittelfeld}. Additional evidence for the LMC was observed in the behavior of particle shifting. 

%------------------------------------
%BIBLIOGRAPHY
%------------------------------------

%\singlespacing
%\nocite{*}
\printbibliography[heading=bibintoc]

\end{document} 