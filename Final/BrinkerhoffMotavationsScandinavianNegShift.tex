% !TEX TS-program = xelatex
% !TEX encoding = UTF-8 Unicode		

\documentclass[12pt, letterpaper]{article}

%%BIBLIOGRAPHY- This uses biber/biblatex to generate bibliographies according to the 
%%Unified Style Sheet for Linguistics
\usepackage[main=american, german]{babel}% Recommended
\usepackage{csquotes}% Recommended
\usepackage[backend=biber,
		style=unified,
		maxcitenames=3,
		maxbibnames=99,
		natbib,
		url=false]{biblatex}
\addbibresource{link_laptop.bib}
% \addbibresource{link_desktop.bib}
\setcounter{biburlnumpenalty}{100}  % allow URL breaks at numbers
%\setcounter{biburlucpenalty}{100}   % allow URL breaks at uppercase letters
%\setcounter{biburllcpenalty}{100}   % allow URL breaks at lowercase letters

%%TYPOLOGY
\usepackage[svgnames]{xcolor} % Specify colors by their 'svgnames', for a full list of all colors available see here: http://www.latextemplates.com/svgnames-colors
%\usepackage[compact]{titlesec}
%\titleformat{\section}[runin]{\normalfont\bfseries}{\thesection.}{.5em}{}[.]
%\titleformat{\subsection}[runin]{\normalfont\scshape}{\thesubsection}{.5em}{}[.]
\usepackage[hmargin=1in,vmargin=1in]{geometry}  %Margins          
\usepackage{graphicx}	%Inserting graphics, pictures, images 		
\usepackage{stackengine} %Package to allow text above or below other text, Also helpful for HG weights 
\usepackage{fontspec} %Selection of fonts must be ran in XeLaTeX
\usepackage{amssymb} %Math symbols
\usepackage{amsmath} % Mathematical enhancements for LaTeX
\usepackage{setspace} %Linespacing
\usepackage{multicol} %Multicolumn text
\usepackage{enumitem} %Allows for continuous numbering of lists over examples, etc.
\usepackage{multirow} %Useful for combining cells in tablesbrew 
\usepackage{hanging}
\usepackage{datetime2}
\usepackage{fancyhdr} %Allows for the 
\pagestyle{fancy}
\fancyhead[L]{\textit{Qualifying Paper}} 
\fancyhead[R]{\textit{\today}} 
\fancyfoot[L,R]{} 
\fancyfoot[C]{\thepage} 
\renewcommand{\headrulewidth}{0.4pt}
\setlength{\headheight}{14.5pt} % ...at least 14.49998pt
% \usepackage{fourier} % This allows for the use of certain wingdings like bombs, frowns, etc.
% \usepackage{fourier-orns} %More useful symbols like bombs and jolly-roger, mostly for OT
\usepackage[colorlinks,allcolors={black},urlcolor={blue}]{hyperref} %allows for hyperlinks and pdf bookmarks
% \usepackage{url} %allows for urls
% \def\UrlBreaks{\do\/\do-} %allows for urls to be broken up
\usepackage[normalem]{ulem} %strike out text. Handy for syntax
\usepackage{tcolorbox}


%%FONTS
\setmainfont{Libertinus Serif}
\setsansfont{Libertinus Sans}
\setmonofont[Scale=MatchLowercase]{Libertinus Mono}

%%PACKAGES FOR LINGUISTICS
%\usepackage{OTtablx} %Generating tableaux with using TIPA
% \usepackage[noipa]{OTtablx} % Use this one generating tableaux without using TIPA
% \usepackage[notipa]{ot-tableau} % Another tableau drawing packing use for posters.
% \usepackage{linguex} % Linguistic examples
% \usepackage{langsci-linguex} % Linguistic examples
\usepackage{langsci-gb4e} % Language Science Press' modification of gb4e
% \usepackage{langsci-avm} % Language Science Press' AVM package
\usepackage{tikz} % Drawing Hasse diagrams
% \usepackage{pst-asr} % Drawing autosegmental features
\usepackage{pstricks} % required for pst-asr, OTtablx, pst-jtree.
\usepackage{pst-jtree} 	% Syntax tree drawing software 
\usepackage{tikz-qtree}	% Another syntax tree drawing software. Uses bracket notation.
\usepackage[linguistics]{forest}	% Another syntax tree drawing software. Uses bracket notation.
% \usepackage{ling-macros} % Various linguistic macros. Does not work with linguex.
% \usepackage{covington} % Another linguistic examples package.
\usepackage{leipzig} %	Offers support for Leipzig Glossing Rules

%%TITLE INFORMATION
\title{TITLE}
\author{Mykel Loren Brinkerhoff}
\date{\today}

%%MACROS
\newcommand{\sub}[1]{\textsubscript{#1}}
\newcommand{\supr}[1]{\textsuperscript{#1}}

\begin{document}
	
%%If using linguex, need the following commands to get correct LSA style spacing
%% these have to be after  \begin{document}
	% \setlength{\Extopsep}{6pt}
	% \setlength{\Exlabelsep}{9pt}		%effect of 0.4in indent from left text edge
%%
	
%% Line spacing setting. Comment out the line spacing you do not need. Comment out all if you want single spacing.
	% \doublespacing
	\onehalfspacing
	
\begin{center}
	{\Large \textbf{Motivations for Scandinavian Negative Indefinite Shift}}
	\vspace{6pt}

	Mykel Loren Brinkerhoff
\end{center}
%\maketitle
%\maketitleinst
\thispagestyle{fancy}

% \tableofcontents

%------------------------------------
\section{Introduction} \label{sec:INTRO}
%------------------------------------
Negative Shifting (NegShift) is a process in the Scandinavian languages where a negative indefinite expression (NI) obligatorily shifts to a position outside of the VP. The Danish examples in (\ref{ex:PRIMA}) show that the NI pronoun \textit{ingenting} `nothing' and the complex DP \textit{ingen bøger} `no books' both shift across the verb into a position that is between the adverbials and the verb in the case of \textit{ingenting} and between the auxiliary and the verb in the case of \textit{ingen bøger}.
	\ea NegShift of pronouns and complex DPs\label{ex:PRIMA}
		\ea
		\gll Manden havde måske \textit{ingenting} [\textsubscript{VP} sagt t\textsubscript{o} ].\\
		man-the had probably nothing ~ said\\
		\glt `The man hadn't said anything.'
		\ex 
		\gll Jeg har \textit{ingen} \textit{bøger} [\textsubscript{VP} lånt børnene t\textsubscript{o}.]\\
		I have no books ~ lent children-the\\
		\glt `I haven't lent the children any books.'
		\z
	\z 

NegShift bears some resemblance to the greater studied Scandinavian Object Shift (OS), which causes a weak pronoun to shift to a position outside of the VP when the verb has raised for V2 \citep{holmbergWordOrderSyntactic1986,holmbergRemarksHolmbergGeneralization1999}.
	\ea {
	\gll Jag kyssade\textsubscript{v} henne\textsubscript{o} inte [\textsubscript{VP} t\textsubscript{v} t\textsubscript{o} ] \\
	I kiss.\Pst{} her \Neg{}\\} \jambox*{Sw}
	\glt `I didn't kiss her.'   
	\z  
Interestingly, several authors have claimed that Scandinavian OS is driven and determined by prosodic factors (see \cite{erteschik-shirSoundPatternsSyntax2005,erteschik-shirScandinavianObjectShift2017,erteschik-shirVariationMainlandScandinavian2020,brinkerhoffMATCHINGPhrasesNorwegian2021} for some recent accounts). However, there are others that claim that OS is best accounted for as syntactic movement to satisfy PF, information structure, or some other syntactic requirement (\cite{holmbergRemarksHolmbergGeneralization1999,thrainssonObjectShiftScrambling2001,bentzenObjectShiftSpoken2013}, among many others). 

However, it is clear that not all instances of NegShift directly correlate to the accounts of OS. One of the chief reasons for this difference is due to the wider range of material that is allowed to undergo NegShift, which includes both pronouns and full DPs, whereas only prosodically weak object pronouns are allowed to undergo OS. Further discussion about the similarities and differences between OS and NegShift is found in §\ref{sec:DISTRIBUTION}. However, even though there are differences the one thing that could unite them is there shared movement of pronouns. There have been several claims that NegShift has a preference for shifting "lighter" NIs over "heavier" ones \citep{christensenInterfacesNegationSyntax2005,penkaNegativeIndefinites2011}. This has the effect that speakers prefer shifting pronouns and small DPs over more complex DPs, see §\ref{sec:PROSODY} for a more detailed discussion. 

The problem that I am focusing on for my qualifying paper is whether or not there is a prosodic motivation for NegShift in the Scandinavian languages given this claim made by \citet{christensenInterfacesNegationSyntax2005} and \citet{penkaNegativeIndefinites2011}. The rest of this paper will discuss the properties of NegShift and OS and how I plan on solving the problem related to whether or not there is prosodic motivation for NegShift. 

%------------------------------------
\section{Distributional properties of NegShift versus OS} \label{sec:DISTRIBUTION}
%------------------------------------

As mentioned above there are certain patterns that NegShift and OS share and differ in. Both OS and NegShift involve the movement of elements from their base position to a position that is to the left of the VP, as seen by the movement across negation/adverbials in the case of OS, (\ref{ex:OS}), and across the verb in the case of NegShift, (\ref{ex:NS}).
	\ea Distributional similarities between OS and NegShift in Swedish.
		\ea \label{ex:OS}
		\gll Jag kyssade\textsubscript{v} henne\textsubscript{o} inte [\textsubscript{VP} t\textsubscript{v} t\textsubscript{o} ] \\
		I kiss.\Pst{} her \Neg{}\\
		\glt `I didn't kiss her.'
		\ex \label{ex:NS}
		\gll Jag har ingen\textsubscript{o} [\textsubscript{VP} kyssat t\textsubscript{o} ]\\
		I have no-one ~ kiss.\Pst{}.\Ptcp{} \\
		\glt `I haven't kissed anyone.'
		\z 
	\z 
Additionally, they are similar in that they both operate on pronouns, weak object pronouns for OS and NI pronouns for NegShift. 

However, in terms of the differences between OS and NegShift, two main differences exist. First; NegShift applies to full negative DPs such as \textit{inga böcker} `no books' in addition to pronouns. Second; NegShift \emph{is not} subject to Holmberg's Generalization but is instead subject to an "Anti-Holmberg Effect" where it can shift across phonological material, whereas OS is subject to Holmberg's Generalization. 

Holmberg's Generalization states that ``[OS] cannot apply across a phonologically visible category asymmetrically c-commanding the object position except adjuncts" \citep[15]{holmbergRemarksHolmbergGeneralization1999}. This means that OS can only occur if there is no phonological material, except adjuncts, between its base position and the position to which it shifts. In contrast to OS, NegShift only applies if the verb has not moved out of the VP or if there is nothing between the raised verb and the base position of the NI \citep{foxCyclicLinearizationSyntactic2005,engelsScandinavianNegativeIndefinites2012}. 
	\ea 
		\ea Verb in-situ NegShift\\
		{\gll Ég hef \textit{engan} [\textsubscript{VP} \textbf{séð} t\textsubscript{o}	].\\
			I have nobody ~ seen\\}\jambox*{Ic}
		\glt `I haven't seen anybody.'
		\ex \label{ex:String} String vacuous NegShift\\
		{\gll Jag \textbf{sa} \textit{ingenting} [\textsubscript{VP} t\textsubscript{v}  t\textsubscript{o}  ].\\
		I said nothing\\}\jambox*{Sw}
		\glt `I said nothing'
		\z
	\z 

We can be confident that movement is occurring in (\ref{ex:String}) because of the interaction of the NI and certain low aspectual adverbials. As discussed in \citet{nilsenAdverbsAshift1997} adverbs are allowed to appear below negation in Norwegian in the exact same order \citet{cinqueAdverbsFunctionalHeads1999} observed for Italian. \citet{svenoniusStrainsNegationNorwegian2002} further remarks that we can see this interaction between negation and low adverbs like \emph{tidlig} `early'. 

\ea
	\ea 
	\gll Fænsene har \emph{på} \emph{intet} \emph{tidspunkt} \emph{tidlig} slått av TV’en.\\
	the.fans have on no time early turned off the.TV\\
	\glt `The fans have at no time turned the TV off early’ 
	\ex[*] {
	\gll Fænsene har \emph{tidlig} \emph{på} \emph{intet} \emph{tidspunkt} slått av TV’en.\\
	the.fans have early on no time turned off the.TV\\
	}
	\z  
\z

Evidence collected and reported in \citet{engelsMicrovariationObjectPositions2011,engelsScandinavianNegativeIndefinites2012} shows that NegShift is in fact more complicated and subject to greater variability than was previously thought. \citeauthor{engelsScandinavianNegativeIndefinites2012} found that NegShift was permissible from a greater number of contexts and was more likely to occur depending on the variety and register that was being used, which is summarized in Table \ref{tab:Distribution} taken from \citet{engelsScandinavianNegativeIndefinites2012}. In Table \ref{tab:Distribution}, ✔︎ indicates that NegShift occurs, * indicates that NegShift cannot occur, ? means that there was idiosyncratic variation between speakers.
\begin{table}[!ht]
	\centering
	\caption{Distribution of NegShift across Scandinavian languages. (WJ = West Jutlandic, Ic = Icelandic, Fa = Faroese, DaL = Danish Linguists, SwL = Swedish Linguists, Scan1 = literary/formal Mainland Scandinavian, Scan2 = colloquial Mainland Scandinavian and Norwegian)}
	\label{tab:Distribution}
\begin{tabular}{llccccccccc}
	\hline 
	NegShift across &  & WJ1 & WJ2 & Ic & Fa & DaL1 & DaL2 & SwL & Scan1 & Scan2 \\ 
	\hline 
	String-vacuous &  & ✔︎ & ✔︎ & ✔︎ & ✔︎ & ✔︎ & ✔︎ & ✔︎ & ✔︎ & ✔︎ \\ 
	Verb &  & ✔︎ & ✔︎ & ✔︎ & ✔︎ & ✔︎ & ✔︎ & ✔︎ & ✔︎ & * \\ 
	IO & verb in situ & ✔︎ & ✔︎ & ✔︎ & ✔︎ & ✔︎ & ✔︎ & ✔︎ & ✔︎ & * \\ 
	& verb moved & * & * & * & * & * & * & * & * & * \\ 
	Preposition & verb in situ & ✔︎ & ✔︎ & ✔︎ & ✔︎ & ? & ? & * & * & * \\ 
	& verb moved & ✔︎ & ✔︎ & ? & * & * & * & * & * & * \\ 
	Infinitive & verb in situ & ✔︎ & ✔︎ & ✔︎ & ✔︎ & ✔︎ & * & ? & * & * \\ 
	& verb moved & ✔︎ & * & * & ✔︎ & * & * & * & * & * \\ 
	\hline 
\end{tabular} 
\end{table}


Of crucial interest to our discussion are the rows showing the behavior of NegShift with respect to crossing a verb and crossing an indirect object.  The reason these rows are of interest lies in how NegShift behaves to these syntactic objects when compared with OS. This comparison is carried out in the following section.

%------------------------------------
\section{Comparison of NegShift and OS} \label{sec:NEG-OS}
%------------------------------------
We can compare certain patterns that NegShift has against patterns that OS has. In so doing we can determine whether they are governed by the same or different factors. If OS and NegShift are derived by the same trigger then we expect them to behave the similarly to one another. If they are not similar than this suggests that they are not derived by the same trigger. 

There are three different metrics that we can use to determine the governing factors. These factors will examine: (i) how obedient to Holmberg's Generalization are NegShift and OS; (2) Where the loci of movement are; and (iii) how NegShift and OS interact with one another.   

I show that these two phenomenon are in fact different with NegShift being governed and derived by syntactic factors and not prosodic as described for OS in \citet{erteschik-shirSoundPatternsSyntax2005,erteschik-shirScandinavianObjectShift2017,erteschik-shirVariationMainlandScandinavian2020,brinkerhoffMATCHINGPhrasesNorwegian2021}.  

%------------------------------------
\subsection{With respect to Homlberg's Generalization} \label{sec:HG}
%------------------------------------

The most defining characteristic for OS is its adherence to Holmberg's Generalization, which is defined formally in \ref{ex:HG}.  
\ea \label{ex:HG} {Holmberg's Generalization:\\
Object Shift cannot apply across a phonologically visible category asymmetrically c-commanding the object position except adjuncts} \jambox*{\citep[15]{holmbergRemarksHolmbergGeneralization1999}}
\z
This generalization captures the fact that OS is only possible if the verb has evacuated the VP and there is no phonologically visible category present, except adverbials. 

\ea Examples of OS 
 	\ea {
	\gll Peter så\textsubscript{v} \textbf{ham\textsubscript{o}} ikke [\textsubscript{VP} t\textsubscript{v} \textbf{t\textsubscript{o}}]\\
    Peter see.\Pst{} him not \\}\jambox*{(Across negation)}
    \glt `Peter didn't see him.' \label{ex:classic}
    \ex {
    \gll Peter så\textsubscript{v} \textbf{ham\textsubscript{o}} ofte [\textsubscript{VP} t\textsubscript{v} \textbf{t\textsubscript{o}}]\\
    Peter see.\Pst{} him often \\}\jambox*{(Across adverbials)} 
    \glt `Peter often saw him.' \label{ex:oft}
	\z 
\z 
If, however, there is anything in the VP then OS is blocked and the object pronoun must remain in situ. Note that there is some variability when it comes to how OS interacts with verbal particles, which remain low. Verbal particles will be discussed in more detail in Section \ref{sec:Particles}.

\ea Blocking OS
	\ea Blocking by verb\\
	\gll Peter har ikke [\sub{VP} sått ham ].\\
	Peter has not ~ seen him\\
	\glt `Peter didn't see him'
	\ex Blocking by object\\
	\gll Peter lånte\sub{v} ikke [\sub{VP}  \textbf{Tore\sub{io}} t\sub{v} {}  \textbf{dem\sub{o}}]\\
    Peter lend.\textsc{past} not ~ Tore them \\
	\glt `Peter didn't lend them to Thor.'
	\z 
\z 


As mentioned at the end of Section \ref{sec:DISTRIBUTION}, NegShift seems to be the polar opposite of OS and generally requires the verb to have remained in situ, which is evidenced by the overwhelming acceptability of the NegShift when the verb remains in situ. 

\ea Examples of NegShfit with the verb in situ. 
	\ea {
		\gll Ég hef \textit{engan}\sub{o} [\sub{VP} \textbf{séð} t\textsubscript{o}.] \\
			I have nobody {} seen\\}\jambox*{Ic}
	\glt `I haven't seen anybody.'
	\ex {
	\gll {Í dag} hevur Petur \textit{einki}\sub{o} [\sub{VP} \textbf{sagt} t\textsubscript{o}].\\
	today has Peter nothing {} said\\}\jambox*{Fa}
	\glt `Peter hasn't said anything today.'
	\ex {
	\gll Manden havde måske \textit{ingenting}\sub{o} [\sub{VP} \textbf{sagt} t\textsubscript{o}].\\
	man-the had probably nothing {} said\\}\jambox*{Da}
	\glt `The man hadn't said anything'
	\z  
\z 

NegShift is also insensitive to any phonologically visible category, which results in NegShift occurring regardless of the amount of phonological material present in the VP. We see this with NegShift being acceptable when it crosses an indirect object. 

\ea NegShift across indirect objects. 
	\ea {
	\gll Jón hefur \textit{ekkert} [\sub{VP} \textbf{sagt} \textbf{Sveini} t\sub{o}]. \\
	Jón has nothing {} said Sveinn\\}\jambox*{Ic}
	\glt `John hasn't told Sveinn anything'
	\ex {
	\gll {Í dag} hevur Petur \textit{einki} [\sub{VP} \textbf{givið} \textbf{Mariu} t\sub{o}]. \\
		today has Peter nothing {} given Mary\\}\jambox*{Fa}
	\glt `Today, Peter hasn't given Mary anything.'
	\ex {
	\gll Jeg har \textit{ingen} \textit{bøger} [\sub{VP} \textbf{lånt} \textbf{børnene} t\sub{o}].\\
	I have no books {} lent children-the\\}\jambox*{WJ/Scan1}
	\glt `I haven't lent the children any books.'
	\z 
\z 

We see that NegShift does not obey Holmberg's Generalization, whereas OS does. In fact, NegShift's generalization seems to be the exact opposite of the Holmberg's Generalization. This behavior that NegShift exhibits in requiring verbs to remain in situ and being able to cross phonologically visible material has lead some authors to call this the \emph{Anti-Holmberg Effect} \citep{foxCyclicLinearizationSyntactic2005,engelsScandinavianNegativeIndefinites2012}. 

%------------------------------------
\subsection{With respect to landing sites} \label{sec:LANDING}
%------------------------------------

We further observe a difference between OS and NegShift as it relates to where the moved elements land. In terms of OS, it is always located to a position higher than all of the adverbials.  
\ea 
	\gll Jeg lånte \emph{hende\sub{IO}} faktisk\sub{Adv} aldri\sub{Adv} [\sub{VP} t\sub{IO} bøgerne].\\
	I lent her actually never ~ ~ books-the\\
	\glt `I actually never lent her the books' 
\z 

This movement always results in the verb and weak pronoun being adjacent, after linearization. This behavior lead \citet{erteschik-shirSoundPatternsSyntax2005} to claim that the phonologically weak pronouns incorporate into the verb prior to the verb movement to C. This incorporation causes the weak object pronoun to appear in position higher than the adverbials, which reside in a hierarchy of functional heads \citep{cinqueAdverbsFunctionalHeads1999}. 
\ea
\z 

More recent accounts assume that this is entirely prosodic and the positioning of the 

\ea NegShift occurs to the right of most adverbials with the exception of low adverbials \citep{nilsenAdverbsAshift1997,svenoniusStrainsNegationNorwegian2002}.
	\ea 
	\gll Jeg har faktisk\sub{Adv} \emph{ingen} \emph{bøger}\sub{DO} [\sub{VP} lånt Othilia t\sub{DO}].\\
	I have actually no books ~ lend.\Pst{}.\Ptcp{} Othilia \\
	\glt `I didn't actually lend Othilia any books.'
	\ex 
	\gll Fænsene har på \emph{intet} \emph{tidspunkt}\sub{O} tidlig\sub{Adv} slått av TV’en.\\
	fans-the have at no time early turned off TV-the\\
	\glt `The fans have at no time turned the TV off early'
	\z 
\z

\begin{tcolorbox}[width=\linewidth]
\textsc{Claim}: OS and NegShift move to different locations.
\end{tcolorbox}

%------------------------------------
\subsection{With respect to NegShift and OS interactions} \label{sec:INTERACTION}
%------------------------------------

\ea When there are both weak object pronouns and NIs in the sentence they both shift to their respective landing sites. 
	\ea
	\gll Jeg lånte \textit{hende} fraktisch \textit{ingen} \textit{bøger}.\\
	I lent her actually no books\\
	\glt `I didn't actually lend her any books.'
	\z 
\ex When the verb remains in situ OS is blocked from occurring but NegShift is still allowed to occur. 
	\ea 
	\gll Jeg har \emph{ingen} \emph{bøger} lånt \emph{hende}.\\
	I have no books lent her\\
	\glt `I haven't lent her any books
	\z
\z

\begin{tcolorbox}[width=\linewidth]
\textsc{Claim:} OS and NegShift are not the same.

\textsc{Big Claim:} OS and NegShift have different triggers for movement. 
\end{tcolorbox}
If both OS and NegShift were allowed to occur \citeauthor{broekhuisUnificationObjectShift2020} notes that they are subject to certain ordering restrictions. Citing examples from \citet[163ff]{christensenInterfacesNegationSyntax2005}, Broekhuis shows the following pair of examples:
	\ea 
		\ea \label{ex:NegShift}
		\gll Jeg har <ingen bøger> lånt hende <*ingen bøger>.\\
		I have no books lent her\\
		\glt `I haven't lent her any books
		\ex \label{ex:NegOS}
		\gll Jeg lånte \textit{henda} fraktisch \textit{ingen} \textit{bøger}.\\
		I lent her actually no books\\
		\glt `I didn't actually lend her any books.'
		\z 
	\z
In (\ref{ex:NegShift}), we see that when we have a negative object that it shifts to a position higher than the \textit{v}P if it were to remain in-situ as it would be ungrammatical and would require the use of \textit{ikke} `not' and the negative polarity item \textit{nogen} `any'.
	\ea
	\gll Jeg har \textit{ikke} lånt hende \textit{nogen} bøger.\\
	I have not lent her any books.\\
	\glt `I haven't lent her any books.'
	\z
However, when the main verb has raised to C⁰ as in (\ref{ex:NegOS}) then the weak pronominal moves to a position higher than the adverb \textit{fraktisch} `actually'. The negative object is not able to move to the similar position that is higher than the adverb. Additionally, this results in OS > NegShift which \citeauthor{broekhuisUnificationObjectShift2020} reports to a universal. This does help us see that that even though these two phenomena appear to be similar they are in fact slightly different, due to the differences in the where the two different movement operations' targets are.


%------------------------------------------------


One of the most interesting aspects from Table \ref{tab:Distribution} is the sharp contrast as to whether or not NegShift can happen with an indirect object. From the table we see that all varieties, except Scandinavian 2 which is equivalent to Norwegian and some colloquial varieties, allow shifting across an indirect object if the verb remains in situ. If the verb has moved for V2 then there are no varieties that allow shifting across the indirect object.

\citet{christensenInterfacesNegationSyntax2005} reports on this behavior between NegShift and indirect objects and summarizes his findings in Table \ref{tab:OSNEGS}. In this table No\textsuperscript{+}/Sw\textsuperscript{+} represent some varieties of Norwegian and Swedish respectively in contrast to more standard Norwegian (No) and Swedish (Sw), FS represent the Swedish variety which is spoken by Swedes in Finland.
\begin{table}[h!]
\centering
\caption{Summary of OS and NegShift accoriding to \citet{christensenInterfacesNegationSyntax2005}.}
\label{tab:OSNEGS}
\begin{tabular}{lccccc}
\hline
IO-DO & Ic & Da/Fa & No/Sw & No\textsuperscript{+}/Sw\textsuperscript{+} & FS  \\
\hline 
Pron-Pron	&	+ +	&	+ +	&	\% \%	&	§ §	&	- -	\\
Pron-NegQP	&	+ +	&	+ +	&	+ +	&	+ +	&	- -	\\
NegQP-Pron	&	+ -	&	+ -	&	+ -	&	+ -	&	- -	\\
Pron-DP	&	+ \%	&	+ -	&	\% -	&	\% -	&	- -	\\
DP-Pron	&	\% -	&	- -	&	- -	&	- -	&	- -	\\
DP-DP	&	\% \%	&	- -	&	- -	&	- -	&	- -	\\
DP-NegQP	&	\% \%	&	- -	&	- -	&	- -	&	- -	\\
NegQP-DP	&	+ -	&	+ -	&	+ -	&	+ -	&	- -	\\
\hline 
\end{tabular}\\
(KEY: + = obligatory, - = blocked, \% = optional, § = optional and `non-parallel’)
\end{table}

The sections on this table that are most interesting are those involving what \citeauthor{christensenInterfacesNegationSyntax2005} calls Negative Quantifier Phrases (equivalent to NIs). However, this does conflate NI determiners and NI pronouns into a single category. According to \citeauthor{christensenInterfacesNegationSyntax2005}, when the IO is a pronoun and the DO is a NegQP both obligatorily shift when the verb has been able to swift to C, (\ref{ex:HER}), otherwise only the NegQP shifts, (\ref{ex:BOOKS}).
	\ea 
		\ea \label{ex:HER}
		\gll Jeg lånte \textit{hende(IO)} faktisk \textit{ingen} \textit{bøger(DO)}\\
		I lent her actually no books\\
		\glt `I actually didn't lend her any books'
		\ex \label{ex:BOOKS}
		\gll Jeg har \textit{ingen} \textit{bøger(DO)} lånt \textit{hende(IO)}\\
		I have no books lend her\\
		\glt `I didn't lend her any books'
		\z 
	\z 
If, however, the IO is a NegQP and the DO is a pronoun then the pronoun is blocked from shifting, producing a freezing effect on OS.
	\ea Freezing effects on OS
		\ea 
		\gll Jeg lånte faktisk \textit{ingen(IO)} \textit{den(DO)}\\
		I lent actually no-one it\\
		\glt `I actually lent it to no-one.'
		\ex[*] {Jeg lånte \textit{den(DO)} faktisk \textit{ingen(IO)}} 
		\z 
	\z

This is actually a very important point for the question of the prosodic nature of the shifting. If we assume that these are moving to a position outside of the VP or are some sort of adjunct to VP then we would assume that OS should be allowed according to Holmberg's Generalization.\footnote{See \citet{thrainssonSyntaxIcelandic2010} for discussion and debate about where negation is located in Scandinavian languages.} However, this is not the case if we follow the logic from Holmberg's Generalization. Holmberg's Generalization requires that OS occur if there is not a phonologically visible category that asymmetrically c-commands the object's base position. Because OS is blocked, then Neg is a phonologically visible category that asymmetrically c-commands the object.\footnote{Another possibility is that we are concerned with the }

An additional case that is interesting is when both the indirect and direct objects are allowed to shift. \citet[417f]{broekhuisUnificationObjectShift2020} observes that weak pronominal object shift behaves differently than full DP objects in what loci there are allowed to inhabit. In the case of weak pronominals they are required to appear outside of the \textit{v}P if there is no intervening phonological material (i.e., Holmberg's Generalization \cite{holmbergWordOrderSyntactic1986,holmbergRemarksHolmbergGeneralization1999}). 
%------------------------------------
\section{Deriving NegShift} \label{sec:ZEIJLSTRA}
%------------------------------------

\citet{zeijlstraSyntacticallyComplexStatus2011} is interested in showing providing an analysis of the split-scope interpretation that exists for negative indefinites in Germanic languages. Split-scope is evident when modals and other auxiliaries are present and the negation scopes higher than the modal/auxiliary's scope where the indefinite resides. \citeauthor{zeijlstraSyntacticallyComplexStatus2011} assumes that this behavior is the result of the compositional status of negative indefinites similar to the claims made by \citet{iatridouNegativeDPsAMovement2011}. He claims that NIs are composed of a negative operator and an indefinite component and that the split-scope interpretation is the result of a copy-theory of movement \citep{chomskyMinimalistProgram2015}. 

Following \citeauthor{chomskyMinimalistProgram2015} movement is not a unique operation that literally moves one element to position higher in the syntax. Instead movement is similar the product of copying a constituent and then remerging it into the syntactic hierarchy. This then results in two or more instances of the constituent, as seen in the syntactic structure for (\ref{ex:Copy}). 

We see as we proceed along the derivation that the verb is first copied and then merged with \emph{v} leaving behind a complete copy of itself in its base-generated position. This operation of copying and then remerging is then continued until we arrive at spell-out. According to \citet{chomskyMinimalistProgram2015}, only the highest copy will be pronounced at PF, this is represented by the lower copies having a strike through them. 
% \pagebreak

\ea Derivation of \emph{Anna har intet hørt.} \label{ex:Copy}\\
\begin{forest}
	[CP [DP\\\emph{Anna}] [C´ [C [T\\\emph{har} ] [C] ] [TP [<DP>\\\sout{\emph{Anna}} ] [T´ [<T>\\\sout{\emph{har}} ] [NegP [DP\\\emph{intet}] [Neg´ [Neg] [\emph{v}P [<DP>\\\sout{\emph{Anna}}] [\emph{v´} [\emph{v} [V\\\emph{hørt}] [\emph{v}] ] [VP [<V>\\\emph{\sout{hørt}}] [<DP>\\\sout{\emph{intet}}] ] ] ] ] ] ] ] ] ] 
\end{forest}
\z 

If we assume that \citet{zeijlstraSyntacticallyComplexStatus2011} is correct in that negative indefinites exist in their base-generated position and in this higher copy because of feature-checking and for interpretation at LF then we expect that the higher copy is the only one that is ever pronounced. However, the facts discussed in §\ref{sec:PROSODY} present a problem. 

As discussed above Danish allows for a pattern where only a negative indefinite of a certain weight is allowed to shift, repeated here from (\ref{ex:weight}). 
	\ea
		\ea 
		\gll Jeg har \textit{intet}\textsubscript{o} hørt t\textsubscript{o}.\\
		I have nothing heard\\
		\glt  `I havn't heard anything.'
		\ex 
		\gll Jeg har [\textit{intet} \textit{nyt}]\textsubscript{o} hørt t\textsubscript{o}.\\
		I have nothing new heard\\
		\glt `I haven't heard anything new'
		\ex[*] {
		\gll Jeg har [\textit{intet} \textit{nyt} \textit{i} \textit{sagen}]\textsubscript{o} hørt t\textsubscript{o}.\\
		I have nothing new in case-\textsc{det} heard\\
		\glt `I haven't heard anything new about the case.'
		}
		\ex[*] {
		\gll Jeg har [\textit{intet} \textit{nyt} \textit{i} \textit{sagen} \textit{om} \textit{de} \textit{stjålne} \textit{malerier}]\textsubscript{o} hørt t\textsubscript{o}.\\
		I have nothing new in case-\textsc{det} about the stolen paintings heard\\
		}
		\glt `I haven't heard anything new in the case about the stolen paintings.'
		\z 
	\z

For NI DPs that are too heavy to shift as a whole constituent one of the potential repairs is to shift only the NI while stranding the rest of the constituent as in (\ref{ex:split}), repeated here.

\ea Jeg har \textit{intet}\textsubscript{i} hørt t\textsubscript{i} [\textsubscript{PP} i sagen om de stjålne malerier ].
\z 

Folloowing \posscitet{chomskyMinimalistProgram2015} theory of copy-movement and \posscitet{zeijlstraSyntacticallyComplexStatus2011} theoy, we would assume that the NI DP would be fully copied from its base-position and merged into NegP.

\ea \resizebox*{\linewidth}{!}{\begin{tikzpicture} 
	\tikzset{every tree node/.style={align=center,anchor=north}} 
	\Tree [.CP 	DP\\\emph{Anna} 
				[.C´ C\\\emph{har} 
				[.TP <DP> 
					[.T´ T 
						[.NegP  
							[.DP \edge[roof]; {intet i sagen om de stjålne malerier}
							]
							[.vP <DP>
								[.v´ \emph{hørt}   
									[.VP <V> 
										[.DP \edge[roof]; {intet i sagen om de stjålne malerier}
										]
									]
								]
							]    
						]
					] 
				]
				] 
			] 
	\end{tikzpicture}}
\z 
At this point during PF, part of the higher copy is deleted leaving only the NI pronoun. In the lower copy \emph{intet} is deleted.
\ea	\label{ex:tree} \resizebox*{\linewidth}{!}{
	\begin{tikzpicture} 
	\tikzset{every tree node/.style={align=center,anchor=north}} 
	\Tree [.CP 	DP\\\emph{Anna} 
				[.C´ C\\\emph{har} 
				[.TP <DP> 
					[.T´ T 
						[.NegP  
							[.DP \edge[roof]; {intet \sout{i sagen om de stjålne malerier}}
							]
							[.vP <DP>
								[.v´ hørt   
									[.VP <V> 
										[.DP \edge[roof]; {\sout{intet} i sagen om de stjålne malerier}
										]
									]
								]
							]    
						]
					] 
				]
				] 
			] 
	\end{tikzpicture}}

\z 
If \citet{chomskyMinimalistInquiriesFramework2000} is correct and only the higher copy should be pronounced, then the question arises as to why only a part of the copy is deleted. 

This is not the first time this sort of problem has arisen. \citet{fanselowRemarksEconomyPronunciation2001,fanselowDistributedDeletion2002} present evidence that partial deletion is in fact possible and the type of deletion that we observe in (\ref{ex:tree}) is found in other Germanic languages and in Slavic as seen in this example from Croatian.
\ea 
\gll \emph{\sout{zanimljive}} \emph{knijge} mi je Marija \emph{zanimljive} \emph{\sout{knijge}} preporučila\\
interesting books me has Mary interesting books recommended\\
\z  
According to \citet{fanselowRemarksEconomyPronunciation2001,fanselowDistributedDeletion2002} the motivation for when partial deletion is possible has its motivations in PF or LF. My QP will explore more about what exactly these motivations are in light of \citeauthor{fanselowRemarksEconomyPronunciation2001,fanselowDistributedDeletion2002} and whether prosody plays a role in what is allowed to delete. The other possibility is that this deletion is motivated by LF considerations. If this is prosodically motivated what drives this deletion?

%------------------------------------
\section{Prosodic restrictions on NegShift} \label{sec:PROSODY}
%------------------------------------

However, as noted earlier not all NegShift is treated equal. \citet[65f]{christensenInterfacesNegationSyntax2005}, speaking on Danish, claims that the ``weight" of the NI plays a crucial factor in whether or not NegShift occurs. 
	\ea \label{ex:weight}
		\ea 
		\gll Jeg har \textit{intet}\textsubscript{o} hørt t\textsubscript{o}.\\
		I have nothing heard\\
		\glt  `I havn't heard anything.'
		\ex 
		\gll Jeg har [\textit{intet} \textit{nyt}]\textsubscript{o} hørt t\textsubscript{o}.\\
		I have nothing new heard\\
		\glt `I haven't heard anything new'
		\ex[*] {
		\gll Jeg har [\textit{intet} \textit{nyt} \textit{i} \textit{sagen}]\textsubscript{o} hørt t\textsubscript{o}.\\
		I have nothing new in case-\textsc{det} heard\\
		\glt `I haven't heard anything new about the case.'
		}
		\ex[*] {
		\gll Jeg har [\textit{intet} \textit{nyt} \textit{i} \textit{sagen} \textit{om} \textit{de} \textit{stjålne} \textit{malerier}]\textsubscript{o} hørt t\textsubscript{o}.\\
		I have nothing new in case-\textsc{det} about the stolen paintings heard\\
		}
		\glt `I haven't heard anything new in the case about the stolen paintings.'
		\z 
	\z
In those instances where the NI is too large one potential repair is to strand the PP while moving just the pronoun or using the negative particle \textit{ikke} and a NPI.
	\ea 
		\ea Jeg har \textit{intet}\textsubscript{i} hørt t\textsubscript{i} [\textsubscript{PP} i sagen om de stjålne malerier ]. \label{ex:split}
		\ex Jeg har \textit{ikke} hørt [ \textit{noget} i sagen om de stjålne malerier ].
		\z 
	\z   
This same behavior has also been remarked upon by \citet{penkaNegativeIndefinites2011} for Swedish.
	\ea 
		\ea 
		\gll Men mänskligheten har \textit{ingenting}\textsubscript{o} lärt sig t\textsubscript{o}.\\
		but mankind-the have nothing taught themselves\\
		\glt `But mankind haven't taught themselves anything.'
		\ex[?] {
		\gll Vi hade \textit{inga} \textit{grottor}\textsubscript{o} undersökt t\textsubscript{o}.\\
		we have no caves explored\\
		}
		\glt `We haven't explored any caves.'
		\z 
	\z 

My qualifying paper will explore whether or not there is indeed this preference for NegShift of pronouns by conducting a study on the Swedish Culturomics Gigaword Corpus \citep{eideSwedishCulturomicsGigaword2016} and how this phenomenon might relate to prosodic analyses of OS such as those from \citet{erteschik-shirVariationMainlandScandinavian2020} and \citet{brinkerhoffMATCHINGPhrasesNorwegian2020} and the more syntactically motivated accounts using Cyclic Linearization \citep{foxCyclicLinearizationSyntactic2005,engelsScandinavianNegativeIndefinites2012} or following \citet{zeijlstraSyntacticallyComplexStatus2011} and \posscitet{iatridouNegativeDPsAMovement2011} accounts for NI movement.

%------------------------------------
\subsection{PF deletion} \label{sec:NEXT}
%------------------------------------
\ea  Following the copy theory of movement \citet{chomskyMinimalistProgramLinguistic1993}, multiple copies of the NI will be present at PF spell-out. 
\ex A question arises as to whether or not phases are have any bearing on the facts presented above. 
\ex There are two potential answers
	\ea Phases do not play a role in determining phonological/prosodic behavior.  
	\ex Phases do play a role in determining phonological/prosodic behavior.
	\z 
\ex Recent evidence from \citet{weberPhasebasedConstraintsMatch2020} suggests that phases do in fact play a role in determining phonological behavior. 
\ex In order for the phonology to interact with two copies at the same time, the base generated position and the landing site both must belong to the same phase. 
	\ea This requires that NegP and the rest of the Cinquean hierarchy of adverbials \citep{cinqueAdverbsFunctionalHeads1999} belong to the same phase as \emph{v}P.
	\z

\ex \citet{kandybowiczGrammarRepetitionNupe2008} makes the observation that multiple copies that are generated by the narrow syntax can be \emph{phonologically} realized when there is a identifiable PF well-formedness condition is avoided. 
	\ea Similarly, I argue that PF can also dictate the amount of material that is deleted in those copies. 
	\ex PF being able to delete is not new and was argued for by \citet{ottDeletionClausalEllipsis2016} to account for German clausal ellipsis. 
	\z 

\ex In the case of NegShift, there are restrictions on \emph{Mittelfeld} well-formedness, which I will call the \textsc{Light Mittelfeld Condition} (LMC).
	\ea It has been observed that only a limited amount of structure is allowed and a wide degree of variation is permitted in the \emph{Mittelfeld} (see \cite{haiderMittelfeldPhenomenaScrambling2017}). 

	\ex I argue that the largest unit that is allowed to remain in the \emph{Mittelfeld} in Scandinavian is a maximal prosodic word (ω\sub{max}).
	\z 

\ex Evidence for this comes from the size of the material that is allowed to ``shift" in these languages. 

\ex As observed for Danish only a pronoun or DP, consisting of just a D and NP, is allowed to occupy the \emph{Mittelfeld} when NegShift occurs. 
	\ea 
	\gll Jeg har \textit{intet}\textsubscript{o} hørt t\textsubscript{o}.\\
	I have nothing heard\\
	\glt  `I havn't heard anything.'
	\ex 
	\gll Jeg har [\textit{intet} \textit{nyt}]\textsubscript{o} hørt t\textsubscript{o}.\\
	I have nothing new heard\\
	\glt `I haven't heard anything new'
	\ex[*] {
	\gll Jeg har [\textit{intet} \textit{nyt} \textit{i} \textit{sagen}]\textsubscript{o} hørt t\textsubscript{o}.\\
	I have nothing new in case-\textsc{det} heard\\
	\glt `I haven't heard anything new about the case.'
	}
	\z  	

\ex This difference between Danish, which allows full DPs, and Swedish which tolerates full DPs, but prefers pronouns, suggests the \emph{Mittelfeld} in Swedish will delete copies until they are just the NI pronoun. 
	\ea This potential comes down to Swedish being a tonal language and Danish not being a tonal language.
	\z 

\ex The LMC deleting different amounts of material in Danish and Swedish can also explain Norwegian's lack of NegShift. 
	\ea Norwegian deletes everything as it prefers not to have any copies in the \emph{Mittelfeld}
	\ex This deleteion would then leave the valued [\textsc{Neg}] feature which gets pronounced as negation which then causes the lower copy to surface with a NPI.  
	\ex This behavior of total deletion is also attested in Danish and Swedish where negation and a NPI is always a potential instead of NegShift.
	\z  

\ex This results in a three-way system in Scandinavian languages. 
	\ea Those that delete until a full DP is left.
	\ex Those that delete until a pronoun is left.
	\ex Those that delete everything and have negation and a NPI.
	\z 

\ex There seems to be some differences in behavior between the tonal and atonal Scandinavian languages. 
	\ea According to \citet{thrainssonFaroeseOverviewReference2004,thrainssonSyntaxIcelandic2010} Faroese and Icelandic pattern the same as Danish in this regard. 
	\ex This further suggests that there is something unique about being a tonal language that limits the acceptability of NegShifting. 
	\z 

\ex This is summarized in Table \ref{tab:Paradigm}.
	\ea It will be noted that if you allow full NI DPs than you also allow pronouns and complete deletion, which results in a negation particle and an NPI.
	\ex If you allow NI pronouns then you allow a negation particle and an NPI
	\z 
\begin{table}[!ht]
	\centering
	\caption{Scandinavian acceptance of NegShift or NPI}
	\label{tab:Paradigm}
\begin{tabular}{lccc}
	\hline 
	& Full DPs & Pronouns & NPI\\
	\hline
	Icelandic & ✔︎ & ✔︎ & ✔︎ \\
	Faroese & ✔︎ & ✔︎ & ✔︎ \\
	Danish & ✔︎ & ✔︎ & ✔︎ \\
	Swedish & * & ✔︎ & ✔︎ \\
	Norwegian & * & * & ✔︎ \\
	\hline 
\end{tabular} 
\end{table}
\z 
%------------------------------------
\subsection{Scandinavian particle shift} \label{sec:Particles}
%------------------------------------
\ea Independent evidence for shifting full DPs into the \emph{Mittelfeld} is observed in particle shifting in Scandinavian languages. 

\ex Following \citet[2]{holmbergRemarksHolmbergGeneralization1999} and \citet{faarlundSyntaxMainlandScandinavian2019} there is a difference in behavior between the different Scandinavian languages with what is allowed to shift across a verbal particle.
	\ea Danish objects, regardless of size, always precedes the verb particle. 
	\ex Norweigan, Icelandic, and Faroese are like English by shifting a pronoun across a particle and optionally for DPs.
	\ex Swedish does not allow anything to shift across the particles. 
	\z 

\ex \gllll Jeg skrev (nummeret/det) op (*nummeret/*det). \hfill Da\\
		Jeg skrev (nummeret/det) opp (nummeret/*det). \hfill No\\
		Jag skrev (*numret/*det) upp (numret/det). \hfill Sw\\
		I wrote (the-number/it) up (the-number/it)\\
\glt `I wrote the number/it down.'

\ex Additionally, Danish places restriction on the verbal compliment if it is too ``heavy" \citep[44f]{mullerDanishHeadDrivenPhraseInpreparation}. 
	\ea If it is larger than a simple DP shifting is blocked.\footnote{Examples are from \emph{KorpusDK} as reported by \citet{mullerDanishHeadDrivenPhraseInpreparation}.} 
	\ex \gll {[…]} så må partiet melde [holdninger] [ud], {[…]}\\
	~ then must party.\Def{} make stances out\\
	\glt `{[…]} then the party must make its stances clear, {[…]}'
	\ex \gll Den danske regering bør snart melde [ud], [at den støtter de amerikanske planer]\\
	the Danish government must soon make out that it supports the American plans\\
	\glt `The Danish government must soon make clear that it supports the American plans.'
	\z 

\ex This is further evidence that the LMC is an active constraint in these languages. 

\ex One explanation for this behavior is the difference in tonal quality between Danish, Norwegian, and Swedish \citep{erteschik-shirVariationMainlandScandinavian2020}.
\z 

\begin{tcolorbox}[width=\linewidth]

Prosody, in the form of the \textsc{Light Mittelfeld Condition}, plays an active role in regulating the size of the material in the \emph{Mittelfeld}
\end{tcolorbox}
%------------------------------------
\section{Conclusion} \label{sec:CONCLUSION}
%------------------------------------

\begin{tcolorbox}[width=\linewidth]
\centering
NegShift is derived by both syntactic and prosodic factors. It is syntactic in movement and prosody is responsible for restricting and regulating the amount of material that is allowed the surface in the higher copy.
\end{tcolorbox}

%------------------------------------
\section{Alternative syntactic accounts} \label{sec:HNPS}
%------------------------------------

% According to \citet{anttilaRoleProsodyEnglish2010}, prosody plays a supporting role in linearization and can be accounted for through OT style constraints on prosodic well-formedness. They are concerned with explaining the three way patterning that dative constructions have: (i) double object constructions, (ii) Prepositional constructions, (iii) Heavy NP Shift constructions.

% Of crucial interest is how they explain Heavy NP Shift and what their criteria for determining weight. \citet[949]{anttilaRoleProsodyEnglish2010} say that the ``weight" of an NP is ``a function of the number of lexically stressed words in [the constituent]". This means that the more lexical stresses a constituent has the heavier it is. This definition, crucially, leaves out functional items and pronouns because they lack lexical stress. Using this definition for weight, we can explain the behavior Danish and Swedish as discussed in §\ref{sec:PROSODY}. 

% %------------------------------------
% \section{OT Account for Heavy NP Shift} \label{sec:OT}
% %------------------------------------

% Based on \citet{anttilaRoleProsodyEnglish2010}, prosody plays a supporting role in linearization and can be accounted for through OT style constraints on prosodic well-formedness. This can be accounted for using Match Theory \citep{selkirkClauseIntonationalPhrase2009,selkirkSyntaxPhonologyInterface2011}. Following \citet{myrbergSisterhoodProsodicBranching2013,myrbergProsodicWordSwedish2013,myrbergProsodicHierarchySwedish2015}, I assume that the subject in Scandinavian languages forms its own phonological phrase, if it is not a pronoun, separate from the rest of the clause. 
% 	\ea Simplified prosodic structure for (\ref{ex:tree})\\
% 	% \resizebox*{\linewidth}{!}{
% 	\begin{tikzpicture} 
% 	\tikzset{every tree node/.style={align=center,anchor=north}} 
% 	\Tree [.ι [.φ\\Anna ] [.φ \edge[roof]; {har intet hørt i sagen om de stjålne malerier} ] ] 
% 	\end{tikzpicture}%}
% 	\z

% Crucially, what we are concerned with the weight of the item shifting. This can be accounted for using a type of \textsc{NoShift} \citep{bennettLightestRightApparently2016} which is sensitive to lexical stresses.

% \ea \textsc{NoShift(Stress)}:\\
% Assign one violation for every word bearing lexical stress that is not in the same linear order as in the input.
% \z 

% If we take the input of (\ref{ex:tree}), this constraint should assign a violation for every item bearing lexical stress that has been relinearized. Using OT we can model how this would behave with the input of (\ref{ex:tree})

% \ea \resizebox*{\linewidth}{!}{
% 	\begin{OTtableau}{3}
% 		\OTdashes{1,2}
% 		\OTtoprow [ ] {\textsc{Match(XP,φ)}, \textsc{Match(φ,XP)}, \textsc{NoShift(Stress)}}
% 		\OTcandrow [\OThand] {( Anna\sub{ω} )\sub{φ} (har \textit{intet}\sub{ω} hørt i sagen om de stjålne malerier)\sub{φ}} { 1 , 1 , 1} 
% 		\OTcandrow [\OThand] {( Anna\sub{ω} )\sub{φ} (har \textit{intet nyt}\sub{ω} hørt i sagen om de stjålne malerier)\sub{φ}} { 1 , 1 , 1} 
% 		\OTcandrow [\OTface*] {( Anna\sub{ω} )\sub{φ} (har \textit{intet nyt\sub{ω} i sagen}\sub{ω} hørt om de stjålne malerier)\sub{φ}} { 1 , 1 , 2!} 
% 		\OTcandrow [\OTface*] {( Anna\sub{ω} )\sub{φ} (har \textit{intet i sagen\sub{ω} om de stjålne\sub{ω} malerier\sub{ω}} hørt )\sub{φ}} { 1 , 1 , 3! } 
% 	\end{OTtableau}
% 	}
% \z 

% I am continuing to explore what the analysis would look like 
%------------------------------------
\subsection{Cyclic Linearization account} \label{sec:CL}
%------------------------------------

An alternative to the theory presented above, Cyclic Linearization is a theory that was developed by \cite{foxCyclicLinearizationSyntactic2005} as a way to account for OS and Holmberg's Generalization. This theory works by stipulating that spell-out of the morphosyntax is cyclic and order preserving, which means that as you spell-out each successive spell-out domain you need to ensure that whatever orders existed when that domain was spelled-out persist at the next spell-out domain's ordering restrictions. This theory also had the benefit of accounting for when OS was allowed or not allowed to occur. 

This proposal was extended by \citet{foxCyclicLinearizationSyntactic2005} and \citet{engelsMicrovariationObjectPositions2011,engelsScandinavianNegativeIndefinites2012} to account for quantifier movement (QM), of which NegShift is a subset under their analyses. QM is subject to an ``Anti-Holmberg Effect'' or an ``Inverse Holmberg Effect''. As previously discussed above Holmberg's Generalization stipulates that OS can only apply if the verb has undergone movement from V-to-T-to-C. The Anti-Holmberg Effect explains that only when the verb remains in situ can we have QM, which is the result of the ordering operations between the different phases being in agreement. 

In order to account for OS, \citeauthor{foxCyclicLinearizationSyntactic2005} propose that the during the spell-out of the VP spell-out domain the V is the leftmost element in its domain\footnote{The position of the V at the left-edge of the phase could be due to the movement of V to \textit{v} in which case it is actually the \textit{v}P that acts as the spell-out domain not the VP.} and at which point the ordering restrictions are in place which state that the V must precede the O. At this point the V moves to T and then to C which results in the object being free to move to its higher position because the order that existed at the VP domain continues to hold at the CP spell-out domain. 
\ea OS and string-vacuous Neg-Shift
\vspace{6pt}
 	\ea {}[\textsubscript{CP} S \rnode{b1}V … [\textsubscript{NegP} \rnode{A1}O adv [\textsubscript{VP} \rnode{b2}t\textsubscript{v} \rnode{A2}t\textsubscript{o} ]]]
	\psset{linearc=2pt} 
	\ncbar[angle=90]{->}{A2}{A1}  
	\ncbar[angle=-90,linestyle=dashed]{->}{b2}{b1} 
	\vspace{6pt}
	\ex VP Ordering: \textbf{V>O}\\
		CP Ordering: S>V, \textbf{V>O}, O>adv, adv>VP
	\z
\z

In the case of NegShift, where it is able to shift across various phonological material, it is proposed that the NI first moves to the left edge of the VP before spell-out of that domain. Once that domain is spelled-out the NI is free to shift to its position outside of the VP, in the case of (\ref{ex:n}) this is to NegP.
	\ea NegShift when V is in-situ. \label{ex:n}
	\vspace{6pt}
		\ea {}[\textsubscript{CP} S aux … [\textsubscript{NegP} \rnode{A1}O [\textsubscript{VP} \rnode{A2}t\textsubscript{o}  V \rnode{A3}t\textsubscript{o} ]]]
		\psset{linearc=2pt} 
		\ncbar[angle=90]{->}{A3}{A2}  
		\ncbar[angle=90]{->}{A2}{A1}
		\ex VP Ordering: \textbf{O>V}\\
		CP Ordering: S>V, aux>O, O>adv, adv>VP → \textbf{O>V}
		\z 
	\z

As part of my QP, I will be providing an alternative account using cyclic linearization and compare it against the account I proposal in §\ref{sec:ZEIJLSTRA}.





%------------------------------------
\section{Next steps} 
%------------------------------------

As previously mentioned, I plan on investigating the prosodic nature of NegShift I, additionally, have been considering the different accounts that have been given for both OS and NegShift with an eye on seeing which account is able to provide the most sensible explanation for the shifting of ``light'' NIs to aid me in presenting my own theoretical analysis of the NegShift.
%------------------------------------
%BIBLIOGRAPHY
%------------------------------------

%\singlespacing
%\nocite{*}
\printbibliography[heading=bibintoc]

\end{document} 