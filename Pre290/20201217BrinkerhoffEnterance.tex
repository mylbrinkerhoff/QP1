% !TEX TS-program = xelatex
% !TEX encoding = UTF-8 Unicode		

\documentclass[12pt, letterpaper]{article}

%%BIBLIOGRAPHY- This uses biber/biblatex to generate bibliographies according to the 
%%Unified Style Sheet for Linguistics
\usepackage[main=american, german]{babel}% Recommended
\usepackage{csquotes}% Recommended
\usepackage[backend=biber,
		style=unified,
		maxcitenames=3,
		maxbibnames=99,
		natbib,
		url=false]{biblatex}
\addbibresource{link_desktop.bib}	%	Use link_laptop.bib when on Laptop and link_desktop.bib when on desktop
\setcounter{biburlnumpenalty}{100}  % allow URL breaks at numbers
%\setcounter{biburlucpenalty}{100}   % allow URL breaks at uppercase letters
%\setcounter{biburllcpenalty}{100}   % allow URL breaks at lowercase letters

%%TYPOLOGY
\usepackage[svgnames]{xcolor} % Specify colors by their 'svgnames', for a full list of all colors available see here: http://www.latextemplates.com/svgnames-colors
%\usepackage[compact]{titlesec}
%\titleformat{\section}[runin]{\normalfont\bfseries}{\thesection.}{.5em}{}[.]
%\titleformat{\subsection}[runin]{\normalfont\scshape}{\thesubsection}{.5em}{}[.]
\usepackage[hmargin=1in,vmargin=1in]{geometry}  %Margins          
\usepackage{graphicx}	%Inserting graphics, pictures, images 		
\usepackage{stackengine} %Package to allow text above or below other text, Also helpful for HG weights 
\usepackage{fontspec} %Selection of fonts must be ran in XeLaTeX
\usepackage{amssymb} %Math symbols
\usepackage{amsmath} % Mathematical enhancements for LaTeX
\usepackage{setspace} %Linespacing
\usepackage{multicol} %Multicolumn text
\usepackage{enumitem} %Allows for continuous numbering of lists over examples, etc.
\usepackage{multirow} %Useful for combining cells in tablesbrew 
\usepackage{hanging}
\usepackage{fancyhdr} %Allows for the 
\pagestyle{fancy}
\fancyhead[L]{\textit{290 Entrance Paper}} 
\fancyhead[R]{\textit{\today}} 
\fancyfoot[L,R]{} 
\fancyfoot[C]{\thepage} 
\renewcommand{\headrulewidth}{0.4pt}
\setlength{\headheight}{14.5pt} % ...at least 14.49998pt
% \usepackage{fourier} % This allows for the use of certain wingdings like bombs, frowns, etc.
% \usepackage{fourier-orns} %More useful symbols like bombs and jolly-roger, mostly for OT
\usepackage[colorlinks,allcolors={black},urlcolor={blue}]{hyperref} %allows for hyperlinks and pdf bookmarks
% \usepackage{url} %allows for urls
% \def\UrlBreaks{\do\/\do-} %allows for urls to be broken up
\usepackage[normalem]{ulem} %strike out text. Handy for syntax

%%FONTS
\setmainfont{Linux Libertine O}
\setsansfont{Linux Biolinum O}
% \setmonofont{DejaVu Sans Mono}

%%PACKAGES FOR LINGUISTICS
%\usepackage{OTtablx} %Generating tableaux with using TIPA
\usepackage[noipa]{OTtablx} % Use this one generating tableaux without using TIPA
%\usepackage[notipa]{ot-tableau} % Another tableau drawing packing use for posters.
% \usepackage{linguex} % Linguistic examples
\usepackage{langsci-gb4e} % Language Science Press' modification of gb4e
% \usepackage{langsci-avm} % Language Science Press' AVM package
\usepackage{tikz} % Drawing Hasse diagrams
%\usepackage{pst-asr} % Drawing autosegmental features
\usepackage{pstricks} % required for pst-asr, OTtablx, pst-jtree.
% \usepackage{pst-jtree} 	% Syntax tree draawing software
\usepackage{tikz-qtree}	% Another syntax tree drawing software. Uses bracket notation.
% \usepackage{forest}	% Another syntax tree drawing software. Uses bracket notation.
%\usepackage{ling-macros} % Various linguistic macros. Does not work with linguex.
%\usepackage{covington} % Another linguistic examples package.
\usepackage{leipzig}

%%TITLE INFORMATION
\title{290 Entrance Paper }
\author{Mykel Loren Brinkerhoff}
\date{\today}


\begin{document}
	
%%If using linguex, need the following commands to get correct LSA style spacing
%% these have to be after  \begin{document}
	% \setlength{\Extopsep}{6pt}
	% \setlength{\Exlabelsep}{9pt}		%effect of 0.4in indent from left text edge
%%
	
%% Line spacing setting. Comment out the line spacing you do not need. Comment out all if you want single spacing.
%	\doublespacing
%	\onehalfspacing
	
\begin{center}
	{\Large \textbf{290 Entrance paper}}\\
	\vspace{6pt}
	Mykel Loren Brinkerhoff\\
\end{center}
%\maketitle
%\maketitleinst
\thispagestyle{fancy}

% \tableofcontents

% %------------------------------------
% \section*{Outline of Handout} \label{sec:OUTLINE}
% %------------------------------------
% \begin{itemize}
% 	\item Overview of Negative Indefinite Shift in §\ref{sec:NEGSHIFT}
% 	\item Discussion of Scandinavian Pronoun structure in §\ref{sec:PRONOUNS}
% 	% \item Discussion of \cite{broekhuisUnificationObjectShift2020} in §\ref{sec:BROEKHUIS}
% 	\item Discussion of \cite{valentinebordalNegationExistentialPredications2017} in §\ref{sec:VB}
% 	\item Discussion of \cite{zeijlstraSyntacticallyComplexStatus2011} in §\ref{sec:ZEIJLSTRA}
% 	\item Next steps in §\ref{sec:NEXT}
% \end{itemize}

%------------------------------------
\section{Introduction} \label{sec:INTRO}
%------------------------------------

Negative Shifting (NegShift) is a process in the Scandinavian languages where a negative indefinite expression (NI) obligatorily shifts to a position outside of the VP. The Danish examples in \ref{ex:PRIMA} show that the NI pronoun \textit{ingenting} `nothing' and the complex DP \textit{ingen bøger} `no books' both shift across the verb into a position that is between the adverbials and the verb in the case of \textit{ingenting} and between the auxiliary and the verb in the case of \textit{ingen bøger}.
	\ea NegShift of pronouns and complex DPs\label{ex:PRIMA}
		\ea
		\gll Manden havde måske \textit{ingenting} [\textsubscript{VP} sagt t\textsubscript{o} ].\\
		man-the had probably nothing ~ said\\
		\glt `The man hadn't said anything.'
		\ex 
		\gll Jeg har \textit{ingen} \textit{bøger} [\textsubscript{VP} lånt børnene t\textsubscript{o}.]\\
		I have no books ~ lent children-the\\
		\glt `I haven't lent the children any books.'
		\z
	\z 

NegShift bears some resemblance to Scandinavian Object Shift (OS), which causes a weak pronoun to also shift to a position outside of the VP when the verb has raised for V2 \citep{holmbergWordOrderSyntactic1986,holmbergRemarksHolmbergGeneralization1999}.
	\ea {
	\gll Jag kyssade\textsubscript{v} henne\textsubscript{o} inte [\textsubscript{VP} t\textsubscript{v} t\textsubscript{o} ] \\
	I kiss.\Pst{} her \Neg{}\\} \jambox*{Sw}
	\glt `I didn't kiss her.' \label{ex:OS}  
	\z  

Interestingly, several authors have claimed that Scandinavian OS is driven and determined by prosodic factors (see \cite{erteschik-shirSoundPatternsSyntax2005,erteschik-shirScandinavianObjectShift2017,erteschik-shirVariationMainlandScandinavian2019,brinkerhoffMATCHINGPhrasesNorwegian2020} for some recent accounts). However, there are others that claim that OS is best accounted for as syntactic movement to satisfy PF, information structure, or some other syntactic requirement (\cite{holmbergRemarksHolmbergGeneralization1999,thrainssonObjectShiftScrambling2001,bentzenObjectShiftSpoken2013,sichelFeaturalLifeNominals2020}, and many others). 

However, it is clear that not all instances of NegShift directly correlate to the accounts of OS. One of the chief reasons for this difference is due to the wider range of material that is allowed to undergo NegShift, which includes both pronouns and full DPs, whereas only prosodically weak object pronouns are allowed to undergo OS. Further discussion about the similarities and differences between OS and NegShift is found in §\ref{sec:DISTRIBUTION}. However, even though there are differences the one thing that could unite them is there shared movement of pronouns. There have been several claims that NegShift has a preference for shifting "lighter" NIs over "heavier" ones \citep{christensenInterfacesNegationSyntax2005,penkaNegativeIndefinites2011}. This has the effect that speakers prefer shifting pronouns and small DPs over more complex DPs.

The problem that I am focusing on for my qualifying paper is whether or not there is a prosodic motivation for NegShift in the Scandinavian languages given this claim made by \citet{christensenInterfacesNegationSyntax2005} and \citet{penkaNegativeIndefinites2011}. 
 
%------------------------------------
\section{Negative indefinite shifting} \label{sec:NEGSHIFT}
%------------------------------------

\ea 
\ex This process occurs to all NIs and is permissible from a large number of different contexts, depending on the variety and register, see Table \ref{tab:Distribution}. 
\begin{table}[h!]
	\centering
	\caption{Distribution of NegShift across Scandinavian languages. WJ = West Jutlandic, Ic = Icelandic, Fa = Faroese, DaL = Danish Linguists, SwL = Swedish Linguists, Scan1 = literary/formal Mainland Scandinavian, Scan2 = colloquial Mainland Scandinavian and Norwegian}
	\label{tab:Distribution}
\begin{tabular}{llccccccccc}
	\hline 
	NegShift across &  & WJ1 & WJ2 & Ic & Fa & DaL1 & DaL2 & SwL & Scan1 & Scan2 \\ 
	\hline 
	String-vacuous &  & ✔︎ & ✔︎ & ✔︎ & ✔︎ & ✔︎ & ✔︎ & ✔︎ & ✔︎ & ✔︎ \\ 
	Verb &  & ✔︎ & ✔︎ & ✔︎ & ✔︎ & ✔︎ & ✔︎ & ✔︎ & ✔︎ & * \\ 
	IO & verb in situ & ✔︎ & ✔︎ & ✔︎ & ✔︎ & ✔︎ & ✔︎ & ✔︎ & ✔︎ & * \\ 
	& verb moved & * & * & * & * & * & * & * & * & * \\ 
	Preposition & verb in situ & ✔︎ & ✔︎ & ✔︎ & ✔︎ & ? & ? & * & * & * \\ 
	& verb moved & ✔︎ & ✔︎ & ? & * & * & * & * & * & * \\ 
	Infinitive & verb in situ & ✔︎ & ✔︎ & ✔︎ & ✔︎ & ✔︎ & * & ? & * & * \\ 
	& verb moved & ✔︎ & * & * & ✔︎ & * & * & * & * & * \\ 
	\hline 
\end{tabular} 
\end{table}
\ex However, not all NegShift is treated equal. \citet[65f]{christensenInterfacesNegationSyntax2005}, speaking on Danish, claims that the "weight" of the NI plays a factor in whether or not NegShift occurs. 
	\ea
	\gll Jeg har \textit{intet}\textsubscript{o} hørt t\textsubscript{o}.\\
	I have nothing heard\\
	\glt  `I havn't heard anything.'
	\ex 
	\gll Jeg har [\textit{intet} \textit{nyt}]\textsubscript{o} hørt t\textsubscript{o}.\\
	I have nothing new heard\\
	\glt `I haven't heard anything new'
	\ex[*] {
	\gll Jeg har [\textit{intet} \textit{nyt} \textit{i} \textit{sagen}]\textsubscript{o} hørt t\textsubscript{o}.\\
	I have nothing new in case-\textsc{det} heard\\
	\glt `I haven't heard anything new about the case.'
	}
	\ex[*] {
	\gll Jeg har [\textit{intet} \textit{nyt} \textit{i} \textit{sagen} \textit{om} \textit{de} \textit{stjålne} \textit{malerier}]\textsubscript{o} hørt t\textsubscript{o}.\\
	I have nothing new in case-\textsc{det} about the stolen paintings heard\\
	}
	\glt `I haven't heard anything new in the case about the stolen paintings.'
	\z
\ex In those instances where the NI is too large one potential repair is to strand the PP while moving just the pronoun or using the negative particle \textit{ikke} and a NPI.
	\ea Jeg har \textit{intet}\textsubscript{i} hørt t\textsubscript{i} [\textsubscript{PP} i sagen om de stjålne malerier ].
	\ex Jeg har ikke hørt [ \textit{noget} i sagen om de stjålne malerier ].
	\z   
\ex This same behavior has also been remarked upon by \citet{penkaNegativeIndefinites2011} for Swedish.
	\ea 
	\gll Men mänskligheten har \textit{ingenting}\textsubscript{o} lärt sig t\textsubscript{o}.\\
	but mankind-the have nothing taught themselves\\
	\glt `But mankind haven't taught themselves anything.'
	\ex[?] {
	\gll Vi hade \textit{inga} \textit{grottor}\textsubscript{o} undersökt t\textsubscript{o}.\\
	we have no caves explored\\
	}
	\glt `We haven't explored any caves.'
	\z 
\ex My QP explores whether or not there is indeed this preference for NegShift of pronouns by conducting a study on the Swedish Culturomics Gigaword Corpus \citep{eideSwedishCulturomicsGigaword2016} and how this phenomenon might relate to prosodic analyses of pronominal obejct shift.
\ex One of the issues for this analysis is that one of the NI pronouns (\textit{ingen|inget|inga}) is identical to the NI determiner (\textit{ingen|inget|inga}).
\z 

%------------------------------------
\section{Engels 2012} \label{sec:ENGELS}
%------------------------------------

\ea \citet{engelsScandinavianNegativeIndefinites2012} provides a very nice table outlining the types of NegShift that is allowed in the different Scandinavian languages. I will first present this table and will have examples of the different types of shifting in subsections following the table.
\ex For each of the subsections they are further subdivided into whether or not the verb remains in situ or has moved.
\ex In the following table  ✔︎ indicates that NegShift occurs, * indicates that NegShift cannot occur, ? means that there was idiosyncratic  variation 
\z

\begin{table}[h!]
	\centering
	\caption{Distribution of NegShift across the different Scandinavian languages. WJ = West Jutlandic, Ic = Icelandic, Fa = Faroese, DaL = Danish Linguists, SwL = Swedish Linguists, Scan1 = literary/formal Mainland Scandinavian varieties, Scan2 = colloquial Mainland Scandinavian varieties and Norwegian}
\begin{tabular}{llccccccccc}
	\hline 
	NegShift across &  & WJ1 & WJ2 & Ic & Fa & DaL1 & DaL2 & SwL & Scan1 & Scan2 \\ 
	\hline 
	String-vacuous &  & ✔︎ & ✔︎ & ✔︎ & ✔︎ & ✔︎ & ✔︎ & ✔︎ & ✔︎ & ✔︎ \\ 
	Verb &  & ✔︎ & ✔︎ & ✔︎ & ✔︎ & ✔︎ & ✔︎ & ✔︎ & ✔︎ & * \\ 
	IO & verb in situ & ✔︎ & ✔︎ & ✔︎ & ✔︎ & ✔︎ & ✔︎ & ✔︎ & ✔︎ & * \\ 
	& verb moved & * & * & * & * & * & * & * & * & * \\ 
	Preposition & verb in situ & ✔︎ & ✔︎ & ✔︎ & ✔︎ & ? & ? & * & * & * \\ 
	& verb moved & ✔︎ & ✔︎ & ? & * & * & * & * & * & * \\ 
	Infinitive & verb in situ & ✔︎ & ✔︎ & ✔︎ & ✔︎ & ✔︎ & * & ? & * & * \\ 
	& verb moved & ✔︎ & * & * & ✔︎ & * & * & * & * & * \\ 
	\hline 
\end{tabular} 
\end{table}

%------------------------------------
\subsection{String-vacuous NegShift}
%------------------------------------

\ea According to \citeauthor{engelsScandinavianNegativeIndefinites2012} all varieties allow string-vacuous NegShift
	\ea 
	\gllllll Ég sagði \textit{ekkert} \I[VP t\textsubscript{v}  t\textsubscript{o}  ] Ic\\
			Eg segði \textit{einki} \I[VP t\textsubscript{v}  t\textsubscript{o}  ] Fa\\
			Jeg sagde \textit{ingenting} \I[VP t\textsubscript{v}  t\textsubscript{o}  ] Da \\
			Jag sa \textit{ingenting} \I[VP t\textsubscript{v}  t\textsubscript{o}  ] Sw\\
			Jeg sa \textit{ingenting} \I[VP t\textsubscript{v}  t\textsubscript{o}  ] No\\
			I said nothing\\
	\glt `I said nothing'
	\z 
\z

%------------------------------------
\subsection{NegShift across verbs}
%------------------------------------

\ea NegShift may cross a verb in situ in Insular Scandinavian languages
	\ea {
		\gll Ég hef \textit{engan} \textbf{séð} t\textsubscript{o}. \\
			I have nobody seen\\}\jambox{Ic}
	\glt `I haven't seen anybody.'%\rcommentg{Ic}
	\ex {
	\gll {Í dag} hevur Petur \textit{einki} \textbf{sagt} t\textsubscript{o}\\
	today has Peter nothing said\\}\jambox{Fa}
	\glt `Peter hasn't said anything today.'%\rcommentg{Fa}
	\z 
\ex It is claimed that NegShift across a verb in situ is found in literature to be stylistically marked. However, it is reported that there is dialectal variation (e.g., West Jutlandic). It was also deemed grammatical by Danish and Swedish linguists.
	\ea Manden havde måske \textit{ingenting} \textbf{sagt} t\textsubscript{o}.\jambox{Scan1}
	\ex[*] {{
		\gll Manden havde måske \textit{ingenting} \textbf{sagt} t\textsubscript{o}.\\
		man-the had probably nothing said\\}\jambox{Scan2}}
	\glt `The man hadn't said anything'
	\z 
\z 

%------------------------------------
\subsection{NegShift across IO}
%------------------------------------

\ea  NegShift across IO is permitted if the verb remains in situ for Icelandic, Faroese, West Jutlandic, and Scandinavian 1.
	\ea {
	\gll Jón hefur \textit{ekkert} \textbf{sagt} \textbf{Sveini} t\textsubscript{o}. \\
	Jón has nothing said Sveinn\\}\jambox{Ic}
	\glt `John hasn't told Sveinn anything'%\rcommentg{Ic}
	\ex {
	\gll {Í dag} hevur Petur \textit{einki} \textbf{givið} \textbf{Mariu} t\textsubscript{o}. \\
		today has Peter nothing given Mary\\}\jambox{Fa}
	\glt `Today, Peter hasn't given Mary anything.'%\rcommentg{Fa}
	\ex {
	\gll Jeg har \textit{ingen} \textit{bøger} \textbf{lånt} \textbf{børnene} t\textsubscript{o}.\\
	I have no books lent children-the\\}\jambox{WJ/Scan1}
	\glt `I haven't lent the children any books.'
	\ex[*] {{Jeg har \textbf{ingen bøger} lånt børnene t\textsubscript{o}.}\jambox{Scan2}}
	\z 
\ex If the verb has undergone V-to-T-to-C movement, NegShift is deemed ungrammatical in all varieties.
	\ea[*] {{
		\gll Jón sagði \textit{ekkert} \textbf{Sveini} t\textsubscript{o}\\
		Jón said nothing Sveinn\\}\jambox{Ic}}
	\glt Intended: `John didn't tell Sveinn anything.'%\rcommentg{Ic}
	\ex[*] {{
		\gll {Í gjár} gv Petur \textit{einki} \textbf{Mariu} t\textsubscript{o}\\
		yesterday gave Peter nothing Maria\\}\jambox{Fa}}
	\glt Intended: `Yesterday, Peter didn't give Mary anything.'%\rcommentg{Fa}
	\ex[*] {{
	\gll Jeg lånte \textit{ingen} \textit{bøger} \textbf{børnene} t\textsubscript{o}\\
	I lent no books children-the\\}\jambox{WJ/Scan1}}
	\glt Intended: `I didn't lend the children any books.'%\rcommentg{WJ/Scan1}
	\z 
\z 

%------------------------------------
\subsection{NegShift across preposition}
%------------------------------------

\ea NegShift across a preposition is not permitted in Mainland Scandinavian.
	\ea[*] {{
		\gll Jeg har \textit{ingen} \textbf{peget} \textbf{på} t\textsubscript{o} \\
		I have nobody pointed at\\}\jambox{Scan1/Scan2}}
		\glt Intended: `I haven't pointed at anybody.'
	\ex[*] {
	\gll Jeg pegede \textit{ingen} \textbf{på} t\textsubscript{o}\\
		I pointed nobody at\\}
	\glt Intended: `I didn't point at anybody.'
	\z 
\ex According to \posscitet{engelsScandinavianNegativeIndefinites2012} investigation there is considerable variation in this regard. It is permitted by the majority of Danish linguists at the University of Aarhus, but ungrammatical if the verb has moved.
	\ea[?] {{Jeg har \textit{ingen} \textbf{peget på} t\textsubscript{o}.}\jambox{DaL}}
	\ex[?] {Jeg har pegede \textit{ingen} \textbf{på} t\textsubscript{o}}
	\z 
\ex Permitted in Faroese if the the verb remains in situ.
	\ea {
		\gll {Í dag} hevur Petur \textit{ongan} \textbf{tosað} \textbf{við} t\textsubscript{o}.\\
		today has Peter nobody spoken with\\}\jambox{Fa}
	\glt `Today Peter hasn't spoken with anybody.'
	\ex[*] {
	\gll {Í dag} tosaði Petur \textit{ongan} \textbf{við} t\textsubscript{o}.\\
	today spoke Peter nobody with\\}
	\glt Intended: `Today Peter didn't speak with anybody.'
	\z 
\ex In Icelandic, NegShift is permitted if the verb remains in situ. If the verb has moved, it is still grammatical but degraded.
	\ea {
	\gll Ég hef \textit{engan} \textbf{talið} \textbf{við} t\textsubscript{o}.\\
	I have nobody spoke with\\}\jambox{Ic}
	\glt `I have spoken to nobody'
	\ex[?] {
	\gll Ég talaði \textit{engan} \textbf{við} t\textsubscript{o}.\\
	I spoke nobody with\\}
	\glt `I spoke with nobody.'
	\z 
\ex In West Jutlandic, NegShift is always permitted across prepositions.
	\ea {
		\gll Måske har hun \textit{ingen} \textbf{snakket} \textbf{med} t\textsubscript{o}.\\
		maybe has she nobody spoken with\\}\jambox{WJ}
	\glt `She maybe hasn't spoken with anybody.'
	\ex 
	\gll {I går} snakkede hun \textit{ingen} \textbf{med} t\textsubscript{o}.\\
	yesterday spoke she nobody with\\
	\glt `Yesterday, she didn't speak with anybody.'
	\z 
\z 

%------------------------------------
\subsection{NegShift out of infinitival clauses}
%------------------------------------

 \ea Icelandic allows NegShift out of infinitival clauses if the matrix verb has remained in situ.
	\ea {
	\gll Hún hefur \textit{engan} \textbf{lofað} \textbf{að} \textbf{kyssa} t\textsubscript{o}\\
	she has nobody promised to kiss\\}\jambox{Ic}
	\glt `She hasn't promised to kiss anybody.'
	\ex[*] {
	\gll Hún lofaði \textit{engan} \textbf{að} \textbf{kyssa} t\textsubscript{o}, var það nokkuð?\\
	she promised nobody to kiss {} was it rather\\}
	\glt Intended: `She didn't promise to kiss anybody, did she?'
	\z 
\ex It is also reported that Icelandic allows NegShift out of multiple infinitival clauses if the verb reamains in situ.
	\ea {
	\gll Petúr hefur \textit{engu} \textit{bréfi} \textbf{lofað} \textbf{að} \textbf{reyna} \textbf{að} \textbf{svara} t\textsubscript{o}.\\
	Peter has no letter promised to try to reply\\}\jambox{Ic}
	\glt`Peter hasn't promised to try to reply to any letter.'
	\z 
\ex NegShift out of a infinitival clause is also permitted by some of the Danish linguists and for some west Jutlandic speakers if the verb has remained in situ. This is also the case for some of the Swedish linguists.
	\ea {
	\gll Han har \textit{ingen} \textit{kager} \textbf{lovet} \textbf{at} \textbf{købe} t\textsubscript{o}.\\
	he has no cakes promised to buy\\}\jambox{DaL1/WJ2}
	\glt`He hasn't promised to buy any cakes.'
	\ex[*] {
	\gll Han lovede \textit{ingen} \textit{kager} \textbf{at} \textbf{købe} t\textsubscript{o}, vel?\\
	he promised no cakes to buy {} well\\}
	\glt Intended: `He didn't promise to buy any cakes, did he?'
	\ex[?] {{
	\gll Per har \textit{inga} \textit{tårta} \textbf{lovat} \textbf{att} \textbf{köpa} t\textsubscript{o}.\\
	 	Per has no cake promised to buy\\}\jambox{SwL}}
 	\glt `Per hasn't promised to buy any cake'
	\ex[*] {
	\gll Per lovade \textit{ingen} \textit{tårta} \textbf{att} \textbf{köpa} t\textsubscript{o}.\\
		Per promised no cake to buy\\}
	\glt Intended: `Per didn't promise to buy any cake.'
	\z 
\ex Other Danish linguists do not permit NegShift out of infinitive clauses at all. This is also true for Scandinavian1 and Scandinavian2.
	\ea[*] {{Han har \textit{ingen} \textit{kager} \textbf{lovet} \textbf{at} \textbf{købe} t\textsubscript{o}.}\jambox{DaL2}}
	\ex[*] {Han lovede \textit{ingen} \textit{kager} \textbf{at} \textbf{købe} t\textsubscript{o}, vel?}
	\ex[*] {{
	\gll Han har \textit{ingen} \textit{bøker} \textbf{prød} \textbf{å} \textbf{lese} t\textsubscript{o}.\\
	he has no books tried to read\\}\jambox{No}}
	\glt Intended: `He hasn't tried to read any books.'
	\ex[*] {
	\gll Han prøvde \textit{ingen} \textit{bøker} \textbf{å} \textbf{lese} t\textsubscript{o}.\\
		he tried no books to read\\}
	\glt Intended: `He didn't try to read any books.'
 	\z 
\ex For other West Jutlandic speakers and Faroese NegShift is permitted regardless if the matrix verb has moved or not.
	\ea {Han har \textit{ingen} \textit{kager} \textbf{lovet} \textbf{at} \textbf{købe} t\textsubscript{o}.}\jambox{WJ1}
	\ex Han lovede \textit{ingen} \textit{kager} \textbf{at} \textbf{købe} t\textsubscript{o}, vel?
	\ex {
	\gll Allarhelst hevur Petur \textit{einki} \textbf{roynt} \textbf{at} \textbf{eta} t\textsubscript{o}. \\
	probably has Peter nothing tried to eat\\}\jambox{Fa}
	\glt `Peter probably hasn't tried to eat anything.'
	\ex 
	\gll Allarhelst royndi Petur heldur \textit{einki} \textbf{at} \textbf{eta} t\textsubscript{o}.\\
	probably tried Peter also nothing to eat\\
	\glt `Peter probably also didn't try to eat anything.'
	\z 
\z 


%------------------------------------
\section{Scandinavian pronouns} \label{sec:PRONOUNS}
%------------------------------------

\ea There two different approaches that we can take when accounting for the syntactic structure of the negative indefinite pronouns in Swedish. The two positions are: (a) the pronoun is the head of a DP on its own; or (b) the pronoun resides in D and takes a null NP complete.\footnote{This second option could also assume that the pronoun originated in NP and moved to D prior to spell-out of the DP phase.}
	\begin{multicols}{2}
	\begin{xlist}
		\ex DP with no complement \label{ex:NNP}\\
		\begin{tikzpicture} 
		\tikzset{every tree node/.style={align=center,anchor=north}} 
		\Tree [.DP\\\emph{pronoun} ] 
		\end{tikzpicture}
		\vspace{2em}\\
		\ex DP with null NP complement \label{ex:DN} \\
		\begin{tikzpicture} 
		\tikzset{every tree node/.style={align=center,anchor=north}} 
		\Tree [.DP D\\\emph{pronoun} [.NP\\$\varnothing$ ] ]
		\end{tikzpicture}
	\end{xlist}
	\end{multicols}
\ex The reason this question is interesting is the presence of two homophonous NIs. 
	\ea the first is the negative indefinite determiner which always appears with a nominal pivot. This is equivalent to our NI determiner \emph{no}.
		\ea 
		\gll Jag bakade honom [\textsubscript{DP} \emph{inget} kaka ].\\
		I baked him ~ no cake\\
		\glt `I didn't bake him any cake.'  
		\ex 
		\gll Hon såg \emph{inga} barn.\\
		she saw no children\\
		\glt `She didn't see any children.'
		\z  
	\ex The second is the negative indefinite pronoun which does not take a nominal pivot and stands independent or with an adjunct, similar to English's \emph{no-one} or \emph{none}.
		\ea 
		\gll Jag bakade honom \emph{inget}.\\
		I baked him none\\
		\glt `I didn't bake him anything.'
		\ex 
		\gll  Jag har \textit{ingen} lånt barnen.\\
		I have none lent children-the\\
		\glt `I haven't lent the children anything.'
		\z 
	\z
\ex Evidence for one structure over the other comes from whether or not modification of the pronoun is allowed. 
	\ea Modification is taken to be additional material that changes what the set of things the NI refers too. 
	\ex This is predominately done with PPs, relative clauses, or infinitival clauses (when the NI is an object of an existential clause).
	\z 
\ex Because of this difference in behavior between determiners and nouns, we can gather evidence that would suggest one of the two syntactic structures is the more likely one for the Swedish negative indefinite pronouns if modification of these pronouns is present.
\ex Data will be drawn from across all of the Scandinavian languages because of the close similarity between the mainland Scandinavian languages syntactically.
\ex We observe in Danish that their negative indefinite pronouns are able to be modified with PPs and CPs as shown in \citet[218ff]{allanDanishComprehensiveGrammar1995}.
	\ea PPs
		\ea 
		\gll Det er \textit{intet} [\textsubscript{PP} i vej-an ].\\
			It is nothing ~ in way-\textsc{det}\\
		\glt `There is nothing wrong'
		\ex 
		\gll Jeg kender \textit{ingen} [\textsubscript{PP} her i byen ].\\
			I know no-one ~ here in town\\
		\glt `I know no one in this town.'
		\z
	\ex CPs
		\ea 
		\gll Der er \textit{ingen}, [\textsubscript{CP} der har set ham ].\\
		there is no-one, ~ who has seen him\\
		\glt `Nobody has seen him'
		\ex
		\gll Der er \textit{intet} [\textsubscript{CP} at være bange for ].\\
		There is nothing ~ to be afraid of\\
		\glt `There is nothing to be afraid of'
		\z 
	\z
\ex This same behavior is also observed in Swedish \citep[197ff]{holmesSwedishComprehensiveGrammar2013}.
	\ea PPs
		\ea \label{ex:inget}
		\gll Han äger \textit{inget} [\textsubscript{PP} av värde ].\\
			He owns nothing ~ of value\\
		\glt `He owns nothing of value.'
		\z 
	\ex CPs
		\ea
		\gll Jag såg \textit{ingen} [\textsubscript{CP} jag {kände igen} ].\\
		I saw no-one ~ I recognize\\
		\glt `I saw no one that I recognize.'
		\ex 
		\gll Jag har \textit{ingenting} [\textsubscript{CP} att säga ].\\
		I have nothing ~ to say\\
		\glt `I have nothing to say.'
		\z 
	\z
\ex Example (\ref{ex:inget}) shows the use of \textit{inget} instead of \textit{ingenting} which according to \citet{holmesSwedishComprehensiveGrammar2013} are in free variation.

\ex Based on this data, we see adjunct extraposition, which is assumed to be the result of the adjunction site of these PPs and CPs being within NP.
	\ea This is a working hypothesis and I am currently looking for further independent evidence that this is is the correct analysis for pronominals in Swedish.
	\z 
\ex Another reason that we can assume that (\ref{ex:DN}), repeated below with a adjunct, is the correct structure, is that if we assume that (\ref{ex:NNP}) is the structure for pronouns then we are left wondering how exactly these pronominal modifiers attach to our structure.
	\ea \attop{
	\begin{tikzpicture} 
	\tikzset{every tree node/.style={align=center,anchor=north}} 
	\Tree [.DP D\\\emph{pronoun} [.NP NP\\$\varnothing$ XP\\\emph{adjunct} ] ]
	\end{tikzpicture}
	}
	\z 
\z 


%------------------------------------
\section{Broekhuis 2020} \label{sec:BROEKHUIS}
%------------------------------------

\ea \cite{broekhuisUnificationObjectShift2020} explores the possibility that object shift and object scrambling are the same phenomenon and concludes that this is in fact the case.
\ex However, \citeauthor[417f]{broekhuisUnificationObjectShift2020} does point out that weak pronominal object shift behaves differently than full DP objects in what loci there are allowed to inhabit. In the case of weak pronominals they are required to appear outside of the \textit{v}P if there is no intervening phonological material (i.e., Holmberg's Generalization \cite{holmbergWordOrderSyntactic1986,holmbergRemarksHolmbergGeneralization1999}).
\ex \citeauthor{broekhuisUnificationObjectShift2020} does have some interesting discussion about the interaction of NegShift and pronominal OS. 
\ex Citing examples from \citet[163ff]{christensenInterfacesNegationSyntax2005}, Broekhuis shows this pair of examples:
	\ea \label{ex:NegShift}
	\gll Jeg har <ingen bøger> lånt hende <*ingen bøger>.\\
	I have no books lent her\\
	\glt `I haven't lent her any books
	\ex \label{ex:NegOS}
	\gll Jeg lånte \textit{henda} fraktisch \textit{ingen} \textit{bøger}.\\
	I lent her actually no books\\
	\glt `I didn't actually lend her any books.'
	\z
\ex In (\ref{ex:NegShift}), we see that when we have a negative object that it shifts to a position higher than the \textit{v}P if it were to remain in-situ as it would be ungrammatical and would require the use of \textit{ikke} `not' and the NPI \textit{nogen}.
	\ea
	\gll Jeg har \textit{ikke} lånt hende \textit{nogen} bøger.\\
	I have not lent her any books.\\
	\glt `I haven't lent her any books.'
	\z
\ex However, when the main verb has raised to C⁰ as in (\ref{ex:NegOS}) then the weak pronominal moves to a position higher than the adverb \textit{fraktisch} `actually'. The negative object is not able to move to the similar position that is higher than the adverb. Additionally, this results in OS > NegShift and according to \citeauthor{broekhuisUnificationObjectShift2020} this is a universal fact.
\ex This does help us see that that even though these two phenomena appear to be similar they are in fact slightly different, due to the differences in the where the different movement operations' target is.
\z 


%------------------------------------
\section{Valentine Bordal 2017} \label{sec:VB}
%------------------------------------

\ea \citet{valentinebordalNegationExistentialPredications2017} is a corpus study of \textit{Språkbanken}\footnote{\url{https://spraakbanken.gu.se/en}} looking at how existential predicates are negated in Swedish.
\ex According to \citeauthor{valentinebordalNegationExistentialPredications2017} declarative sentences are negated using the negative particle \textit{inte} only. 
	\ea
	\gll Anna är \textit{inte} doktor.\\
	Anna is not doctor\\
	\glt `Anna is not a doctor.'
	\z 
\ex If the verbal predicate has a non-lexical verb (i.e., modals and auxiliaries) then the sentence is negated with either the negative particle \textit{inte} or with a negative indefinite.\footnote{I disagree with her claim that NIs are only used with non-lexical verbs, because there are plenty of examples in the literature that show NIs negating sentences that contain lexical verbs (see \cite{engelsScandinavianNegativeIndefinites2012} and \cite{christensenInterfacesNegationSyntax2005} for examples).} 
	\ea
	\gll Jag har \textit{inte} några barn.\\
	I have not any children\\
	\glt `I don't have any children'
	\ex 
	\gll Jag har \textit{inga} barn.\\
	I have no children\\
	\glt `I don't have children.'
	\z 
\ex According to her results, NIs were the most frequent method for negating existential predicates. 
	
	\begin{table}[h!]
	\centering
	\caption{Frequency of negation strategy}
	\begin{tabular}{lll}
	\hline
	Negated existential & Raw count & Proportion\\
	\hline
	Existential predicates negated with an indefinite pronoun & 212 918 & 89\% \\
	Existential predicates negated with inte & 27 437 & 11\% \\
	\hline
	Total & 240 355 & 100\%  \\
	\hline
	\end{tabular}
	\end{table}

\ex Under her analysis of NIs, they are all treated as a pronomial which can or cannot appear with a nominal pivot.
	\ea This means that her "modifier" is best treated as a NI determiner which appears with a nominal pivot and her "head" is best treated as a pronoun which doesn't appear with a nominal pivot.
	\z 

	\begin{table}[h!]
	\centering
	\caption{Frequency of NI property}
	\begin{tabular}{lll}
	Syntactic property of the NI & Frequency & Proportion\\
	\hline
	Modifier & 687 & 78\% \\
	Head & 198 & 22\% \\
	\hline
	Total & 885 & 100\%  \\
	\hline 
	\end{tabular}
	\end{table}

\ex She also claims that out of all the negative indefinites, \textit{inga} only appears as a modifier, \textit{ingen} and \textit{inget} both appear equally as modifiers and pronominal heads, and \textit{ingenting} only ever appear as a head. 
\z 

%------------------------------------
\section{Zeijlstra 2011} \label{sec:ZEIJLSTRA}
%------------------------------------

\ea \cite{zeijlstraSyntacticallyComplexStatus2011} is interested in showing providing an analysis of the split-scope interpretation that exists for negative indefinites in Germanic languages. 
\ex Split-scope is evident when modals and other auxiliaries are present and the negation scopes higher than the modal/auxiliary's scope where the indefinite resides. 
\ex \citeauthor{zeijlstraSyntacticallyComplexStatus2011} assumes that this behavior is the result of the compositional status of negative indefinites similar to the claims made by \citet{iatridouNegativeDPsAMovement2011}. Unlike \citeauthor{iatridouNegativeDPsAMovement2011}, who simply claim that negation takes scope higher than the modal's scope and the indefinite scopes low, he claims that NIs are composed of a negative operator and an indefinite component.
\ex He further claims that the split-scope interpretation is the result of a copy-theory of movement where the indefinite interpretation is interpreted in the lower copy while the negative operator is interpreted in the higher copy after quantifier rising. 
\z 

%------------------------------------
\section{Interaction of NegShift and OS} \label{sec:NEG-OS}
%------------------------------------

\ea \citet{christensenInterfacesNegationSyntax2005} provides a clear and detailed summary of the interactions of OS and NegShift. His summary is detailed below in Table \ref{tab:OSNEGS}.
\ex In this table No\textsuperscript{+}/Sw\textsuperscript{+} represent some varieties of Norwegian and Swedish respectively in contrast to more standard Norwegian (No) and Swedish (Sw), FS represent the Swedish variety which is spoken by Swedes in Finland.
\begin{table}[h!]
\centering
\caption{Summary of OS and NegShift accoriding to \citet{christensenInterfacesNegationSyntax2005}.}
\label{tab:OSNEGS}
\begin{tabular}{lccccc}
\hline
IO-DO & Ic & Da/Fa & No/Sw & No\textsuperscript{+}/Sw\textsuperscript{+} & FS  \\
\hline 
Pron-Pron	&	+ +	&	+ +	&	\% \%	&	§ §	&	- -	\\
Pron-NegQP	&	+ +	&	+ +	&	+ +	&	+ +	&	- -	\\
NegQP-Pron	&	+ -	&	+ -	&	+ -	&	+ -	&	- -	\\
Pron-DP	&	+ \%	&	+ -	&	\% -	&	\% -	&	- -	\\
DP-Pron	&	\% -	&	- -	&	- -	&	- -	&	- -	\\
DP-DP	&	\% \%	&	- -	&	- -	&	- -	&	- -	\\
DP-NegQP	&	\% \%	&	- -	&	- -	&	- -	&	- -	\\
NegQP-DP	&	+ -	&	+ -	&	+ -	&	+ -	&	- -	\\
\hline 
\end{tabular}\\
(KEY: + = obligatory, - = blocked, \% = optional, § = optional and `non-parallel’)
\end{table}

\ex The sections on this table that are most interesting are those involving what \citeauthor{christensenInterfacesNegationSyntax2005} calls Negative Quantifier Phrases, which is equivalent to NIs. However, this does conflate NI determiners and NI pronouns into a single category.
\ex According to \citeauthor{christensenInterfacesNegationSyntax2005}, when the IO is a pronoun and the DO is a NegQP both obligatorily shift when the verb has been able to swift to C, (a), otherwise only the NegQP shifts, (b).
	\ea 
	\gll Jeg lånte \textit{hende(IO)} faktisk \textit{ingen} \textit{bøger(DO)}\\
	I lent her actually no books\\
	\glt `I actually didn't lend her any books'
	\ex 
	\gll Jeg har \textit{ingen} \textit{bøger(DO)} lånt \textit{hende(IO)}\\
	I have no books lend her\\
	\glt `I didn't lend her any books'
	\z 
\ex If, however, the IO is a NegQP and the DO is a pronoun then the pronoun is blocked from shifting. \textbf{This produces a freezing effect on OS \citep[164]{christensenInterfacesNegationSyntax2005}.}
	\ea 
	\gll Jeg lånte faktisk \textit{ingen(IO)} \textit{den(DO)}\\
	I lent actually no-one it\\
	\glt `I actually lent it to no-one.'
	\ex[*] {Jeg lånte \textit{den(DO)} faktisk \textit{ingen(IO)}} 
	\z
\ex This is actually a very important point for the question of the prosodic nature of the shifting. If we assume that these are moving to a position outside of the VP or are some sort of adjunct to VP\footnote{This is actually well argued by \citet{thrainssonSyntaxIcelandic2010} because if it was the head of a NegP that resided above the VP then we would assume that Neg, including NIs, should block head movement from occurring. Because we do not observe this and instead we see Neg and its reflexes acting more along the lines of actual adverbs. However, I am not entirely convinced as will be seen in (\ref{ex:HGfail})} then we would assume that OS should be allowed according to Holmberg's Generalization, (a). 
	\ea Object Shift cannot apply across a phonologically visible category asymmetrically c-commanding the object position except adjuncts \citep[15]{holmbergRemarksHolmbergGeneralization1999}.
	\z 
\ex \label{ex:HGfail} However, this might not actually be the case if we follow the logic from Holmberg's Generalization. HG requires that OS occur if there is not a phonologically visible category that asymmetrically c-commands the object's base position. Because OS is blocked, then Neg is a phonologically visible category that asymmetrically c-commands the object and is not an adjunct. This then suggests that it is a head of a phrase, where the head asymmetrically c-commands the object. This leaves us with two possibilities: either, (i) the NI does not move in these contexts; or (ii) the NI does move but to the head of some phrase that is in the domain of HG.
\ex Additionally, when the IO is a full DP and the DO is a NegQP the full DP IO induces a freezing effect on overt NegShift, which results in NegShift occurring covertly. This is also true when the IO is left in-situ, (b).
	\ea 
	\gll Ég gaf \emph{manninum(IO)} víst \textit{enga} \textit{gjöf(DO)}\\
	I gave man-the.\textsc{dat} \textsc{prt} no present\\
	\glt `But i di give the man the present'
	\ex Ég gaf víst \textit{manninum(IO)} \textit{enga} \textit{gjöf(DO)}
	\z
\z 

%------------------------------------
\section{Adverbial orders} \label{sec:ADV}
%------------------------------------

\ea There was some discussion about \emph{faktisch} `actually' being the wrong type of adverb to show movement of NIs and object pronouns. 
\ex However looking into the ordering of adverbs this actually seems to be the best type of adverb. This is because there is a strict ordering of adverbs in the middle field. 
\ex \citet{holmesIntroductionSociolinguistics2017} stats that there are four distinct levels that these adverbs are allowed to appear in.
	\ea The first level contains short modal adverbs\\

	\emph{då} `then'
	\ex The second level contains short pronominal adverbs or conjunctional adverbs\\
	\emph{därför} `therefore'
	\ex The third level contains longer modal adverbs\\
	\emph{faktiskt} `actually'
	\ex The fourth and last level are adverbs of negation\\
	\emph{aldrig} `never', \emph{inte} `not'
	\z 
\ex Examples of multiple adverbs
	\ea
	\gll De har ju\textsubscript{1} därför\textsubscript{2} faktiskt\textsubscript{3} aldrig\textsubscript{4} rest utomlands.\\
	they have certainly therefore actually never traveled abroad\\
	\glt `They have therefore actually never traveled abroad'
	\ex
	\gll Vi får väl\textsubscript{1} ändå\textsubscript{2} {trots allt}\textsubscript{3} inte\textsubscript{4} ge upp.\\
	we must nevertheless despite everything not give up\\
	\glt `We must nevertheless despite everything not give up'
	\z
\z


%------------------------------------
\section{Summary of where we are} \label{sec:RN}
%------------------------------------

%------------------------------------
\subsection{Distributional properties of NegShift versus OS} \label{sec:DISTRIBUTION}
%------------------------------------

\ea As mentioned above there are certain patterns that NegShift and OS share and differ in.
\z

%------------------------------------
\subsubsection{Similarities} \label{sec:SIM}
%------------------------------------

\ea Both OS and NegShift involve the movement of elements from their base position to a position that is to the left of the VP, as seen by the movement across negation/adverbials in the case of OS, (\ref{ex:OS}), and across the verb in the case of NegShift, (\ref{ex:NS}).
	\ea \label{ex:OS} 
	\gll Jag kyssade\textsubscript{v} henne\textsubscript{o} inte [\textsubscript{VP} t\textsubscript{v} t\textsubscript{o} ] \\
	I kiss.\Pst{} her \Neg{}\\
	\glt `I didn't kiss her.'
	\ex \label{ex:NS}
	\gll Jag har ingen\textsubscript{o} [\textsubscript{VP} kyssat t\textsubscript{o} ]\\
	I have no-one ~ kiss.\Pst{}.\Ptcp{} \\
	\glt `I haven't kissed anyone.''
	\z 
\ex Additionally, they are similar in that they both operate on pronouns, weak object pronouns for OS and negative indefinite pronouns for NegShift.
\ex As mentioned above NegShift has a preference for shifting pronouns \citep{christensenInterfacesNegationSyntax2005,penkaNegativeIndefinites2011}.
\z 

%------------------------------------
\subsubsection{Differences} \label{sec:DIF}
%------------------------------------

\ea There are two primary differences when it comes to OS and NegShift.
	\ea NegShift applies to full negative DPs such as \textit{inga böcker} `no books' in addition to pronouns. There is however a restriction in the size of the moved NI \citep{christensenInterfacesNegationSyntax2005,penkaNegativeIndefinites2011}. OS can only apply to weak object pronouns.
	\ex NegShift \emph{is not} subject to Holmberg's Generalization but is instead subject to an "Anti-Holmberg Effect" where it can shift across phonological material, whereas OS is subject to Holmberg's Generalization. 
	\z 
\z

%------------------------------------
\subsection{Cyclic Linearization} \label{sec:CL}
%------------------------------------

\ea Cyclic Linearization is a theory that was developed by \cite{foxCyclicLinearizationSyntactic2005} as a way to account for OS and Holmberg's Generalization.
\ex This theory works by stipulating that spell-out of the morpho-syntax is cyclic and order preserving, which means that as you spell-out each successive phase you need to ensure that whatever orders existed when that phase was spelled-out persist at the next phase's ordering restrictions. This theory also had the benefit of accounting for when OS was allowed or not allowed to occur. 
\ex This proposal was extended by \citet{foxCyclicLinearizationSyntactic2005} and \citet{engelsMicrovariationObjectPositions2011,engelsScandinavianNegativeIndefinites2012} to account for quantifier movement (QM), which NegShift is a subset of. 
\ex Under this proposal QM is subject to an ``Anti-Holmberg Effect'' or an ``Inverse Holmberg Effect'', which are identical in principle
	\ea Under Holmberg's Generalization, OS can only apply if the verb has undergone movement from V to T to C.
	\ex The Anti-Holmberg Effect explains that only when the verb remains in situ can we have QM, which is the result of the ordering operations between the different phases being in agreement. 
	\z
	\vspace{6pt} 

\ex In order to account for OS, \citeauthor{foxCyclicLinearizationSyntactic2005} propose that the during the spell-out of the VP phase the V is the leftmost element in its phase\footnote{The position of the V at the left-edge of the phase could be due to the movement of V to \textit{v} in which case it is actually the \textit{v}P that acts as the phase not the VP.} and at which point the ordering restrictions are in place which state that the V must precede the O. 

\ex At this point the V moves to T and then to C at this point the object is free to move to a position higher because the order that existed at the VP phase continues to hold. 
\ex OS and string-vacuous Neg-Shift
 	\ea {}[\textsubscript{CP} S \rnode{b1}V … [\textsubscript{NegP} \rnode{A1}O adv [\textsubscript{VP} \rnode{b2}t\textsubscript{v} \rnode{A2}t\textsubscript{o} ]]]
	\psset{linearc=2pt} 
	\ncbar[angle=90]{->}{A2}{A1}  
	\ncbar[angle=-90,linestyle=dashed]{->}{b2}{b1} 
	\vspace{6pt}
	\ex VP Ordering: \textbf{V>O}\\
		CP Ordering: S>V, \textbf{V>O}, O>adv, adv>VP
	\z

\ex If the DO were to move instead of the IO this would now result in the DO being ordered before the IO at the spell-out at the CP phase. By moving the DO, we now introduce a mismatch between the ordering restrictions at the VP phase and the CP phase explaining why such utterances are ungrammatical. 

\ex In the case of NegShift, where it is able to shift across various phonological material it is proposed that the NI first moves to the left edge of the VP before spell-out of that phase. 
\vspace{6pt}
	\ea {}[\textsubscript{CP} S aux … [\textsubscript{NegP} \rnode{A1}O [\textsubscript{VP} \rnode{A2}t\textsubscript{o}  V \rnode{A3}t\textsubscript{o} ]]]
	\psset{linearc=2pt} 
	\ncbar[angle=90]{->}{A3}{A2}  
	\ncbar[angle=90]{->}{A2}{A1}
	\ex VP Ordering: \textbf{O>V}\\
	CP Ordering: S>V, aux>O, O>adv, adv>VP → \textbf{O>V}
	\z

\ex The benefit of using Cyclic Linearization comes from being able to account for why certain orders are fixed throughout the entire derivation.
\z 


%------------------------------------
\section{Next steps} \label{sec:NEXT}
%------------------------------------
\begin{itemize}
	\item Write my entrance paper for 290.
	\item QP committee?
	\item Looking into the interaction of object shift and NegShift with a comparison of the size of shifted elements.
	\begin{itemize}
		\item Additionally, looking into \citet{christensenInterfacesNegationSyntax2005}, \citet{viknerObjectShiftScandinavian2017,viknerVerbMovementVariation2001}, and \citet{thrainssonSyntaxIcelandic2010} for further descriptions of NegShift.
		\item \citeauthor{thrainssonSyntaxIcelandic2010} also describes the fact that \emph{all} quantifiers undergo a similar process to NIs. This behavior has led him to claim that NegShift is just \textbf{Quantifier Scrambling}.
	\end{itemize}
	\item Looking into the Anti-Holmberg effect \citep{foxCyclicLinearizationSyntactic2005,engelsScandinavianNegativeIndefinites2012}.
\end{itemize}


%------------------------------------
%BIBLIOGRAPHY
%------------------------------------

%\singlespacing
% \nocite{*}
% \printbibliography[heading=bibintoc, keyword={QP1}]
\printbibliography[heading=bibintoc]
\end{document}