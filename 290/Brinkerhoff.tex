% !TEX TS-program = xelatex
% !TEX encoding = UTF-8 Unicode		

\documentclass[12pt, letterpaper]{article}

%%BIBLIOGRAPHY- This uses biber/biblatex to generate bibliographies according to the 
%%Unified Style Sheet for Linguistics
\usepackage[main=american, german]{babel}% Recommended
\usepackage{csquotes}% Recommended
\usepackage[backend=biber,
		style=unified,
		maxcitenames=3,
		maxbibnames=99,
		natbib,
		url=false]{biblatex}
\addbibresource{link_desktop.bib}	%	Use link_laptop.bib when on Laptop and link_desktop.bib when on desktop
\setcounter{biburlnumpenalty}{100}  % allow URL breaks at numbers
%\setcounter{biburlucpenalty}{100}   % allow URL breaks at uppercase letters
%\setcounter{biburllcpenalty}{100}   % allow URL breaks at lowercase letters

%%TYPOLOGY
\usepackage[svgnames]{xcolor} % Specify colors by their 'svgnames', for a full list of all colors available see here: http://www.latextemplates.com/svgnames-colors
%\usepackage[compact]{titlesec}
%\titleformat{\section}[runin]{\normalfont\bfseries}{\thesection.}{.5em}{}[.]
%\titleformat{\subsection}[runin]{\normalfont\scshape}{\thesubsection}{.5em}{}[.]
\usepackage[hmargin=1in,vmargin=1in]{geometry}  %Margins          
\usepackage{graphicx}	%Inserting graphics, pictures, images 		
\usepackage{stackengine} %Package to allow text above or below other text, Also helpful for HG weights 
\usepackage{fontspec} %Selection of fonts must be ran in XeLaTeX
\usepackage{amssymb} %Math symbols
\usepackage{amsmath} % Mathematical enhancements for LaTeX
\usepackage{setspace} %Linespacing
\usepackage{multicol} %Multicolumn text
\usepackage{enumitem} %Allows for continuous numbering of lists over examples, etc.
\usepackage{multirow} %Useful for combining cells in tablesbrew 
\usepackage{hanging}
\usepackage{fancyhdr} %Allows for the 
\pagestyle{fancy}
\fancyhead[L]{\textit{290 Presentation}} 
\fancyhead[R]{\textit{\today}} 
\fancyfoot[L,R]{} 
\fancyfoot[C]{\thepage} 
\renewcommand{\headrulewidth}{1pt}
\setlength{\headheight}{14.5pt} % ...at least 14.49998pt
% \usepackage{fourier} % This allows for the use of certain wingdings like bombs, frowns, etc.
% \usepackage{fourier-orns} %More useful symbols like bombs and jolly-roger, mostly for OT
\usepackage[colorlinks,allcolors={black},urlcolor={blue}]{hyperref} %allows for hyperlinks and pdf bookmarks
% \usepackage{url} %allows for urls
% \def\UrlBreaks{\do\/\do-} %allows for urls to be broken up
\usepackage[normalem]{ulem} %strike out text. Handy for syntax
\usepackage{tcolorbox}

%%FONTS
\setmainfont{Libertinus Serif}
\setsansfont{Libertinus Sans}
\setmonofont[Scale=MatchLowercase]{Libertinus Mono}

%%PACKAGES FOR LINGUISTICS
%\usepackage{OTtablx} %Generating tableaux with using TIPA
\usepackage[noipa]{OTtablx} % Use this one generating tableaux without using TIPA
%\usepackage[notipa]{ot-tableau} % Another tableau drawing packing use for posters.
% \usepackage{linguex} % Linguistic examples
\usepackage{langsci-gb4e} % Language Science Press' modification of gb4e
% \usepackage{langsci-avm} % Language Science Press' AVM package
\usepackage{tikz} % Drawing Hasse diagrams
%\usepackage{pst-asr} % Drawing autosegmental features
\usepackage{pstricks} % required for pst-asr, OTtablx, pst-jtree.
% \usepackage{pst-jtree} 	% Syntax tree draawing software
\usepackage{tikz-qtree}	% Another syntax tree drawing software. Uses bracket notation.
% \usepackage{forest}	% Another syntax tree drawing software. Uses bracket notation.
%\usepackage{ling-macros} % Various linguistic macros. Does not work with linguex.
%\usepackage{covington} % Another linguistic examples package.
\usepackage{leipzig}

%%TITLE INFORMATION
\title{Motivations for Scandinavian Negative Indefinite Shift}
\author{Mykel Loren Brinkerhoff}
\date{\today}

\newcommand{\sub}[1]{\textsubscript{#1}}
\newcommand{\supr}[1]{\textsuperscript{#1}}

\begin{document}
	
%%If using linguex, need the following commands to get correct LSA style spacing
%% these have to be after  \begin{document}
	% \setlength{\Extopsep}{6pt}
	% \setlength{\Exlabelsep}{9pt}		%effect of 0.4in indent from left text edge
%%
	
%% Line spacing setting. Comment out the line spacing you do not need. Comment out all if you want single spacing.
%	\doublespacing
%	\onehalfspacing
	
\begin{center}
	{\Large \textbf{Motivations for Scandinavian Negative Indefinite Shift}}\\
	\vspace{6pt}
	Mykel Loren Brinkerhoff\\
\end{center}
%\maketitle
%\maketitleinst
\thispagestyle{fancy}

% \tableofcontents

% %------------------------------------
% \section*{Outline of Handout} \label{sec:OUTLINE}
% %------------------------------------
% \begin{itemize}
% 	\item Overview of Negative Indefinite Shift in §\ref{sec:NEGSHIFT}
% 	\item Discussion of Scandinavian Pronoun structure in §\ref{sec:PRONOUNS}
% 	% \item Discussion of \cite{broekhuisUnificationObjectShift2020} in §\ref{sec:BROEKHUIS}
% 	\item Discussion of \cite{valentinebordalNegationExistentialPredications2017} in §\ref{sec:VB}
% 	\item Discussion of \cite{zeijlstraSyntacticallyComplexStatus2011} in §\ref{sec:ZEIJLSTRA}
% 	\item Next steps in §\ref{sec:NEXT}
% \end{itemize}

%------------------------------------
\section{Introduction} \label{sec:NEGSHIFT}
%------------------------------------
\ea Negative Shifting (NegShift) is a process in the Scandinavian languages where a negative indefinite (NI) obligatorily shifts to a position outside of the VP.
	\ea
	\gll Manden havde måske \textit{ingenting} [\textsubscript{VP} sagt t\textsubscript{o} ].\\
	man-the had probably nothing ~ said\\
	\glt `The man hadn't said anything.'
	\ex 
	\gll Jeg har \textit{ingen} \textit{bøger} [\textsubscript{VP} lånt børnene t\textsubscript{o}.]\\
	I have no books ~ lent children-the\\
	\glt `I haven't lent the children any books.'
	\z

\ex This process occurs to all NIs and is permissible from a large number of different contexts, depending on the variety and register (see Table \ref{tab:Distribution}). 

\ex This process bears some resemblance to Scandinavian Object Shift (OS), which is where a weak object pronoun shifts to a position outside of the verb phrase \citep{holmbergWordOrderSyntactic1986,holmbergRemarksHolmbergGeneralization1999}.
	\ea  {
	\gll Jag kyssade\textsubscript{v} henne\textsubscript{o} inte [\textsubscript{VP} t\textsubscript{v} t\textsubscript{o} ] \\
	I kiss.\Pst{} her \Neg{}\\} \jambox*{Sw}
	\glt `I didn't kiss her.'
	\z 

\ex There has been some evidence that OS is prosodically driven \citep{erteschik-shirSoundPatternsSyntax2005,erteschik-shirScandinavianObjectShift2017,erteschik-shirVariationMainlandScandinavian2020,brinkerhoffMATCHINGPhrasesNorwegian2021}. 

\ex Despite similarities between NegShift and OS, NegShift doesn't directly correlate to the accounts of OS. 

\ex Chiefly, a wider range of material is allowed to undergo NegShift, which includes both pronouns and full DPs, whereas only prosodically weak object pronouns are allowed to undergo OS.

\ex However, not all NegShift is treated equal. \citet[65f]{christensenInterfacesNegationSyntax2005}, speaking on Danish, claims that the ``weight" of the NI plays a factor in whether or not NegShift occurs. 
	\ea
	\gll Jeg har [\textit{intet} \textit{nyt}]\textsubscript{o} hørt t\textsubscript{o}.\\
	I have nothing new heard\\
	\glt `I haven't heard anything new.'
	\ex[*] {
	\gll Jeg har [\textit{intet} \textit{nyt} \textit{i} \textit{sagen}]\textsubscript{o} hørt t\textsubscript{o}.\\
	I have nothing new in case-\textsc{det} heard\\
	\glt `I haven't heard anything new about the case.'
	}
	\z

\begin{tcolorbox}[width=\linewidth]
In this paper, I explore to what extent prosodic factors play in NegShift and claim that this weight sensitivity has similarities to Heavy NP Shift.
\end{tcolorbox}	
% \ex In those instances where the NI is too large one potential repair is to strand the PP while moving just the pronoun or using the negative particle \textit{ikke} and a NPI.
% 	\ea Jeg har \textit{intet}\textsubscript{i} hørt t\textsubscript{i} [\textsubscript{PP} i sagen om de stjålne malerier ].
% 	\ex Jeg har ikke hørt [ \textit{noget} i sagen om de stjålne malerier ].
% 	\z   
% \ex This same behavior has also been remarked upon by \citet{penkaNegativeIndefinites2011} for Swedish.
% 	\ea 
% 	\gll Men mänskligheten har \textit{ingenting}\textsubscript{o} lärt sig t\textsubscript{o}.\\
% 	but mankind-the have nothing taught themselves\\
% 	\glt `But mankind haven't taught themselves anything.'
% 	\ex[?] {
% 	\gll Vi hade \textit{inga} \textit{grottor}\textsubscript{o} undersökt t\textsubscript{o}.\\
% 	we have no caves explored\\
% 	}
% 	\glt `We haven't explored any caves.'
% 	\z 
% \ex My QP explores whether or not there is indeed this preference for NegShift of pronouns by conducting a study on the Swedish Culturomics Gigaword Corpus \citep{eideSwedishCulturomicsGigaword2016} and how this phenomenon might relate to prosodic analyses of pronominal obejct shift.
% \ex One of the issues for this analysis is that one of the NI pronouns (\textit{ingen|inget|inga}) is identical to the NI determiner (\textit{ingen|inget|inga}).
\z 

%------------------------------------
\section{Distributional properties of NegShift versus OS} \label{sec:ENGELS}
%------------------------------------

\ea As mentioned above there are some similarities between OS and NegShift with them both operating on pronouns and moving them to a position outside of the VP. 
	\ea \label{ex:OS}
		\gll Jag kyssade\textsubscript{v} henne\textsubscript{o} inte [\textsubscript{VP} t\textsubscript{v} t\textsubscript{o} ] \\
		I kiss.\Pst{} her \Neg{}\\
		\glt `I didn't kiss her.'
	\ex \label{ex:NS}
		\gll Jag har ingen\textsubscript{o} [\textsubscript{VP} kyssat t\textsubscript{o} ]\\
		I have no-one ~ kiss.\Pst{}.\Ptcp{} \\
		\glt `I haven't kissed anyone.'
	\z 
	
\ex However, NegShift also operates on full negative DPs and is not subject to Holmberg's Generalization \citep{foxCyclicLinearizationSyntactic2005,engelsScandinavianNegativeIndefinites2012}
		\ea Verb in-situ NegShift\\
		{\gll Ég hef \textit{engan} [\textsubscript{VP} \textbf{séð} t\textsubscript{o}	].\\
			I have nobody ~ seen\\}\jambox*{Ic}
		\glt `I haven't seen anybody.'
		\ex String vacuous NegShift\\
		{\gll Jag \textbf{sa} \textit{ingenting} [\textsubscript{VP} t\textsubscript{v}  t\textsubscript{o}  ].\\
			I said nothing\\}\jambox*{Sw}
		\glt `I said nothing'
		\z
\ex Additional evidence from \citet{engelsScandinavianNegativeIndefinites2012} shows that NegShift is allowed out of a wider range of positions than OS, which are summarized in Table \ref{tab:Distribution}
\ex In the following table  ✔︎ indicates that NegShift occurs, * indicates that NegShift cannot occur, ? means that there was idiosyncratic  variation 
\z

\begin{table}[!ht]
	\centering
	\label{tab:Distribution}
	\caption{Distribution of NegShift across the different Scandinavian languages. WJ = West Jutlandic, Ic = Icelandic, Fa = Faroese, DaL = Danish Linguists, SwL = Swedish Linguists, Scan1 = literary/formal Mainland Scandinavian varieties, Scan2 = colloquial Mainland Scandinavian varieties and Norwegian}
\begin{tabular}{llccccccccc}
	\hline 
	NegShift across &  & WJ1 & WJ2 & Ic & Fa & DaL1 & DaL2 & SwL & Scan1 & Scan2 \\ 
	\hline 
	String-vacuous &  & ✔︎ & ✔︎ & ✔︎ & ✔︎ & ✔︎ & ✔︎ & ✔︎ & ✔︎ & ✔︎ \\ 
	Verb &  & ✔︎ & ✔︎ & ✔︎ & ✔︎ & ✔︎ & ✔︎ & ✔︎ & ✔︎ & * \\ 
	IO & verb in situ & ✔︎ & ✔︎ & ✔︎ & ✔︎ & ✔︎ & ✔︎ & ✔︎ & ✔︎ & * \\ 
	& verb moved & * & * & * & * & * & * & * & * & * \\ 
	Preposition & verb in situ & ✔︎ & ✔︎ & ✔︎ & ✔︎ & ? & ? & * & * & * \\ 
	& verb moved & ✔︎ & ✔︎ & ? & * & * & * & * & * & * \\ 
	Infinitive & verb in situ & ✔︎ & ✔︎ & ✔︎ & ✔︎ & ✔︎ & * & ? & * & * \\ 
	& verb moved & ✔︎ & * & * & ✔︎ & * & * & * & * & * \\ 
	\hline 
\end{tabular} 
\end{table}

%------------------------------------
\section{Interaction of NegShift and OS} \label{sec:NEG-OS}
%------------------------------------

\ea As seen in Table \ref{tab:Distribution} we see a sharp contrast involving indirect objects


\ex \citet{christensenInterfacesNegationSyntax2005} provides a clear and detailed summary of the interactions of NegShift and indirect objects. His summary is detailed below in Table \ref{tab:OSNEGS}.
\ex In this table No\textsuperscript{+}/Sw\textsuperscript{+} represent some varieties of Norwegian and Swedish respectively in contrast to more standard Norwegian (No) and Swedish (Sw), FS represent the Swedish variety which is spoken by Swedes in Finland.
\begin{table}[h!]
\centering
\caption{Summary of OS and NegShift accoriding to \citet{christensenInterfacesNegationSyntax2005}.}
\label{tab:OSNEGS}
\begin{tabular}{lccccc}
\hline
IO-DO & Ic & Da/Fa & No/Sw & No\textsuperscript{+}/Sw\textsuperscript{+} & FS  \\
\hline 
Pron-Pron	&	+ +	&	+ +	&	\% \%	&	§ §	&	- -	\\
Pron-NegQP	&	+ +	&	+ +	&	+ +	&	+ +	&	- -	\\
NegQP-Pron	&	+ -	&	+ -	&	+ -	&	+ -	&	- -	\\
Pron-DP	&	+ \%	&	+ -	&	\% -	&	\% -	&	- -	\\
DP-Pron	&	\% -	&	- -	&	- -	&	- -	&	- -	\\
DP-DP	&	\% \%	&	- -	&	- -	&	- -	&	- -	\\
DP-NegQP	&	\% \%	&	- -	&	- -	&	- -	&	- -	\\
NegQP-DP	&	+ -	&	+ -	&	+ -	&	+ -	&	- -	\\
\hline 
\end{tabular}\\
(KEY: + = obligatory, - = blocked, \% = optional, § = optional and `non-parallel’)
\end{table}

\ex The sections on this table that are most interesting are those involving what \citeauthor{christensenInterfacesNegationSyntax2005} calls Negative Quantifier Phrases, which is equivalent to NIs. However, this does conflate NI determiners and NI pronouns into a single category.
\ex According to \citeauthor{christensenInterfacesNegationSyntax2005}, when the IO is a pronoun and the DO is a NegQP both obligatorily shift when the verb has been able to swift to C, (a), otherwise only the NegQP shifts, (b).
	\ea 
	\gll Jeg lånte \textit{hende(IO)} faktisk \textit{ingen} \textit{bøger(DO)}\\
	I lent her actually no books\\
	\glt `I actually didn't lend her any books'
	\ex 
	\gll Jeg har \textit{ingen} \textit{bøger(DO)} lånt \textit{hende(IO)}\\
	I have no books lend her\\
	\glt `I didn't lend her any books'
	\z 
\ex If, however, the IO is a NegQP and the DO is a pronoun then the pronoun is blocked from shifting. This produces a freezing effect on OS \citep[164]{christensenInterfacesNegationSyntax2005}.
	\ea 
	\gll Jeg lånte faktisk \textit{ingen(IO)} \textit{den(DO)}\\
	I lent actually no-one it\\
	\glt `I actually lent it to no-one.'
	\ex[*] {Jeg lånte \textit{den(DO)} faktisk \textit{ingen(IO)}} 
	\z
\ex This is actually a very important point for the question of the prosodic nature of the shifting. If we assume that these are moving to a position outside of the VP or are some sort of adjunct to VP then we would assume that OS should be allowed according to Holmberg's Generalization, (a). 
	\ea Object Shift cannot apply across a phonologically visible category asymmetrically c-commanding the object position except adjuncts \citep[15]{holmbergRemarksHolmbergGeneralization1999}.
	\z 

\ex \citet{broekhuisUnificationObjectShift2020} does have some interesting discussion about the interaction of NegShift and pronominal OS. 

\ex Citing examples from \citet[163ff]{christensenInterfacesNegationSyntax2005}, Broekhuis shows this pair of examples:
	\ea \label{ex:NegShift}
		\ea \label{ex:HER}
		\gll Jeg lånte \textit{hende(IO)} faktisk \textit{ingen} \textit{bøger(DO)}\\
		I lent her actually no books\\
		\glt `I actually didn't lend her any books'
		\ex \label{ex:BOOKS}
		\gll Jeg har \textit{ingen} \textit{bøger(DO)} lånt \textit{hende(IO)}\\
		I have no books lend her\\
		\glt `I didn't lend her any books'
		\z 
	\z

\ex In (\ref{ex:NegShift}), we see that when we have a negative object that it shifts to a position higher than the \textit{v}P if it were to remain in-situ as it would be ungrammatical and would require the use of \textit{ikke} `not' and the NPI \textit{nogen}.
	\ea
	\gll Jeg har \textit{ikke} lånt hende \textit{nogen} bøger.\\
	I have not lent her any books.\\
	\glt `I haven't lent her any books.'
	\z
\ex However, when the main verb has raised to C⁰ as in (\ref{ex:HER}) then the weak pronominal moves to a position higher than the adverb \textit{fraktisch} `actually'. The negative object is not able to move to the similar position that is higher than the adverb. Additionally, this results in OS > NegShift and according to \citeauthor{broekhuisUnificationObjectShift2020} this is a universal fact.
\ex This does help us see that that even though these two phenomena appear to be similar they are in fact slightly different, due to the differences in the where the different movement operations' target is.
\z 

%------------------------------------
\section{Prosodic restrictions on NegShift} \label{sec:PROSODY}
%------------------------------------

\ea As noted earlier not all NegShift is treated equal. \citet[65f]{christensenInterfacesNegationSyntax2005}, speaking on Danish, claims that the ``weight" of the NI plays a crucial factor in whether or not NegShift occurs. 
	\ea 
	\gll Jeg har \textit{intet}\textsubscript{o} hørt t\textsubscript{o}.\\
	I have nothing heard\\
	\glt  `I havn't heard anything.'
	\ex 
	\gll Jeg har [\textit{intet} \textit{nyt}]\textsubscript{o} hørt t\textsubscript{o}.\\
	I have nothing new heard\\
	\glt `I haven't heard anything new'
	\ex[*] {
	\gll Jeg har [\textit{intet} \textit{nyt} \textit{i} \textit{sagen}]\textsubscript{o} hørt t\textsubscript{o}.\\
	I have nothing new in case-\textsc{det} heard\\
	\glt `I haven't heard anything new about the case.'
	}
	\ex[*] {
	\gll Jeg har [\textit{intet} \textit{nyt} \textit{i} \textit{sagen} \textit{om} \textit{de} \textit{stjålne} \textit{malerier}]\textsubscript{o} hørt t\textsubscript{o}.\\
	I have nothing new in case-\textsc{det} about the stolen paintings heard\\
	}
	\glt `I haven't heard anything new in the case about the stolen paintings.'
	\z

\ex In those instances where the NI is too large one potential repair is to strand the PP while moving just the pronoun or using the negative particle \textit{ikke} and a NPI.
	\ea Jeg har \textit{intet}\textsubscript{i} hørt t\textsubscript{i} [\textsubscript{PP} i sagen om de stjålne malerier ].
	\ex Jeg har \textit{ikke} hørt [ \textit{noget} i sagen om de stjålne malerier ].
	\z 
 
\ex This same behavior has also been remarked upon by \citet{penkaNegativeIndefinites2011} for Swedish.
	\ea 
	\gll Men mänskligheten har \textit{ingenting}\textsubscript{o} lärt sig t\textsubscript{o}.\\
	but mankind-the have nothing taught themselves\\
	\glt `But mankind haven't taught themselves anything.'
	\ex[?] {
	\gll Vi hade \textit{inga} \textit{grottor}\textsubscript{o} undersökt t\textsubscript{o}.\\
	we have no caves explored\\
	}
	\glt `We haven't explored any caves.'
	\z 

\z 
%------------------------------------
\section{Copy and partial deletion account} \label{sec:ZEIJLSTRA}
%------------------------------------

\ea One way to account for this behavior is following \cite{zeijlstraSyntacticallyComplexStatus2011} account for the split-scope that these NIs introduce in Germanic languages.\footnote{See \citet{svenoniusStrainsNegationNorwegian2002} which shows that there is some differences in interpretation depending on if NegShift has occurred or not.} 
\ex Split-scope is evident when modals and other auxiliaries are present and the negation scopes higher than the modal/auxiliary's scope where the indefinite resides. 
\ex \citeauthor{zeijlstraSyntacticallyComplexStatus2011} assumes that this behavior is the result of the compositional status of negative indefinites similar to the claims made by \citet{iatridouNegativeDPsAMovement2011}. Unlike \citeauthor{iatridouNegativeDPsAMovement2011}, who simply claim that negation takes scope higher than the modal's scope and the indefinite scopes low, he claims that NIs are composed of a negative operator and an indefinite component.
\ex He further claims that the split-scope interpretation is the result of a copy-theory of movement where the indefinite interpretation is interpreted in the lower copy while the negative operator is interpreted in the higher copy after quantifier rising. 

\ex He claims that the when we move items we copy them entirely and remerge them into the syntactic structure. This results in multiple copies of the item being in the syntactic structure. Following \citet{fanselowRemarksEconomyPronunciation2001,fanselowDistributedDeletion2002} only at a later stage, after the syntactic structure has been spell-out, parts of the copy are removed in PF or LF. This could result in either a partial or complete deletion of any of these copied-elements at any location that they are found. 

\ex For example, we see this in the following syntactic structure. 
	\ea \attop{
	\resizebox*{\linewidth}{!}{\begin{tikzpicture} 
	\tikzset{every tree node/.style={align=center,anchor=north}} 
	\Tree [.CP 	DP\\\emph{Anna} 
				[.C´ C\\\emph{har} 
				[.TP <DP> 
					[.T´ T 
						[.NegP  
							[.DP \edge[roof]; {intet i sagen om de stjålne malerier}
							]
							[.vP <DP>
								[.v´ hørt   
									[.VP <V> 
										[.DP \edge[roof]; {intet i sagen om de stjålne malerier}
										]
									]
								]
							]    
						]
					] 
				]
				] 
			] 
	\end{tikzpicture}}
	}
	\z

\ex At this point during PF, part of the higher copy is elided leaving only the NI pronoun. In the lower copy \emph{intet} is deleted, resulting in the following structure.
\ea \label{ex:tree} \attop{
	\resizebox*{\linewidth}{!}{
	\begin{tikzpicture} 
	\tikzset{every tree node/.style={align=center,anchor=north}} 
	\Tree [.CP 	DP\\\emph{Anna} 
				[.C´ C\\\emph{har} 
				[.TP <DP> 
					[.T´ T 
						[.NegP  
							[.DP \edge[roof]; {intet \sout{i sagen om de stjålne malerier}}
							]
							[.vP <DP>
								[.v´ hørt   
									[.VP <V> 
										[.DP \edge[roof]; {\sout{intet} i sagen om de stjålne malerier}
										]
									]
								]
							]    
						]
					] 
				]
				] 
			] 
	\end{tikzpicture}}
	}
	\z

\ex The question arises as to what the motivation is for determining what amount of material is deleted during this process. I propose that the solution is similar to Heavy NP Shift and can be modeled using Optimality Theory \citep{princeOptimalityTheoryConstraint2004} and making reference to the prosody of these constituents.

\z 

%------------------------------------
\section{Prosodic motivations for Heavy NP Shift} \label{sec:HNPS}
%------------------------------------

\ea According to \citet{anttilaRoleProsodyEnglish2010}, prosody plays a supporting role in linearization and can be accounted for through OT style constraints on prosodic well-formedness. 
\ex \citeauthor{anttilaRoleProsodyEnglish2010} are concerned with explaining the three way patterning that dative constructions have and how the different verbal arguments are linearized with respect to each other. 
\ex The three different constructions that are possible in dative constructions are:
	\ea Double object constructions
	\ex Prepositional constructions
	\ex Heavy NP Shift constructions
	\z

\ex Of crucial interest is how they explain Heavy NP Shift and what their criteria for determining weight.

\ex \citet[949]{anttilaRoleProsodyEnglish2010} say that the ``''weight" of an NP is ``a function of the number of lexically stressed words in [the constituent]".

\ex This means that the more lexical stresses a constituent has the heavier it is. This definition, crucially, leaves out functional items and pronouns because they lack lexical stress. 

\ex Using this definition for weight, we can explain the behavior Danish and Swedish as discussed in §\ref{sec:NEGSHIFT}. This means that Danish and Swedish have an upper bound on the weight of the shifted element. In the case of Danish only elements of weight ≤1 are allowed to shift from their base generated position. 

\ex Swedish on the other hand has a greater weight allowance before it is treated as ungrammatical. 
\z 

%------------------------------------
\subsection{OT Account for Heavy NP Shift} \label{sec:HNPS}
%------------------------------------

\ea Based on \citet{anttilaRoleProsodyEnglish2010}, prosody plays a supporting role in linearization and can be accounted for through OT style constraints on prosodic well-formedness. 

\ex This can be accounted for using Match Theory \citep{selkirkClauseIntonationalPhrase2009,selkirkSyntaxPhonologyInterface2011}.

\ex Following \citet{myrbergSisterhoodProsodicBranching2013,myrbergProsodicWordSwedish2013,myrbergProsodicHierarchySwedish2015}, I assume that the subject in Scandinavian languages forms its own phonological phrase, if it is not a pronoun, separate from the rest of the clause. 
	\ea Simplified prosodic structure for (\ref{ex:tree})\\
	% \resizebox*{\linewidth}{!}{
	\begin{tikzpicture} 
	\tikzset{every tree node/.style={align=center,anchor=north}} 
	\Tree [.ι [.φ\\Anna ] [.φ \edge[roof]; {har intet hørt i sagen om de stjålne malerier} ] ] 
	\end{tikzpicture}%}

	\z

\ex Crucially, what we are concerned with the weight of the item shifting. This can be accounted for using a type of \textsc{NoShift} \citep{bennettLightestRightApparently2016} which is sensitive to lexical stresses, following \posscitet{anttilaRoleProsodyEnglish2010} definition for phonological weight.

\ex \textsc{NoShift(Stress)}:
Assign one violation for every lexical stress that is not in the same linear order as in the input.

\ex This constraint operates by considering where a word bearing lexical stress is located in the input and whether it follows the same linear order as the input and assigns a violation for every lexical stress that is not in the same linear order.

\ex If we take the input of (\ref{ex:tree}) this constraint should assign a violation for every item bearing lexical stress that has been relinearized. 

\ex Using OT we can model how this would behave with the input of (\ref{ex:tree})
	\resizebox*{\linewidth}{!}{
	\begin{OTtableau}{3}
		\OTdashes{1,2}
		\OTtoprow [ ] {\textsc{Match(XP,φ)}, \textsc{Match(φ,XP)}, \textsc{NoShift(Stress)}}
		\OTcandrow [\OThand] {( Anna\sub{ω} )\sub{φ} (har \textit{intet}\sub{ω} hørt i sagen om de stjålne malerier)\sub{φ}} { 1 , 1 , 1} 
		\OTcandrow [\OThand] {( Anna\sub{ω} )\sub{φ} (har \textit{intet nyt}\sub{ω} hørt i sagen om de stjålne malerier)\sub{φ}} { 1 , 1 , 1} 
		\OTcandrow [\OTface*] {( Anna\sub{ω} )\sub{φ} (har \textit{intet nyt\sub{ω} i sagen}\sub{ω} hørt om de stjålne malerier)\sub{φ}} { 1 , 1 , 2!} 
		\OTcandrow [\OTface*] {( Anna\sub{ω} )\sub{φ} (har \textit{intet i sagen\sub{ω} om de stjålne\sub{ω} malerier\sub{ω}} hørt )\sub{φ}} { 1 , 1 , } 
	\end{OTtableau}
	}
\z 
%------------------------------------
\subsection{Alternative accounts} \label{sec:MaximalW}
%------------------------------------

\ea However this could also be something to do either the maximal or minimal prosodic word that is shifting. This comes down to the fact that Swedish tonal accents are associated with the maximal prosodic word while stress is associated with the minimal prosodic word \citep{myrbergProsodicWordSwedish2013}. 

\ex This is also the case in Danish where word stress is associated with the minimal prosodic word and Danish stød is associated with the maximal prosodic word \citep[see discussion of stød placement in][]{kalivodaProsodicRecursionPseudocyclicity2018}.

\z 

%------------------------------------
\section{To be continued...}
%------------------------------------


%------------------------------------
%BIBLIOGRAPHY
%------------------------------------

%\singlespacing
% \nocite{*}
% \printbibliography[heading=bibintoc, keyword={QP1}]
\printbibliography[heading=bibintoc]

\part*{Appendix}

%------------------------------------
\section*{Cyclic Linearization} \label{sec:CL}
%------------------------------------

\ea Cyclic Linearization is a theory that was developed by \cite{foxCyclicLinearizationSyntactic2005} as a way to account for OS and Holmberg's Generalization.
\ex This theory works by stipulating that spell-out of the morpho-syntax is cyclic and order preserving, which means that as you spell-out each successive phase you need to ensure that whatever orders existed when that phase was spelled-out persist at the next phase's ordering restrictions. This theory also had the benefit of accounting for when OS was allowed or not allowed to occur. 
\ex This proposal was extended by \citet{foxCyclicLinearizationSyntactic2005} and \citet{engelsMicrovariationObjectPositions2011,engelsScandinavianNegativeIndefinites2012} to account for quantifier movement (QM), which NegShift is a subset of. 
\ex Under this proposal QM is subject to an ``Anti-Holmberg Effect'' or an ``Inverse Holmberg Effect'', which are identical in principle
	\ea Under Holmberg's Generalization, OS can only apply if the verb has undergone movement from V to T to C.
	\ex The Anti-Holmberg Effect explains that only when the verb remains in situ can we have QM, which is the result of the ordering operations between the different phases being in agreement. 
	\z
	\vspace{6pt} 

\ex In order to account for OS, \citeauthor{foxCyclicLinearizationSyntactic2005} propose that the during the spell-out of the VP phase the V is the leftmost element in its phase\footnote{The position of the V at the left-edge of the phase could be due to the movement of V to \textit{v} in which case it is actually the \textit{v}P that acts as the phase not the VP.} and at which point the ordering restrictions are in place which state that the V must precede the O. 

\ex At this point the V moves to T and then to C at this point the object is free to move to a position higher because the order that existed at the VP phase continues to hold. 
\ex OS and string-vacuous Neg-Shift
\vspace{6pt}
 	\ea {}[\textsubscript{CP} S \rnode{b1}V … [\textsubscript{NegP} \rnode{A1}O adv [\textsubscript{VP} \rnode{b2}t\textsubscript{v} \rnode{A2}t\textsubscript{o} ]]]
	\psset{linearc=2pt} 
	\ncbar[angle=90]{->}{A2}{A1}  
	\ncbar[angle=-90,linestyle=dashed]{->}{b2}{b1} 
	\vspace{6pt}
	\ex VP Ordering: \textbf{V>O}\\
		CP Ordering: S>V, \textbf{V>O}, O>adv, adv>VP
	\z

\ex If the DO were to move instead of the IO this would now result in the DO being ordered before the IO at the spell-out at the CP phase. By moving the DO, we now introduce a mismatch between the ordering restrictions at the VP phase and the CP phase explaining why such utterances are ungrammatical. 

\ex In the case of NegShift, where it is able to shift across various phonological material it is proposed that the NI first moves to the left edge of the VP before spell-out of that phase. 
\vspace{6pt}
	\ea {}[\textsubscript{CP} S aux … [\textsubscript{NegP} \rnode{A1}O [\textsubscript{VP} \rnode{A2}t\textsubscript{o}  V \rnode{A3}t\textsubscript{o} ]]]
	\psset{linearc=2pt} 
	\ncbar[angle=90]{->}{A3}{A2}  
	\ncbar[angle=90]{->}{A2}{A1}
	\ex VP Ordering: \textbf{O>V}\\
	CP Ordering: S>V, aux>O, O>adv, adv>VP → \textbf{O>V}
	\z

\ex The benefit of using Cyclic Linearization comes from being able to account for why certain orders are fixed throughout the entire derivation.
\z 


\end{document}