% !TEX TS-program = xelatex
% !TEX encoding = UTF-8 Unicode		

\documentclass[12pt, letterpaper]{article}

%%BIBLIOGRAPHY- This uses biber/biblatex to generate bibliographies according to the 
%%Unified Style Sheet for Linguistics
\usepackage[main=american, german]{babel}% Recommended
\usepackage{csquotes}% Recommended
\usepackage[backend=biber,
		style=unified,
		maxcitenames=3,
		maxbibnames=99,
		natbib,
		url=false]{biblatex}
% \addbibresource{link_laptop.bib}
\addbibresource{link_desktop.bib}	%	Use link_laptop.bib when on Laptop and link_desktop.bib when on desktop

\setcounter{biburlnumpenalty}{100}  % allow URL breaks at numbers
%\setcounter{biburlucpenalty}{100}   % allow URL breaks at uppercase letters
%\setcounter{biburllcpenalty}{100}   % allow URL breaks at lowercase letters

%%TYPOLOGY
\usepackage[svgnames]{xcolor} % Specify colors by their 'svgnames', for a full list of all colors available see here: http://www.latextemplates.com/svgnames-colors
%\usepackage[compact]{titlesec}
%\titleformat{\section}[runin]{\normalfont\bfseries}{\thesection.}{.5em}{}[.]
%\titleformat{\subsection}[runin]{\normalfont\scshape}{\thesubsection}{.5em}{}[.]
\usepackage[hmargin=1in,vmargin=1in]{geometry}  %Margins          
\usepackage{graphicx}	%Inserting graphics, pictures, images 		
\usepackage{stackengine} %Package to allow text above or below other text, Also helpful for HG weights 
\usepackage{fontspec} %Selection of fonts must be ran in XeLaTeX
\usepackage{amssymb} %Math symbols
\usepackage{amsmath} % Mathematical enhancements for LaTeX
\usepackage{setspace} %Linespacing
\usepackage{multicol} %Multicolumn text
\usepackage{enumitem} %Allows for continuous numbering of lists over examples, etc.
\usepackage{multirow} %Useful for combining cells in tablesbrew 
\usepackage{hanging}
\usepackage{fancyhdr} %Allows for the 
\pagestyle{fancy}
\fancyhead[L]{\textit{290 Presentation}} 
\fancyhead[R]{\textit{\today}} 
\fancyfoot[L,R]{} 
\fancyfoot[C]{\thepage} 
\renewcommand{\headrulewidth}{1pt}
\setlength{\headheight}{14.5pt} % ...at least 14.49998pt
% \usepackage{fourier} % This allows for the use of certain wingdings like bombs, frowns, etc.
% \usepackage{fourier-orns} %More useful symbols like bombs and jolly-roger, mostly for OT
\usepackage[colorlinks,allcolors={black},urlcolor={blue}]{hyperref} %allows for hyperlinks and pdf bookmarks
% \usepackage{url} %allows for urls
% \def\UrlBreaks{\do\/\do-} %allows for urls to be broken up
\usepackage[normalem]{ulem} %strike out text. Handy for syntax
\usepackage{tcolorbox}

%%FONTS
\setmainfont{Libertinus Serif}
\setsansfont{Libertinus Sans}
\setmonofont[Scale=MatchLowercase]{Libertinus Mono}

%%PACKAGES FOR LINGUISTICS
%\usepackage{OTtablx} %Generating tableaux with using TIPA
\usepackage[noipa]{OTtablx} % Use this one generating tableaux without using TIPA
%\usepackage[notipa]{ot-tableau} % Another tableau drawing packing use for posters.
% \usepackage{linguex} % Linguistic examples
\usepackage{langsci-gb4e} % Language Science Press' modification of gb4e
% \usepackage{langsci-avm} % Language Science Press' AVM package
\usepackage{tikz} % Drawing Hasse diagrams
%\usepackage{pst-asr} % Drawing autosegmental features
\usepackage{pstricks} % required for pst-asr, OTtablx, pst-jtree.
% \usepackage{pst-jtree} 	% Syntax tree draawing software
\usepackage{tikz-qtree}	% Another syntax tree drawing software. Uses bracket notation.
\usepackage[linguistics]{forest}	% Another syntax tree drawing software. Uses bracket notation.
%\usepackage{ling-macros} % Various linguistic macros. Does not work with linguex.
%\usepackage{covington} % Another linguistic examples package.
\usepackage{leipzig}

%%TITLE INFORMATION
\title{Motivations for Scandinavian Negative Indefinite Shift}
\author{Mykel Loren Brinkerhoff}
\date{\today}

\newcommand{\sub}[1]{\textsubscript{#1}}
\newcommand{\supr}[1]{\textsuperscript{#1}}

\begin{document}
	
%%If using linguex, need the following commands to get correct LSA style spacing
%% these have to be after  \begin{document}
	% \setlength{\Extopsep}{6pt}
	% \setlength{\Exlabelsep}{9pt}		%effect of 0.4in indent from left text edge
%%
	
%% Line spacing setting. Comment out the line spacing you do not need. Comment out all if you want single spacing.
%	\doublespacing
%	\onehalfspacing
	
\begin{center}
	{\Large \textbf{Motivations for Scandinavian Negative Indefinite Shift}}\\
	\vspace{6pt}
	Mykel Loren Brinkerhoff\\
\end{center}
%\maketitle
%\maketitleinst
\thispagestyle{fancy}

% \tableofcontents

% %------------------------------------
% \section*{Outline of Handout} \label{sec:OUTLINE}
% %------------------------------------
% \begin{itemize}
% 	\item Overview of Negative Indefinite Shift in §\ref{sec:NEGSHIFT}
% 	\item Discussion of Scandinavian Pronoun structure in §\ref{sec:PRONOUNS}
% 	% \item Discussion of \cite{broekhuisUnificationObjectShift2020} in §\ref{sec:BROEKHUIS}
% 	\item Discussion of \cite{valentinebordalNegationExistentialPredications2017} in §\ref{sec:VB}
% 	\item Discussion of \cite{zeijlstraSyntacticallyComplexStatus2011} in §\ref{sec:ZEIJLSTRA}
% 	\item Next steps in §\ref{sec:NEXT}
% \end{itemize}

% \begin{tcolorbox}[width=\linewidth]
% \textsc{big questions}:
% \begin{itemize}
% \item What are the ways that movement can happen?
% \item What role does prosody play in movement?
% \item Does prosody play a role in Negative Indefinite Shifting in Scandinavian languages?
% \end{itemize}
% \end{tcolorbox}
\begin{tcolorbox}[width=\linewidth]
\textsc{big questions}:
\begin{itemize}
\item What are the ways in which movement happens?
\item What role does prosody play in movement?
\item Does prosody play a role in Negative Indefinite Shifting in Scandinavian languages?
\end{itemize}
\end{tcolorbox}
%------------------------------------
\section{Introduction} \label{sec:NEGSHIFT}
%------------------------------------
\ea Negative Shifting (NegShift) is a process in the Scandinavian languages where a negative indefinite (NI) obligatorily shifts to a position outside of the VP.
	\ea
	\gll Manden havde måske \textit{ingenting} [\textsubscript{VP} sagt t\textsubscript{o} ].\\
	man-the had probably nothing ~ said\\
	\glt `The man hadn't said anything.'
	\ex 
	\gll Jeg har \textit{ingen} \textit{bøger} [\textsubscript{VP} lånt børnene t\textsubscript{o}.]\\
	I have no books ~ lent children-the\\
	\glt `I haven't lent the children any books.'
	\z

\ex This process occurs to all NIs and is permissible from a large number of different contexts, depending on the variety and register (see Table \ref{tab:Distribution}). 

\ex This process bears some resemblance to Scandinavian Object Shift (OS), which is where a weak object pronoun shifts to a position outside of the verb phrase \citep{holmbergWordOrderSyntactic1986,holmbergRemarksHolmbergGeneralization1999}.
	\ea  {
	\gll Jag kyssade\textsubscript{v} henne\textsubscript{o} inte [\textsubscript{VP} t\textsubscript{v} t\textsubscript{o} ] \\
	I kiss.\Pst{} her \Neg{}\\} \jambox*{Sw}
	\glt `I didn't kiss her.'
	\z 

\ex There has been some evidence that OS is prosodically driven \citep{erteschik-shirSoundPatternsSyntax2005,erteschik-shirScandinavianObjectShift2017,erteschik-shirVariationMainlandScandinavian2020,brinkerhoffMATCHINGPhrasesNorwegian2021}. 

\ex Despite similarities between NegShift and OS, NegShift doesn't directly correlate to the accounts of OS. 

\ex Chiefly, a wider range of material is allowed to undergo NegShift, which includes both pronouns and full DPs, whereas only prosodically weak object pronouns are allowed to undergo OS.

\ex However, not all NegShift is treated equal. \citet[65f]{christensenInterfacesNegationSyntax2005}, speaking on Danish, claims that the ``weight" of the NI plays a factor in whether or not NegShift occurs. 
	\ea
	\gll Jeg har [\textit{intet} \textit{nyt}]\textsubscript{o} hørt t\textsubscript{o}.\\
	I have nothing new heard\\
	\glt `I haven't heard anything new.'
	\ex[*] {
	\gll Jeg har [\textit{intet} \textit{nyt} \textit{i} \textit{sagen}]\textsubscript{o} hørt t\textsubscript{o}.\\
	I have nothing new in case-\textsc{det} heard\\
	\glt `I haven't heard anything new about the case.'
	}
	\z

\begin{tcolorbox}[width=\linewidth]
\centering
I claim that NegShift is mostly a syntactic operation with prosody playing a supporting role in determining what material is allowed to surface at PF. 
\end{tcolorbox}	
\z 

%------------------------------------
\section{Distributional properties of NegShift versus OS} \label{sec:ENGELS}
%------------------------------------

\ea As mentioned above there are some similarities between OS and NegShift with them both operating on pronouns and moving them to a position outside of the VP. 
	\ea \label{ex:OS}
		\gll Jag kyssade\textsubscript{v} henne\textsubscript{o} inte [\textsubscript{VP} t\textsubscript{v} t\textsubscript{o} ] \\
		I kiss.\Pst{} her \Neg{}\\
		\glt `I didn't kiss her.'
	\ex \label{ex:NS}
		\gll Jag har ingen\textsubscript{o} [\textsubscript{VP} kyssat t\textsubscript{o} ]\\
		I have no-one ~ kiss.\Pst{}.\Ptcp{} \\
		\glt `I haven't kissed anyone.'
	\z 
	
\ex However, NegShift also operates on full negative DPs and is not subject to Holmberg's Generalization \citep{foxCyclicLinearizationSyntactic2005,engelsScandinavianNegativeIndefinites2012}
		\ea Verb in-situ NegShift\\
		{\gll Ég hef \textit{engan} [\textsubscript{VP} \textbf{séð} t\textsubscript{o}	].\\
			I have nobody ~ seen\\}\jambox*{Ic}
		\glt `I haven't seen anybody.'
		% \ex String vacuous NegShift\\
		% {\gll Jag \textbf{sa} \textit{ingenting} [\textsubscript{VP} t\textsubscript{v}  t\textsubscript{o}  ].\\
		% 	I said nothing\\}\jambox*{Sw}
		% \glt `I said nothing'
		\z
\ex Additional evidence from \citet{engelsScandinavianNegativeIndefinites2012} shows that NegShift is allowed out of a wider range of positions than OS, which are summarized in Table \ref{tab:Distribution}.

\begin{table}[!ht]
	\centering
	\caption{Distribution of NegShift across the different Scandinavian languages. WJ = West Jutlandic, Ic = Icelandic, Fa = Faroese, DaL = Danish Linguists, SwL = Swedish Linguists, Scan1 = literary/formal Mainland Scandinavian varieties, Scan2 = colloquial Mainland Scandinavian varieties and Norwegian}
	\label{tab:Distribution}
\begin{tabular}{llccccccccc}
	\hline 
	NegShift across &  & WJ1 & WJ2 & Ic & Fa & DaL1 & DaL2 & SwL & Scan1 & Scan2 \\ 
	\hline 
	String-vacuous &  & ✔︎ & ✔︎ & ✔︎ & ✔︎ & ✔︎ & ✔︎ & ✔︎ & ✔︎ & ✔︎ \\ 
	Verb &  & ✔︎ & ✔︎ & ✔︎ & ✔︎ & ✔︎ & ✔︎ & ✔︎ & ✔︎ & * \\ 
	IO & verb in situ & ✔︎ & ✔︎ & ✔︎ & ✔︎ & ✔︎ & ✔︎ & ✔︎ & ✔︎ & * \\ 
	& verb moved & * & * & * & * & * & * & * & * & * \\ 
	Preposition & verb in situ & ✔︎ & ✔︎ & ✔︎ & ✔︎ & ? & ? & * & * & * \\ 
	& verb moved & ✔︎ & ✔︎ & ? & * & * & * & * & * & * \\ 
	Infinitive & verb in situ & ✔︎ & ✔︎ & ✔︎ & ✔︎ & ✔︎ & * & ? & * & * \\ 
	& verb moved & ✔︎ & * & * & ✔︎ & * & * & * & * & * \\ 
	\hline 
\end{tabular} 
\end{table}
\z


%------------------------------------
\section{Comparison of NegShift and OS} \label{sec:Comparison}
%------------------------------------

%------------------------------------
\subsection{Holmberg's Generalization} \label{sec:HG}
%------------------------------------
\ea OS only occurs if the verb has moved to its V2 position and must obey Holmberg's Generalization.
\ea {Holmberg's Generalization:\\
Object Shift cannot apply across a phonologically visible category asymmetrically c-commanding the object position except adjuncts} \jambox*{\citep[15]{holmbergRemarksHolmbergGeneralization1999}}
\z
\ex Examples of OS 
 	\ea {
	\gll Peter så\textsubscript{v} \textbf{ham\textsubscript{o}} ikke [\textsubscript{VP} t\textsubscript{v} \textbf{t\textsubscript{o}}]\\
    Peter see.\Pst{} him not \\}\jambox*{(Across negation)}
    \glt `Peter didn't see him.' \label{ex:classic}
    \ex {
    \gll Peter så\textsubscript{v} \textbf{ham\textsubscript{o}} ofte [\textsubscript{VP} t\textsubscript{v} \textbf{t\textsubscript{o}}]\\
    Peter see.\Pst{} him often \\}\jambox*{(Across adverbials)} 
    \glt `Peter often saw him.' \label{ex:oft}
	\z 

\ex As seen in Table \ref{tab:Distribution}, NegShift occurs if the verb remains in situ. 

\ex Examples of NegShfit with the verb in situ. 
	\ea {
		\gll Ég hef \textit{engan}\sub{o} [\sub{VP} \textbf{séð} t\textsubscript{o}.] \\
			I have nobody {} seen\\}\jambox*{Ic}
	\glt `I haven't seen anybody.'%\rcommentg{Ic}
	\ex {
	\gll {Í dag} hevur Petur \textit{einki}\sub{o} [\sub{VP} \textbf{sagt} t\textsubscript{o}].\\
	today has Peter nothing {} said\\}\jambox*{Fa}
	\glt `Peter hasn't said anything today.'%\rcommentg{Fa}
	\ex {
	\gll Manden havde måske \textit{ingenting}\sub{o} [\sub{VP} \textbf{sagt} t\textsubscript{o}].\\
	man-the had probably nothing {} said\\}\jambox*{Da}
	\glt `The man hadn't said anything'
	\z  

\ex Additionally, it is allowed to occur across any phonological material in the VP.
	\ea {
	\gll Jón hefur \textit{ekkert} [\sub{VP} \textbf{sagt} \textbf{Sveini} t\textsubscript{o}]. \\
	Jón has nothing {} said Sveinn\\}\jambox*{Ic}
	\glt `John hasn't told Sveinn anything'%\rcommentg{Ic}
	\ex {
	\gll {Í dag} hevur Petur \textit{einki} [\sub{VP} \textbf{givið} \textbf{Mariu} t\textsubscript{o}]. \\
		today has Peter nothing {} given Mary\\}\jambox*{Fa}
	\glt `Today, Peter hasn't given Mary anything.'%\rcommentg{Fa}
	\ex {
	\gll Jeg har \textit{ingen} \textit{bøger} [\sub{VP} \textbf{lånt} \textbf{børnene} t\textsubscript{o}].\\
	I have no books {} lent children-the\\}\jambox*{WJ/Scan1}
	\glt `I haven't lent the children any books.'
	\z 

\ex This behavior is called the \emph{Anti-Holmberg Effect} \citep{foxCyclicLinearizationSyntactic2005,engelsScandinavianNegativeIndefinites2012}
\z

\begin{tcolorbox}[width=\linewidth]
\textsc{Claim:} OS and NegShift are not beholden to the same generalizations. 
\end{tcolorbox}
%------------------------------------
\subsection{Landing site} \label{sec:Landing}
%------------------------------------
\ea When OS occurs it must occur to the left of all adverbials.
	\ea 
	\gll Jeg lånte \emph{hende\sub{IO}} faktisk\sub{Adv} aldri\sub{Adv} [\sub{VP} t\sub{IO} bøgerne].\\
	I lent her actually never {} {} books-the\\
	\glt `I actually never lent her the books' 
	\z 

\ex When NegShift occurs it is located in a position to the right of all adverbials.
	\ea 
	\gll Jeg har faktisk\sub{Adv} \emph{ingen} \emph{bøger}\sub{DO} [\sub{VP} lånt Othilia t\sub{DO}].\\
	I have actually no books {} lent Othilia \\
	\glt `I didn't actually lend Othilia any books.'
	\z 
\z

\begin{tcolorbox}[width=\linewidth]
\textsc{Claim: OS and NegShift move to different locations.}
\end{tcolorbox}
%------------------------------------
\subsection{Interaction of NegShift and OS} \label{sec:NEG-OS}
%------------------------------------
\ea When there are both weak object pronouns and NIs in the sentence they both shift to their respective landing sites. 
	\ea
	\label{ex:NegOS}
	\gll Jeg lånte \textit{hende} fraktisch \textit{ingen} \textit{bøger}.\\
	I lent her actually no books\\
	\glt `I didn't actually lend her any books.'
	\z 
\ex When the verb remains in situ OS is blocked from occurring but NegShift is still allowed to occur. 
	\ea 
	\gll Jeg har \emph{ingen} \emph{bøger} lånt \emph{hende}.\\
	I have no books lent her\\
	\glt `I haven't lent her any books
	\z
\z

\begin{tcolorbox}[width=\linewidth]
\textsc{Claim:} OS and NegShift are not the same.

\textsc{Big Claim:} OS and NegShift are not derived by the same mechanisms. 
\end{tcolorbox}
%------------------------------------
\section{Prosodic restrictions on NegShift} \label{sec:PROSODY}
%------------------------------------

\ea \citet[65f]{christensenInterfacesNegationSyntax2005} claims that the ``weight" of the NI plays a crucial factor in whether or not NegShift occurs. 
	\ea 
	\gll Jeg har \textit{intet}\textsubscript{o} hørt t\textsubscript{o}.\\
	I have nothing heard\\
	\glt  `I havn't heard anything.'
	\ex 
	\gll Jeg har [\textit{intet} \textit{nyt}]\textsubscript{o} hørt t\textsubscript{o}.\\
	I have nothing new heard\\
	\glt `I haven't heard anything new'
	\ex[*] {
	\gll Jeg har [\textit{intet} \textit{nyt} \textit{i} \textit{sagen}]\textsubscript{o} hørt t\textsubscript{o}.\\
	I have nothing new in case-\textsc{det} heard\\
	\glt `I haven't heard anything new about the case.'
	}
	\ex[*] {
	\gll Jeg har [\textit{intet} \textit{nyt} \textit{i} \textit{sagen} \textit{om} \textit{de} \textit{stjålne} \textit{malerier}]\textsubscript{o} hørt t\textsubscript{o}.\\
	I have nothing new in case-\textsc{det} about the stolen paintings heard\\
	}
	\glt `I haven't heard anything new in the case about the stolen paintings.'
	\z

\ex In those instances where the NI is too large one potential repair is to strand the PP while moving just the pronoun or using the negative particle \textit{ikke} and a NPI.
	\ea Jeg har \textit{intet}\textsubscript{i} hørt t\textsubscript{i} [\textsubscript{PP} i sagen om de stjålne malerier ].
	\ex Jeg har \textit{ikke} hørt [ \textit{noget} i sagen om de stjålne malerier ].
	\z 
 
\ex This same behavior has also been remarked upon by \citet{penkaNegativeIndefinites2011} for Swedish.
	\ea 
	\gll Men mänskligheten har \textit{ingenting}\textsubscript{o} lärt sig t\textsubscript{o}.\\
	but mankind-the have nothing taught themselves\\
	\glt `But mankind haven't taught themselves anything.'
	\ex[?] {
	\gll Vi hade \textit{inga} \textit{grottor}\textsubscript{o} undersökt t\textsubscript{o}.\\
	we have no caves explored\\
	}
	\glt `We haven't explored any caves.'
	\z 
\z 
\begin{tcolorbox}[width=\linewidth]
\textsc{Claim:} Prosody plays a supporting role in restricting and regulating the amount of material that undergoes NegShift.
\end{tcolorbox}
%------------------------------------
\section{Copy and partial deletion account} \label{sec:ZEIJLSTRA}
%------------------------------------

\ea One way to account for this behavior is following \cite{zeijlstraSyntacticallyComplexStatus2011} account for the split-scope that these NIs introduce in Germanic languages.

\ex Split-scope is evident when modals and other auxiliaries are present and the negation scopes higher than the modal/auxiliary's scope.

\ex Split-scope is the result of a copy-theory of movement where the indefinite interpretation is in the lower copy while the negative operator is interpreted in the higher copy.

\ex Following \citet{fanselowRemarksEconomyPronunciation2001,fanselowDistributedDeletion2002} parts of the copy are removed in PF or LF.

\ex For example, we see this in the following syntactic structure.\\
	\resizebox*{\linewidth}{!}{\begin{tikzpicture} 
	\tikzset{every tree node/.style={align=center,anchor=north}} 
	\Tree [.CP 	DP\\\emph{Anna} 
				[.C´ C\\\emph{har} 
				[.TP <DP> 
					[.T´ T 
						[.NegP  
							[.DP \edge[roof]; {intet i sagen om de stjålne malerier}
							]
							[.vP <DP>
								[.v´ hørt   
									[.VP <V> 
										[.DP \edge[roof]; {intet i sagen om de stjålne malerier}
										]
									]
								]
							]    
						]
					] 
				]
				] 
			] 
	\end{tikzpicture}}
	

\ex At this point during PF, part of the higher copy is deleted leaving only the NI pronoun. In the lower copy \emph{intet} is deleted.\\
\label{ex:tree}
	\resizebox*{\linewidth}{!}{
	\begin{tikzpicture} 
	\tikzset{every tree node/.style={align=center,anchor=north}} 
	\Tree [.CP 	DP\\\emph{Anna} 
				[.C´ C\\\emph{har} 
				[.TP <DP> 
					[.T´ T 
						[.NegP  
							[.DP \edge[roof]; {intet \sout{i sagen om de stjålne malerier}}
							]
							[.vP <DP>
								[.v´ hørt   
									[.VP <V> 
										[.DP \edge[roof]; {\sout{intet} i sagen om de stjålne malerier}
										]
									]
								]
							]    
						]
					] 
				]
				] 
			] 
	\end{tikzpicture}}

\z 

\begin{tcolorbox}[width=\linewidth]
\textsc{Question:} What decides which material is deleted?

\end{tcolorbox}

%------------------------------------
\section{Prosodic motivations for Heavy NP Shift} \label{sec:HNPS}
%------------------------------------

\ea According to \citet{anttilaRoleProsodyEnglish2010}, prosody plays a supporting role in linearization and can be accounted for through OT style constraints on prosodic well-formedness. 
\ex They are concerned with explaining the three way patterning that dative constructions have:
	\ea Double object constructions
	\ex Prepositional constructions
	\ex Heavy NP Shift constructions
	\z

\ex Of crucial interest is how they explain Heavy NP Shift and what their criteria for determining weight.

\ex \citet[949]{anttilaRoleProsodyEnglish2010} say that the ``''weight" of an NP is ``a function of the number of lexically stressed words in [the constituent]".

\ex This means that the more lexical stresses a constituent has the heavier it is. This definition, crucially, leaves out functional items and pronouns because they lack lexical stress. 

\ex Using this definition for weight, we can explain the behavior Danish and Swedish as discussed in §\ref{sec:NEGSHIFT}. 

\z 

%------------------------------------
\section{OT Account for Heavy NP Shift} \label{sec:OT}
%------------------------------------

\ea Based on \citet{anttilaRoleProsodyEnglish2010}, prosody plays a supporting role in linearization and can be accounted for through OT style constraints on prosodic well-formedness. 

\ex This can be accounted for using Match Theory \citep{selkirkClauseIntonationalPhrase2009,selkirkSyntaxPhonologyInterface2011}.

\ex Following \citet{myrbergSisterhoodProsodicBranching2013,myrbergProsodicWordSwedish2013,myrbergProsodicHierarchySwedish2015}, I assume that the subject in Scandinavian languages forms its own phonological phrase, if it is not a pronoun, separate from the rest of the clause. 
	\ea Simplified prosodic structure for (\ref{ex:tree})\\
	% \resizebox*{\linewidth}{!}{
	\begin{tikzpicture} 
	\tikzset{every tree node/.style={align=center,anchor=north}} 
	\Tree [.ι [.φ\\Anna ] [.φ \edge[roof]; {har intet hørt i sagen om de stjålne malerier} ] ] 
	\end{tikzpicture}%}

	\z

\ex Crucially, what we are concerned with the weight of the item shifting.
	\ea This can be accounted for using a type of \textsc{NoShift} \citep{bennettLightestRightApparently2016} which is sensitive to lexical stresses.
	\z 

\ex \textsc{NoShift(Stress)}:\\
Assign one violation for every word bearing lexical stress that is not in the same linear order as in the input.

\ex If we take the input of (\ref{ex:tree}) this constraint should assign a violation for every item bearing lexical stress that has been relinearized. 

\ex Using OT we can model how this would behave with the input of (\ref{ex:tree})
	\resizebox*{\linewidth}{!}{
	\begin{OTtableau}{3}
		\OTdashes{1,2}
		\OTtoprow [ ] {\textsc{Match(XP,φ)}, \textsc{Match(φ,XP)}, \textsc{NoShift(Stress)}}
		\OTcandrow [\OThand] {( Anna\sub{ω} )\sub{φ} (har \textit{intet}\sub{ω} hørt i sagen om de stjålne malerier)\sub{φ}} { 1 , 1 , 1} 
		\OTcandrow [\OThand] {( Anna\sub{ω} )\sub{φ} (har \textit{intet nyt}\sub{ω} hørt i sagen om de stjålne malerier)\sub{φ}} { 1 , 1 , 1} 
		\OTcandrow [\OTface*] {( Anna\sub{ω} )\sub{φ} (har \textit{intet nyt\sub{ω} i sagen}\sub{ω} hørt om de stjålne malerier)\sub{φ}} { 1 , 1 , 2!} 
		\OTcandrow [\OTface*] {( Anna\sub{ω} )\sub{φ} (har \textit{intet i sagen\sub{ω} om de stjålne\sub{ω} malerier\sub{ω}} hørt )\sub{φ}} { 1 , 1 , 3! } 
	\end{OTtableau}
	}
\z 

%------------------------------------
\subsection{Prosodic structure} \label{sec:OT}
%------------------------------------
\ea One alternative is that the prosodic structure is deciding the amount of material that is deleted.  

\ex We can assume that the following prosodic structure would correspond to the tree in (\ref{ex:tree})
	\ea
	% \resizebox*{\linewidth}{!}{
	\begin{forest}
	[ι
		[φ\\\textit{Anna}]
		[φ [ω\\\textit{har}]
			[φ
				[φ  [ω\\\emph{intet}]
					[φ [ω\\\emph{i=sagen}]
						[φ [ω\\\emph{om=de}]
							[ω [ω\\\emph{stålne}]
								[ω\\\emph{malerier}]
							]
						]	
					]
				]
				[φ [ω\\\textit{hørt}]
					[φ  [ω\\\emph{intet}]
						[φ [ω\\\emph{i=sagen}]
							[φ [ω\\\emph{om=de}]
								[ω [ω\\\emph{stålne}]
									[ω\\\emph{malerier}]
								]
							]	
						]
					]
				]
			]
		]
	]
	\end{forest}%} 
	\z 

\ex At this point in the derivation the we want our higher copy to consist of at most a prosodic word so all material that is larger than a prosodic word is deleted.

\ex Another constraint come along and deletes the material that remains in the higher copy from the lower resulting in our final prosodic structure
	% \resizebox*{\linewidth}{!}{
	\begin{forest}
	[ι
		[φ\\\textit{Anna}]
		[φ [ω\\\textit{har}]
			[φ [ω\\\emph{intet}]
				[φ [ω\\\textit{hørt}]
					[φ [ω\\\emph{i=sagen}]
						[φ [ω\\\emph{om=de}]
							[ω [ω\\\emph{stålne}]
								[ω\\\emph{malerier}]
							]
						]	
					]
				]
			]
		]
	]
	\end{forest}%} 
\z 

%------------------------------------
%BIBLIOGRAPHY
%------------------------------------

%\singlespacing
% \nocite{*}
% \printbibliography[heading=bibintoc, keyword={QP1}]
\printbibliography[heading=bibintoc]

\part*{Appendix}

%------------------------------------
\section*{Cyclic Linearization} \label{sec:CL}
%------------------------------------

\ea Cyclic Linearization is a theory that was developed by \cite{foxCyclicLinearizationSyntactic2005} as a way to account for OS and Holmberg's Generalization.
\ex This theory works by stipulating that spell-out is cyclic and order preserving. 

\ex This proposal was extended by \citet{foxCyclicLinearizationSyntactic2005} and \citet{engelsMicrovariationObjectPositions2011,engelsScandinavianNegativeIndefinites2012} to account for quantifier movement (QM), which NegShift is a subset of. 
\ex Under this proposal QM is subject to an ``Anti-Holmberg Effect'' or an ``Inverse Holmberg Effect'', which are identical in principle
	\ea Under Holmberg's Generalization, OS can only apply if the verb has undergone movement from V to T to C.
	\ex The Anti-Holmberg Effect explains that only when the verb remains in situ can we have QM, which is the result of the ordering operations between the different phases being in agreement. 
	\z
	\vspace{6pt} 

\ex In order to account for OS, \citeauthor{foxCyclicLinearizationSyntactic2005} propose that the during the spell-out of the VP phase the V is the leftmost element in its phase\footnote{The position of the V at the left-edge of the phase could be due to the movement of V to \textit{v} in which case it is actually the \textit{v}P that acts as the phase not the VP.} and at which point the ordering restrictions are in place which state that the V must precede the O. 

\ex At this point the V moves to T and then to C at this point the object is free to move to a position higher because the order that existed at the VP phase continues to hold. 
\ex OS and string-vacuous Neg-Shift
\vspace{6pt}
 	\ea {}[\textsubscript{CP} S \rnode{b1}V … [\textsubscript{NegP} \rnode{A1}O adv [\textsubscript{VP} \rnode{b2}t\textsubscript{v} \rnode{A2}t\textsubscript{o} ]]]
	\psset{linearc=2pt} 
	\ncbar[angle=90]{->}{A2}{A1}  
	\ncbar[angle=-90,linestyle=dashed]{->}{b2}{b1} 
	\vspace{6pt}
	\ex VP Ordering: \textbf{V>O}\\
		CP Ordering: S>V, \textbf{V>O}, O>adv, adv>VP
	\z

\ex If the DO were to move instead of the IO this would now result in the DO being ordered before the IO at the spell-out at the CP phase. 

\ex In the case of NegShift, where it is able to shift across various phonological material it is proposed that the NI first moves to the left edge of the VP before spell-out of that phase. 
\vspace{6pt}
	\ea {}[\textsubscript{CP} S aux … [\textsubscript{NegP} \rnode{A1}O [\textsubscript{VP} \rnode{A2}t\textsubscript{o}  V \rnode{A3}t\textsubscript{o} ]]]
	\psset{linearc=2pt} 
	\ncbar[angle=90]{->}{A3}{A2}  
	\ncbar[angle=90]{->}{A2}{A1}
	\ex VP Ordering: \textbf{O>V}\\
	CP Ordering: S>V, aux>O, O>adv, adv>VP → \textbf{O>V}
	\z

\ex The benefit of using Cyclic Linearization comes from being able to account for why certain orders are fixed throughout the entire derivation.
\z 

%------------------------------------
\section*{Svenonius 2002}
%------------------------------------

\ea	\citet{svenoniusStrainsNegationNorwegian2002} describes that \emph{in Norwegian} NIs only license sentential negation when it has moved to a position outside of the VP. 

\ex According to \citeauthor{svenoniusStrainsNegationNorwegian2002} there are five distinct cases involving NIs.
	\ea \emph{ingen} is a \emph{licensor} of and does not in and of itself express negation.
	\ex \textsc{Trifling Negation} \emph{ingen} can be narrowly interpreted to mean something like "zero" or "a trifle", when this is the case NegShift has not taken place. 
	\ex \textsc{P Negation} is not as tightly confined like in trifling negation, but is not interpreted at the sentential level, this involves predicates only. 
	\ex \textsc{Narrow Scope} is a cover term for trifling and P negation. 
	\ex \emph{ingen} can appear in double negation expressions.
	\z 

\ex Before looking at examples it is important to remember that Norwegian does not allow NegShift in colloquial Norwegian. 
	\ea When it does occur it is marked and is indicative of a literary or archaic style. 
	\z 

\ex I will only focus on sentential and trifling negation. 

\ex Examples of sentential negation in Norwegian, the mark (*†) shows that the interpretation is ungrammatical in the colloquial register but attested in literary or archaic styles.
	\ea 
	\gll Vi vant ingen konkurranse.\\
	We want no competition\\
	\trans `We did not win any competition'
	\ex[*†] { 
	\gll Vi kunne ingen konkurranse vinne.\\
	We could no competition won\\}
	\ex[*] {
	\gll Vi vant ingen konkurranse i.\\
	We won no competition in\\
	}
	\ex[*†] {
	\gll …at vi ingen konkurranse vant\\
	that we no competition won\\
	}
	\z  

\ex The sentential interpretation is only possible if:
	\ea The NI moves outside of VP, and
	\ex Is subject to Holmberg's Generalization
	\z 

% \ex I disagree with \citeauthor{svenoniusStrainsNegationNorwegian2002} about NegShift being subject to Holmberg's Generalization.
% 	\ea As discussed in \citet{foxCyclicLinearizationSyntactic2005,engelsScandinavianNegativeIndefinites2012} there is evidence that NegShift is actually subject to an Anti-Holmberg effect. 
% 	\ex Anti-Holmberg states that NegShift is only possible if the verb remains \emph{in-situ}. 
% 	\z 

\ex When it comes to the trifling negation, it is only the constituent that is negated. This type of reading is quite restricted in Norwegian and generally means something like "zero" when used. 
	\ea[*] {
	\gll De har gitt Norge ingen poeng, og det har heller ikke irene. \\
	they have given Norway no points and that have either not the.Irish\\
	} 
	\trans (intended: `They have given Norway no points, and neither have the Irish') \label{ex:neither}
	\ex 
	\gll De har gitt Norge ingen poeng, og det har også esterne.\\
	they have given Norway no points and that have also the.Estonians\\
	\trans ‘They have given Norway no points, and so have the Estonians \label{ex:ConstituentNegation}
	\z 
	
\ex We know that we get the constituent negation when we compare (\ref{ex:neither}), which uses \emph{neither}, with (\ref{ex:ConstituentNegation}).

\ex \citeauthor{svenoniusStrainsNegationNorwegian2002} claims that the only way that we get sentential negation is through NegShift.
	\ea It is unclear if the cases involving 
	\z 

\z

\end{document}