% !TEX TS-program = xelatex
% !TEX encoding = UTF-8 Unicode		

\documentclass[12pt, letterpaper]{article}

%%BIBLIOGRAPHY- This uses biber/biblatex to generate bibliographies according to the 
%%Unified Style Sheet for Linguistics
\usepackage[main=american, german]{babel}% Recommended
\usepackage{csquotes}% Recommended
\usepackage[backend=biber,
		style=unified,
		maxcitenames=3,
		maxbibnames=99,
		natbib,
		url=false]{biblatex}
% \addbibresource{link_laptop.bib}
\addbibresource{link_desktop.bib}	%	Use link_laptop.bib when on Laptop and link_desktop.bib when on desktop

\setcounter{biburlnumpenalty}{100}  % allow URL breaks at numbers
%\setcounter{biburlucpenalty}{100}   % allow URL breaks at uppercase letters
%\setcounter{biburllcpenalty}{100}   % allow URL breaks at lowercase letters

%%TYPOLOGY
\usepackage[svgnames]{xcolor} % Specify colors by their 'svgnames', for a full list of all colors available see here: http://www.latextemplates.com/svgnames-colors
%\usepackage[compact]{titlesec}
%\titleformat{\section}[runin]{\normalfont\bfseries}{\thesection.}{.5em}{}[.]
%\titleformat{\subsection}[runin]{\normalfont\scshape}{\thesubsection}{.5em}{}[.]
\usepackage[hmargin=1in,vmargin=1in]{geometry}  %Margins          
\usepackage{graphicx}	%Inserting graphics, pictures, images 		
\usepackage{stackengine} %Package to allow text above or below other text, Also helpful for HG weights 
\usepackage{fontspec} %Selection of fonts must be ran in XeLaTeX
\usepackage{datetime2}
\usepackage{amssymb} %Math symbols
\usepackage{amsmath} % Mathematical enhancements for LaTeX
\usepackage{setspace} %Linespacing
\usepackage{multicol} %Multicolumn text
\usepackage{enumitem} %Allows for continuous numbering of lists over examples, etc.
\usepackage{multirow} %Useful for combining cells in tablesbrew 
\usepackage{hanging}
\usepackage{fancyhdr} %Allows for the 
\pagestyle{fancy}
\fancyhead[L]{\textit{Graduate Research Symposium 2021}} 
\fancyhead[R]{\textit{Brinkerhoff}} 
\fancyfoot[L,R]{} 
\fancyfoot[C]{\thepage} 
\renewcommand{\headrulewidth}{1pt}
\setlength{\headheight}{14.5pt} % ...at least 14.49998pt
% \usepackage{fourier} % This allows for the use of certain wingdings like bombs, frowns, etc.
% \usepackage{fourier-orns} %More useful symbols like bombs and jolly-roger, mostly for OT
\usepackage[colorlinks,
			allcolors={black},
			urlcolor={blue}
			]{hyperref} %allows for hyperlinks and pdf bookmarks
% \usepackage{url} %allows for urls
% \def\UrlBreaks{\do\/\do-} %allows for urls to be broken up
\usepackage[normalem]{ulem} %strike out text. Handy for syntax
\usepackage{tcolorbox}

%%FONTS
\setmainfont{Libertinus Serif}
\setsansfont{Libertinus Sans}
\setmonofont[Scale=MatchLowercase]{Libertinus Mono}

%%PACKAGES FOR LINGUISTICS
%\usepackage{OTtablx} %Generating tableaux with using TIPA
\usepackage[noipa]{OTtablx} % Use this one generating tableaux without using TIPA
%\usepackage[notipa]{ot-tableau} % Another tableau drawing packing use for posters.
% \usepackage{linguex} % Linguistic examples
\usepackage{langsci-gb4e} % Language Science Press' modification of gb4e
% \usepackage{langsci-avm} % Language Science Press' AVM package
\usepackage{tikz} % Drawing Hasse diagrams
%\usepackage{pst-asr} % Drawing autosegmental features
\usepackage{pstricks} % required for pst-asr, OTtablx, pst-jtree.
% \usepackage{pst-jtree} 	% Syntax tree draawing software
\usepackage{tikz-qtree}	% Another syntax tree drawing software. Uses bracket notation.
\usepackage[linguistics]{forest}	% Another syntax tree drawing software. Uses bracket notation.
%\usepackage{ling-macros} % Various linguistic macros. Does not work with linguex.
%\usepackage{covington} % Another linguistic examples package.
\usepackage{leipzig}

%%FOREST CONFIGURATION
\forestset{
fairly nice empty nodes/.style={ 
delay={where content={}
{shape=coordinate, for siblings={anchor=north}}{}},
for tree={s sep=4mm}
}
}


%%TITLE INFORMATION
\title{Motivations for Scandinavian Negative Indefinite Shift}
\author{Mykel Loren Brinkerhoff}
\date{\today}

\newcommand{\sub}[1]{\textsubscript{#1}}
\newcommand{\supr}[1]{\textsuperscript{#1}}

\begin{document}

%%If using linguex, need the following commands to get correct LSA style spacing
%% these have to be after  \begin{document}
	% \setlength{\Extopsep}{6pt}
	% \setlength{\Exlabelsep}{9pt}		%effect of 0.4in indent from left text edge
%%
	
%% Line spacing setting. Comment out the line spacing you do not need. Comment out all if you want single spacing.
%	\doublespacing
	\onehalfspacing
	
\begin{center}
	{\Large \textbf{Motivations for Scandinavian Negative Indefinite Shift}}\\
	\vspace{6pt}
	Mykel Loren Brinkerhoff\\
\end{center}
%\maketitle
%\maketitleinst
\thispagestyle{fancy}

% \tableofcontents

% %------------------------------------
% \section*{Outline of Handout} \label{sec:OUTLINE}
% %------------------------------------
% \begin{itemize}
% 	\item Overview of Negative Indefinite Shift in §\ref{sec:NEGSHIFT}
% 	\item Discussion of Scandinavian Pronoun structure in §\ref{sec:PRONOUNS}
% 	% \item Discussion of \cite{broekhuisUnificationObjectShift2020} in §\ref{sec:BROEKHUIS}
% 	\item Discussion of \cite{valentinebordalNegationExistentialPredications2017} in §\ref{sec:VB}
% 	\item Discussion of \cite{zeijlstraSyntacticallyComplexStatus2011} in §\ref{sec:ZEIJLSTRA}
% 	\item Next steps in §\ref{sec:NEXT}
% \end{itemize}

% \begin{tcolorbox}[width=\linewidth]
% \textsc{big questions}:
% \begin{itemize}
% \item What are the ways that movement can happen?
% \item What role does prosody play in movement?
% \item Does prosody play a role in Negative Indefinite Shifting in Scandinavian languages?
% \end{itemize}
% \end{tcolorbox}
\begin{tcolorbox}[width=\linewidth]
\textsc{Big Questions}:
\begin{itemize}
	\item What are the triggers for movement?
	\item In what ways can prosody interact with syntactic movement?
	\item Does prosody play a role in Negative Indefinite Shifting in Scandinavian languages?
\end{itemize}
\textsc{Answers:}
\begin{itemize}
	\item NegShift occurs as a result of feature valuation. 
	\item Prosody controls and moderates the heaviness of moved copies. 
\end{itemize}
\end{tcolorbox}
%------------------------------------
\section{Introduction} \label{sec:NEGSHIFT}
%------------------------------------
\ea Negative Shifting (NegShift) is a process in the Scandinavian languages where a negative indefinite (NI) obligatorily shifts to a position outside of the VP.
	\ea
	\gll Manden havde måske \textit{ingenting} [\textsubscript{VP} sagt t\textsubscript{o} ].\\
	man-the had probably nothing ~ said\\
	\glt `The man hadn't said anything.'
	\ex 
	\gll Jeg har \textit{ingen} \textit{bøger} [\textsubscript{VP} lånt børnene t\textsubscript{o}.]\\
	I have no books ~ lent children-the\\
	\glt `I haven't lent the children any books.'
	\z

\ex This process occurs to all NIs and is permissible from a large number of different contexts, depending on the variety and register (see Table \ref{tab:Distribution}). 

\ex This process bears some resemblance to Scandinavian Object Shift (OS), which is where a weak object pronoun shifts to a position outside of the verb phrase \citep{holmbergWordOrderSyntactic1986,holmbergRemarksHolmbergGeneralization1999}.
	\ea  {
	\gll Jag kyssade\textsubscript{v} henne\textsubscript{o} inte [\textsubscript{VP} t\textsubscript{v} t\textsubscript{o} ] \\
	I kiss.\Pst{} her \Neg{}\\} \jambox*{Sw}
	\glt `I didn't kiss her.'
	\z 

\ex There has been some evidence that OS is prosodically driven \citep{erteschik-shirSoundPatternsSyntax2005,erteschik-shirScandinavianObjectShift2017,erteschik-shirVariationMainlandScandinavian2020,brinkerhoffMATCHINGPhrasesNorwegian2021}. 

\ex Despite similarities between NegShift and OS, NegShift doesn't directly correlate to the accounts of OS. 

\ex Chiefly, a wider range of material is allowed to undergo NegShift, which includes both pronouns and full DPs, whereas only prosodically weak object pronouns are allowed to undergo OS.

\ex However, not all NegShift is treated equal. \citet[65f]{christensenInterfacesNegationSyntax2005}, speaking on Danish, claims that the ``weight" of the NI plays a factor in whether or not NegShift occurs. 
	\ea
	\gll Jeg har [\textit{intet} \textit{nyt}]\textsubscript{o} hørt t\textsubscript{o}.\\
	I have nothing new heard\\
	\glt `I haven't heard anything new.'
	\ex[*] {
	\gll Jeg har [\textit{intet} \textit{nyt} \textit{i} \textit{sagen}]\textsubscript{o} hørt t\textsubscript{o}.\\
	I have nothing new in case-\textsc{det} heard\\
	\glt `I haven't heard anything new about the case.'
	}
	\z
\z 

%------------------------------------
\section{Distributional properties of NegShift versus OS} \label{sec:ENGELS}
%------------------------------------

\ea OS and NegShift appear on the surface to be similar to one another. 
\ex Both OS and NegShift operate on pronouns and moving them to a position outside of the VP. 
	\ea \label{ex:OS}
		\gll Jag kyssade\textsubscript{v} henne\textsubscript{o} inte [\textsubscript{VP} t\textsubscript{v} t\textsubscript{o} ] \\
		I kiss.\Pst{} her \Neg{}\\
		\glt `I didn't kiss her.'
	\ex \label{ex:NS}
		\gll Jag har ingen\textsubscript{o} [\textsubscript{VP} kyssat t\textsubscript{o} ]\\
		I have no-one ~ kiss.\Pst{}.\Ptcp{} \\
		\glt `I haven't kissed anyone.'
	\z 
	
\ex Unlike OS, NegShift also operates on full negative DPs.
		\ea {\gll Jens har \textit{ingen} \textit{hunder} [\textsubscript{VP} \textbf{sluppet} t\textsubscript{o} ud].\\
			Jens have no hounds ~ let ~ out\\}\jambox*{Da}
			\glt `Jens haven't let out any dogs.'
		\ex {\gll Jeg har \textit{ingen} \textit{bøger} [\textsubscript{VP} \textbf{lånt} børnene t\textsubscript{o} ].\\
			I have no books ~ lent children-the\\}\jambox*{WJ/Scan1}
			\glt `I haven't lent the children any books.'
		\z
\ex Additional evidence from \citet{engelsScandinavianNegativeIndefinites2012} shows that NegShift is allowed out of a wider range of contexts than OS, which are summarized in Table \ref{tab:Distribution}.
	\ea We see a strong preference for NegShifting across the verb and phonological material as crucially seen in the bolded and colored rows. 
	\z 
\begin{table}[!ht]
	\centering
	\caption{Distribution of NegShift across different Scandinavian languages. WJ = West Jutlandic, Ic = Icelandic, Fa = Faroese, DaL = Danish Linguists, SwL = Swedish Linguists, Scan1 = literary/formal Mainland Scandinavian, Scan2 = colloquial Mainland Scandinavian and Norwegian}
	\label{tab:Distribution}
\begin{tabular}{llccccccccc}
	\hline 
	NegShift across &  & WJ1 & WJ2 & Ic & Fa & DaL1 & DaL2 & SwL & Scan1 & Scan2 \\ 
	\hline 
	String-vacuous &  & ✔︎ & ✔︎ & ✔︎ & ✔︎ & ✔︎ & ✔︎ & ✔︎ & ✔︎ & ✔︎ \\ 
	\textcolor{DarkBlue}{\textbf{Verb}} &  & \textcolor{DarkBlue}{✔︎} & \textcolor{DarkBlue}{✔︎} & \textcolor{DarkBlue}{✔︎} & \textcolor{DarkBlue}{✔︎} & \textcolor{DarkBlue}{✔︎} & \textcolor{DarkBlue}{✔︎} & \textcolor{DarkBlue}{✔︎} & \textcolor{DarkBlue}{✔︎} & \textcolor{DarkBlue}{*} \\ 
	\textcolor{DarkBlue}{\textbf{IO}} & \textcolor{DarkBlue}{\textbf{verb in situ}} & \textcolor{DarkBlue}{✔︎} & \textcolor{DarkBlue}{✔︎} & \textcolor{DarkBlue}{✔︎} & \textcolor{DarkBlue}{✔︎} & \textcolor{DarkBlue}{✔︎} & \textcolor{DarkBlue}{✔︎} & \textcolor{DarkBlue}{✔︎} & \textcolor{DarkBlue}{✔︎} & \textcolor{DarkBlue}{*} \\ 
	& \textcolor{DarkBlue}{\textbf{verb moved}} & \textcolor{DarkBlue}{*} & \textcolor{DarkBlue}{*} & \textcolor{DarkBlue}{*} & \textcolor{DarkBlue}{*} & \textcolor{DarkBlue}{*} & \textcolor{DarkBlue}{*} & \textcolor{DarkBlue}{*} & \textcolor{DarkBlue}{*} & \textcolor{DarkBlue}{*} \\ 
	Preposition & verb in situ & ✔︎ & ✔︎ & ✔︎ & ✔︎ & ? & ? & * & * & * \\ 
	& verb moved & ✔︎ & ✔︎ & ? & * & * & * & * & * & * \\ 
	Infinitive & verb in situ & ✔︎ & ✔︎ & ✔︎ & ✔︎ & ✔︎ & * & ? & * & * \\ 
	& verb moved & ✔︎ & * & * & ✔︎ & * & * & * & * & * \\ 
	\hline 
\end{tabular} 
\end{table}
\z


%------------------------------------
\section{Comparison of NegShift and OS} \label{sec:Comparison}
%------------------------------------

\ea By comparing NegShift's patterns against OS's patterns, we can determine whether they are governed by the same or different factors. 

\ex There are three different metrics that we can use to determine the governing factors:
	\ea Obedience to Holmberg's Generalization.
	\ex Where the loci of movement are.
	\ex How NegShift and OS interact with one another.   
	\z 

\ex If OS and NegShift are derived by the same trigger then we expect them to behave the same in each of the three metrics.

\ex I show that this is not the case and claim that NegShift and OS are governed by different movement triggers. 
\z 

%------------------------------------
\subsection{Holmberg's Generalization} \label{sec:HG}
%------------------------------------
\ea OS only occurs if the verb has moved to V2 and must obey Holmberg's Generalization.

\ea {Holmberg's Generalization:\\
Object Shift cannot apply across a phonologically visible category asymmetrically c-commanding the object position except adjuncts} \jambox*{\citep[15]{holmbergRemarksHolmbergGeneralization1999}}
\z

\ex Examples of OS 
 	\ea {
	\gll Peter så\textsubscript{v} \textbf{ham\textsubscript{o}} ikke [\textsubscript{VP} t\textsubscript{v} \textbf{t\textsubscript{o}}]\\
    Peter see.\Pst{} him not \\}\jambox*{(Across negation)}
    \glt `Peter didn't see him.' \label{ex:classic}
    \ex {
    \gll Peter så\textsubscript{v} \textbf{ham\textsubscript{o}} ofte [\textsubscript{VP} t\textsubscript{v} \textbf{t\textsubscript{o}}]\\
    Peter see.\Pst{} him often \\}\jambox*{(Across adverbials)} 
    \glt `Peter often saw him.' \label{ex:oft}
	\z 

\ex NegShift does not obey Holmberg's Generalization. In fact the verb has to remain in situ for it to occur. 

\ex Examples of NegShfit with the verb in situ. 
	\ea {
		\gll Ég hef \textit{engan}\sub{o} [\sub{VP} \textbf{séð} t\textsubscript{o}.] \\
			I have nobody {} seen\\}\jambox*{Ic}
	\glt `I haven't seen anybody.'
	\ex {
	\gll {Í dag} hevur Petur \textit{einki}\sub{o} [\sub{VP} \textbf{sagt} t\textsubscript{o}].\\
	today has Peter nothing {} said\\}\jambox*{Fa}
	\glt `Peter hasn't said anything today.'
	\ex {
	\gll Manden havde måske \textit{ingenting}\sub{o} [\sub{VP} \textbf{sagt} t\textsubscript{o}].\\
	man-the had probably nothing {} said\\}\jambox*{Da}
	\glt `The man hadn't said anything'
	\z  

\ex Additionally, it occurs across any phonological material in the VP.
	\ea {
	\gll Jón hefur \textit{ekkert} [\sub{VP} \textbf{sagt} \textbf{Sveini} t\textsubscript{o}]. \\
	Jón has nothing {} said Sveinn\\}\jambox*{Ic}
	\glt `John hasn't told Sveinn anything'
	\ex {
	\gll {Í dag} hevur Petur \textit{einki} [\sub{VP} \textbf{givið} \textbf{Mariu} t\textsubscript{o}]. \\
		today has Peter nothing {} given Mary\\}\jambox*{Fa}
	\glt `Today, Peter hasn't given Mary anything.'
	\ex {
	\gll Jeg har \textit{ingen} \textit{bøger} [\sub{VP} \textbf{lånt} \textbf{børnene} t\textsubscript{o}].\\
	I have no books {} lent children-the\\}\jambox*{WJ/Scan1}
	\glt `I haven't lent the children any books.'
	\z 

\ex This behavior is called the \emph{Anti-Holmberg Effect} \citep{foxCyclicLinearizationSyntactic2005,engelsScandinavianNegativeIndefinites2012}
\z

\begin{tcolorbox}[width=\linewidth]
\textsc{Claim:} NegShift does not obey Holmberg's Generalization. 
\end{tcolorbox}

%------------------------------------
\subsection{Landing site} \label{sec:Landing}
%------------------------------------
\ea OS occurs to the left of all adverbials.
	\ea 
	\gll Jeg lånte \emph{hende\sub{IO}} faktisk\sub{Adv} aldri\sub{Adv} [\sub{VP} t\sub{IO} bøgerne].\\
	I lent her actually never {} {} books-the\\
	\glt `I actually never lent her the books' 
	\z 

\ex NegShift occurs to the right of most adverbials with the exception of low adverbials \citep{nilsenAdverbsAshift1997,svenoniusStrainsNegationNorwegian2002}.
	\ea 
	\gll Jeg har faktisk\sub{Adv} \emph{ingen} \emph{bøger}\sub{DO} [\sub{VP} lånt Othilia t\sub{DO}].\\
	I have actually no books {} lent Othilia \\
	\glt `I didn't actually lend Othilia any books.'
	\ex 
	\gll Fænsene har på \emph{intet} \emph{tidspunkt}\sub{O} tidlig\sub{Adv} slått av TV’en.\\
	fans-the have at no time early turned off TV-the\\
	\glt `The fans have at no time turned the TV off early'
	\z 


\z

\begin{tcolorbox}[width=\linewidth]
\textsc{Claim}: OS and NegShift move to different locations.
\end{tcolorbox}

%------------------------------------
\subsection{Interaction of NegShift and OS} \label{sec:NEG-OS}
%------------------------------------
\ea When there are both weak object pronouns and NIs in the sentence they both shift to their respective landing sites. 
	\ea
	\label{ex:NegOS}
	\gll Jeg lånte \textit{hende} fraktisch \textit{ingen} \textit{bøger}.\\
	I lent her actually no books\\
	\glt `I didn't actually lend her any books.'
	\z 
\ex When the verb remains in situ OS is blocked from occurring but NegShift is still allowed to occur. 
	\ea 
	\gll Jeg har \emph{ingen} \emph{bøger} lånt \emph{hende}.\\
	I have no books lent her\\
	\glt `I haven't lent her any books
	\z
\z

\begin{tcolorbox}[width=\linewidth]
\textsc{Claim:} OS and NegShift are not the same.

\textsc{Big Claim:} OS and NegShift have different triggers for movement. 
\end{tcolorbox}


%------------------------------------
\section{Deriving of NegShift} \label{sec:ZEIJLSTRA}
%------------------------------------

\ea Following \cite{zeijlstraSyntacticallyComplexStatus2011}, I assume the NI moves to Spec,NegP following the copy theory of movement \citep{chomskyMinimalistProgramLinguistic1993} to value a [$u$\textsc{Neg}] feature.
	\ea Copy theory of movement required for NIs having split-scope \citep{iatridouNegativeDPsAMovement2011,zeijlstraSyntacticallyComplexStatus2011}. 
	\ex Split-scope is evident when modals and other auxiliaries are present and the indefinite interpretation is in the lower copy while the negative operator is interpreted in the higher copy.
	\z 

\ex
\resizebox*{0.75\linewidth}{!}{
\begin{forest}
for tree={s sep=10mm, inner sep=0, l=0}
[CP [DP\\\emph{Jeg}]
	[C´ [C\\\emph{har}]
		[TP [<DP>]
			[T´ [<T>]
				[NegP [DP\\\emph{ingen bøger}\\{[\textsc{Neg}]}, name=tgt]
					[Neg´ [Neg\\{[\sout{$u$\textsc{Neg}}]}]
						[\emph{v}P [<DP>]
							[\emph{v´} [\emph{v}\\\emph{lånt}]
								[VP [DP\\\emph{børnene}]
									[V´ [<V>]
										[DP\\\emph{ingen bøger}\\{[\textsc{Neg}]}, name=src]
									]
								]
							]
						]
					]
				]
			]
		]
	]
] 
\draw[->] (src) to[out=south west,in=south] (tgt); 
\end{forest}}

\ex The derivation is then sent to PF where the higher copy is pronounced and the lower copy is deleted.

\ex We see that following simple syntactic operations we derive the behavior of NegShifting. 
	\ea Simple feature valuing allows for the \emph{Anti-Holmberg Effect}.
	\ex Also explains the difference in landing sites between NegShift and OS.
	\ex If OS is entirely prosodic in nature than this explains why it moves to different locations and interacts with NegShift in the way that it does.\footnote{See \cite{brinkerhoffHolmbergGeneralizationSyntaxprosody2021} for a more detailed discussion on this.} 
	\z  
\z 

%------------------------------------
\section{Prosodic restrictions on NegShift} \label{sec:PROSODY}
%------------------------------------

\ea However, syntax does not appear to be the only factor determining NegShift.  

\ex \citet[65f]{christensenInterfacesNegationSyntax2005} claims that the ``weight" of the NI plays a crucial factor in whether or not NegShift occurs. 
	\ea 
	\gll Jeg har \textit{intet}\textsubscript{o} hørt t\textsubscript{o}.\\
	I have nothing heard\\
	\glt  `I havn't heard anything.'
	\ex 
	\gll Jeg har [\textit{intet} \textit{nyt}]\textsubscript{o} hørt t\textsubscript{o}.\\
	I have nothing new heard\\
	\glt `I haven't heard anything new'
	\ex[*] {
	\gll Jeg har [\textit{intet} \textit{nyt} \textit{i} \textit{sagen}]\textsubscript{o} hørt t\textsubscript{o}.\\
	I have nothing new in case-\textsc{det} heard\\
	\glt `I haven't heard anything new about the case.'
	}
	\ex[*] {
	\gll Jeg har [\textit{intet} \textit{nyt} \textit{i} \textit{sagen} \textit{om} \textit{de} \textit{stjålne} \textit{malerier}]\textsubscript{o} hørt t\textsubscript{o}.\\
	I have nothing new in case-\textsc{det} about the stolen paintings heard\\
	}
	\glt `I haven't heard anything new in the case about the stolen paintings.'
	\z

\ex In those instances where the NI is too large one potential repair is to strand the PP while moving just the pronoun or using the negative particle \textit{ikke} and a NPI.
	\ea Jeg har \textit{intet}\textsubscript{i} hørt t\textsubscript{i} [\textsubscript{PP} i sagen om de stjålne malerier ].
	\ex Jeg har \textit{ikke} hørt [ \textit{noget} i sagen om de stjålne malerier ].
	\z 
 
\ex This same behavior has also been remarked upon by \citet{penkaNegativeIndefinites2011} for Swedish.
	\ea 
	\gll Men mänskligheten har \textit{ingenting}\textsubscript{o} lärt sig t\textsubscript{o}.\\
	but mankind-the have nothing taught themselves\\
	\glt `But mankind haven't taught themselves anything.'
	\ex[?] {
	\gll Vi hade \textit{inga} \textit{grottor}\textsubscript{o} undersökt t\textsubscript{o}.\\
	we have no caves explored\\
	}
	\glt `We haven't explored any caves.'
	\z 

\ex This suggests that during the derivation part of the higher copy is deleted leaving only the NI. In the lower copy the NI is deleted.
\\
\label{ex:tree}
	\resizebox*{\linewidth}{!}{
	\begin{tikzpicture} 
	\tikzset{every tree node/.style={align=center,anchor=north}} 
	\Tree [.CP 	DP\\\emph{Anna} 
				[.C´ C\\\emph{har} 
				[.TP <DP> 
					[.T´ T 
						[.NegP  
							[.DP \edge[roof]; {intet nyt \sout{i sagen om de stjålne malerier}}
							]
							[.Neg´ Neg 
								[.vP <DP>
								[.v´ \emph{hørt}   
									[.VP <V> 
										[.DP \edge[roof]; {\sout{intet nyt} i sagen om de stjålne malerier}
										]
									]
								]
							]
							]    
						]
					] 
				]
				] 
			] 
	\end{tikzpicture}}
\z 
\begin{tcolorbox}[width=\linewidth]
\textsc{Question:} What decides which material is deleted?\\
\textsc{Claim:} Prosody is responsible for restricting and regulating the amount of material that is allowed in the \textit{Mittelfeld}.
\end{tcolorbox}

%------------------------------------
\subsection{PF deletion} \label{sec:PFDeletion}
%------------------------------------
\ea  Following the copy theory of movement \citet{chomskyMinimalistProgramLinguistic1993}, multiple copies of the NI will be present at PF spell-out. 
\ex A question arises as to whether or not phases are have any bearing on the facts presented above. 
\ex There are two potential answers
	\ea Phases do not play a role in determining phonological/prosodic behavior.  
	\ex Phases do play a role in determining phonological/prosodic behavior.
	\z 
\ex Recent evidence from \citet{weberPhasebasedConstraintsMatch2020} suggests that phases do in fact play a role in determining phonological behavior. 
\ex In order for the phonology to interact with two copies at the same time, the base generated position and the landing site both must belong to the same phase. 
	\ea This requires that NegP and the rest of the Cinquean hierarchy of adverbials \citep{cinqueAdverbsFunctionalHeads1999} belong to the same phase as \emph{v}P.
	\z

\ex \citet{kandybowiczGrammarRepetitionNupe2008} makes the observation that multiple copies that are generated by the narrow syntax can be \emph{phonologically} realized when there is a identifiable PF well-formedness condition is avoided. 
	\ea Similarly, I argue that PF can also dictate the amount of material that is deleted in those copies. 
	\ex PF being able to delete is not new and was argued for by \citet{ottDeletionClausalEllipsis2016} to account for German clausal ellipsis. 
	\z 

\ex In the case of NegShift, there are restrictions on \emph{Mittelfeld} well-formedness, which I will call the \textsc{Light Mittelfeld Condition} (LMC).
	\ea It has been observed that only a limited amount of structure is allowed and a wide degree of variation is permitted in the \emph{Mittelfeld} (see \cite{haiderMittelfeldPhenomenaScrambling2017}). 

	\ex I argue that the largest unit that is allowed to remain in the \emph{Mittelfeld} in Scandinavian is a maximal prosodic word (ω\sub{max}).
	\z 

\ex Evidence for this comes from the size of the material that is allowed to ``shift" in these languages. 

\ex As observed for Danish only a pronoun or DP, consisting of just a D and NP, is allowed to occupy the \emph{Mittelfeld} when NegShift occurs. 
	\ea 
	\gll Jeg har \textit{intet}\textsubscript{o} hørt t\textsubscript{o}.\\
	I have nothing heard\\
	\glt  `I havn't heard anything.'
	\ex 
	\gll Jeg har [\textit{intet} \textit{nyt}]\textsubscript{o} hørt t\textsubscript{o}.\\
	I have nothing new heard\\
	\glt `I haven't heard anything new'
	\ex[*] {
	\gll Jeg har [\textit{intet} \textit{nyt} \textit{i} \textit{sagen}]\textsubscript{o} hørt t\textsubscript{o}.\\
	I have nothing new in case-\textsc{det} heard\\
	\glt `I haven't heard anything new about the case.'
	}
	\z  	

\ex This difference between Danish, which allows full DPs, and Swedish which tolerates full DPs, but prefers pronouns, suggests the \emph{Mittelfeld} in Swedish will delete copies until they are just the NI pronoun. 
	\ea This potential comes down to Swedish being a tonal language and Danish not being a tonal language.
	\z 

\ex The LMC deleting different amounts of material in Danish and Swedish can also explain Norwegian's lack of NegShift. 
	\ea Norwegian deletes everything as it prefers not to have any copies in the \emph{Mittelfeld}
	\ex This deleteion would then leave the valued [\textsc{Neg}] feature which gets pronounced as negation which then causes the lower copy to surface with a NPI.  
	\ex This behavior of total deletion is also attested in Danish and Swedish where negation and a NPI is always a potential instead of NegShift.
	\z  

\ex This results in a three-way system in Scandinavian languages. 
	\ea Those that delete until a full DP is left.
	\ex Those that delete until a pronoun is left.
	\ex Those that delete everything and have negation and a NPI.
	\z 

\ex There seems to be some differences in behavior between the tonal and atonal Scandinavian languages. 
	\ea According to \citet{thrainssonFaroeseOverviewReference2004,thrainssonSyntaxIcelandic2010} Faroese and Icelandic pattern the same as Danish in this regard. 
	\ex This further suggests that there is something unique about being a tonal language that limits the acceptability of NegShifting. 
	\z 

\ex This is summarized in Table \ref{tab:Paradigm}.
	\ea It will be noted that if you allow full NI DPs than you also allow pronouns and complete deletion, which results in a negation particle and an NPI.
	\ex If you allow NI pronouns then you allow a negation particle and an NPI
	\z 
\begin{table}[!ht]
	\centering
	\caption{Scandinavian acceptance of NegShift or NPI}
	\label{tab:Paradigm}
\begin{tabular}{lccc}
	\hline 
	& Full DPs & Pronouns & NPI\\
	\hline
	Icelandic & ✔︎ & ✔︎ & ✔︎ \\
	Faroese & ✔︎ & ✔︎ & ✔︎ \\
	Danish & ✔︎ & ✔︎ & ✔︎ \\
	Swedish & * & ✔︎ & ✔︎ \\
	Norwegian & * & * & ✔︎ \\
	\hline 
\end{tabular} 
\end{table}
\z 

%------------------------------------
\subsection{Particle Shifting in Scandinavian} \label{sec:HNPS}
%------------------------------------

\ea Independent evidence for shifting full DPs into the \emph{Mittelfeld} is observed in particle shifting in Scandinavian languages. 

\ex Following \citet[2]{holmbergRemarksHolmbergGeneralization1999} and \citet{faarlundSyntaxMainlandScandinavian2019} there is a difference in behavior between the different Scandinavian languages with what is allowed to shift across a verbal particle.
	\ea Danish objects, regardless of size, always precedes the verb particle. 
	\ex Norweigan, Icelandic, and Faroese are like English by shifting a pronoun across a particle and optionally for DPs.
	\ex Swedish does not allow anything to shift across the particles. 
	\z 

\ex \gllll Jeg skrev (nummeret/det) op (*nummeret/*det). \hfill Da\\
		Jeg skrev (nummeret/det) opp (nummeret/*det). \hfill No\\
		Jag skrev (*numret/*det) upp (numret/det). \hfill Sw\\
		I wrote (the-number/it) up (the-number/it)\\
\glt `I wrote the number/it down.'

\ex Additionally, Danish places restriction on the verbal compliment if it is too ``heavy" \citep[44f]{mullerDanishHeadDrivenPhraseInpreparation}. 
	\ea If it is larger than a simple DP shifting is blocked.\footnote{Examples are from \emph{KorpusDK} as reported by \citet{mullerDanishHeadDrivenPhraseInpreparation}.} 
	\ex \gll {[…]} så må partiet melde [holdninger] [ud], {[…]}\\
	~ then must party.\Def{} make stances out\\
	\glt `{[…]} then the party must make its stances clear, {[…]}'
	\ex \gll Den danske regering bør snart melde [ud], [at den støtter de amerikanske planer]\\
	the Danish government must soon make out that it supports the American plans\\
	\glt `The Danish government must soon make clear that it supports the American plans.'
	\z 

\ex This is further evidence that the LMC is an active constraint in these languages. 

\ex One explanation for this behavior is the difference in tonal quality between Danish, Norwegian, and Swedish \citep{erteschik-shirVariationMainlandScandinavian2020}.
\z 

\begin{tcolorbox}[width=\linewidth]

Prosody, in the form of the \textsc{Light Mittelfeld Condition}, plays an active role in regulating the size of the material in the \emph{Mittelfeld}
\end{tcolorbox}
%------------------------------------
\section{Conclusion} \label{sec:CONCLUSION}
%------------------------------------

\begin{tcolorbox}[width=\linewidth]
\centering
NegShift is derived by both syntactic and prosodic factors. It is syntactic in movement and prosody is responsible for restricting and regulating the amount of material that is allowed the surface in the higher copy.
\end{tcolorbox}

%------------------------------------
%BIBLIOGRAPHY
%------------------------------------

%\singlespacing
% \nocite{*}
% \printbibliography[heading=bibintoc, keyword={QP1}]
\printbibliography[heading=bibintoc]

% \part*{Appendix}

% %------------------------------------
% \section*{Cyclic Linearization} \label{sec:CL}
% %------------------------------------

% \ea Cyclic Linearization is a theory that was developed by \cite{foxCyclicLinearizationSyntactic2005} as a way to account for OS and Holmberg's Generalization.
% \ex This theory works by stipulating that spell-out is cyclic and order preserving. 

% \ex This proposal was extended by \citet{foxCyclicLinearizationSyntactic2005} and \citet{engelsMicrovariationObjectPositions2011,engelsScandinavianNegativeIndefinites2012} to account for quantifier movement (QM), which NegShift is a subset of. 
% \ex Under this proposal QM is subject to an ``Anti-Holmberg Effect'' or an ``Inverse Holmberg Effect'', which are identical in principle
% 	\ea Under Holmberg's Generalization, OS can only apply if the verb has undergone movement from V to T to C.
% 	\ex The Anti-Holmberg Effect explains that only when the verb remains in situ can we have QM, which is the result of the ordering operations between the different phases being in agreement. 
% 	\z
% 	\vspace{6pt} 

% \ex In order to account for OS, \citeauthor{foxCyclicLinearizationSyntactic2005} propose that the during the spell-out of the VP phase the V is the leftmost element in its phase\footnote{The position of the V at the left-edge of the phase could be due to the movement of V to \textit{v} in which case it is actually the \textit{v}P that acts as the phase not the VP.} and at which point the ordering restrictions are in place which state that the V must precede the O. 

% \ex At this point the V moves to T and then to C at this point the object is free to move to a position higher because the order that existed at the VP phase continues to hold. 
% \ex OS and string-vacuous Neg-Shift
% \vspace{6pt}
%  	\ea {}[\textsubscript{CP} S \rnode{b1}V … [\textsubscript{NegP} \rnode{A1}O adv [\textsubscript{VP} \rnode{b2}t\textsubscript{v} \rnode{A2}t\textsubscript{o} ]]]
% 	\psset{linearc=2pt} 
% 	\ncbar[angle=90]{->}{A2}{A1}  
% 	\ncbar[angle=-90,linestyle=dashed]{->}{b2}{b1} 
% 	\vspace{6pt}
% 	\ex VP Ordering: \textbf{V>O}\\
% 		CP Ordering: S>V, \textbf{V>O}, O>adv, adv>VP
% 	\z

% \ex If the DO were to move instead of the IO this would now result in the DO being ordered before the IO at the spell-out at the CP phase. 

% \ex In the case of NegShift, where it is able to shift across various phonological material it is proposed that the NI first moves to the left edge of the VP before spell-out of that phase. 
% \vspace{6pt}
% 	\ea {}[\textsubscript{CP} S aux … [\textsubscript{NegP} \rnode{A1}O [\textsubscript{VP} \rnode{A2}t\textsubscript{o}  V \rnode{A3}t\textsubscript{o} ]]]
% 	\psset{linearc=2pt} 
% 	\ncbar[angle=90]{->}{A3}{A2}  
% 	\ncbar[angle=90]{->}{A2}{A1}
% 	\ex VP Ordering: \textbf{O>V}\\
% 	CP Ordering: S>V, aux>O, O>adv, adv>VP → \textbf{O>V}
% 	\z

% \ex The benefit of using Cyclic Linearization comes from being able to account for why certain orders are fixed throughout the entire derivation.
% \z 

\end{document}