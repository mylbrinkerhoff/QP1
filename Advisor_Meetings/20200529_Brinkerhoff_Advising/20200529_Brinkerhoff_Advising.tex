\documentclass[12pt, letterpaper]{article}

%%BIBLIOGRAPHY- This uses biber/biblatex to generate bibliographies according to the 
%%Unified Style Sheet for Linguistics
\usepackage[main=american, german]{babel}% Recommended
\usepackage{csquotes}% Recommended
\usepackage[backend=biber,
		bibstyle=biblatex-sp-unified,
		citestyle=sp-authoryear-comp,
		maxcitenames=3,
		maxbibnames=99,
		natbib,
		url=false]{biblatex}
\addbibresource{QP1.bib}
\setcounter{biburlnumpenalty}{100}  % allow breaks at numbers
%\setcounter{biburlucpenalty}{100}   % allow breaks at uppercase letters
%\setcounter{biburllcpenalty}{100}   % allow breaks at lowercase letters

%%TYPOLOG
%\usepackage[compact]{titlesec}
%\titleformat{\section}[runin]{\normalfont\bfseries}{\thesection.}{.5em}{}[.]
%\titleformat{\subsection}[runin]{\normalfont\scshape}{\thesubsection}{.5em}{}[.]
\usepackage[hmargin=1in,vmargin=1in]{geometry}  %Margins          
\usepackage{graphicx}	%Inserting graphics, pictures, images 		
\usepackage{stackengine} %Package to allow text above or below other text, Also for HG 
\usepackage{fontspec} %Selection of fonts must be ran in XeLaTeX
\usepackage{amssymb} %Math symbols
\usepackage{amsmath}
\usepackage{setspace} %Linespacing
\usepackage{multicol} %Multicolumn text
\usepackage{enumitem} %Allows for continuous numbering of lists over examples, etc.
\usepackage{multirow} %Useful for combining cells in tablesbrew 
\usepackage{hanging}
\usepackage{fancyhdr}
\pagestyle{fancy}
\fancyhead[L]{\textit{QP Advising}} 
\fancyhead[R]{\textit{\today}} 
\fancyfoot[L,R]{} 
\fancyfoot[C]{\thepage} 
\renewcommand{\headrulewidth}{0.4pt}
%\usepackage{fourier-orns} %More useful symbols like bombs and jolly-roger, mostly for OT
\usepackage[colorlinks,allcolors={black},urlcolor={blue}]{hyperref} %allows for hyperlinks and pdf bookmarks
%\usepackage{url} %allows for urls
%\def\UrlBreaks{\do\/\do-} %allows for urls to be broken up
\usepackage[normalem]{ulem} %strike out text. Handy for syntax

%%FONTS
\setmainfont{Linux Libertine O}
\setsansfont{Linux Biolinum O}

%%PACKAGES FOR LINGUISTICS
%\usepackage{OTtablx} %Generating tableaux without using TIPA
\usepackage[noipa]{OTtablx} %Generating tableaux without using TIPA
\usepackage{linguex} %Linguistic examples
\usepackage{tikz} %Drawing Hasse diagrams
%\usepackage{pst-asr} %Drawing autosegmental features
\usepackage{pstricks} % required for pst-asr, OTtablx, pst-jtree.
\usepackage{pst-jtree} 	%Syntax tree draawing software
%\usepackage{lingmacros} %Various linguistic macros. Does not work with linguex.
%\usepackage{covington} %Another linguistic examples package.
%\usepackage{gb43} %Another linguistic examples package. Works with linguex very nicely.
\newcommand{\rcommentg}[1]{\hfill\raisebox{1.9\baselineskip}[0pt][0pt]{#1}} %This allows for hfill when doing glosses

%%TITLE INFORMATION
\title{Assignment 5}
\author{Mykel Loren Brinkerhoff}
\date{\today}


\begin{document}
	
%%If using linguex, need the following commands to get correct LSA style spacing
%% these have to be after  \begin{document}
	\setlength{\Extopsep}{6pt}
	\setlength{\Exlabelsep}{9pt}		%effect of 0.4in indent from left text edge
%%
	
%% Line spacing setting. Comment out the line spacing you do not need. Comment out all if you want single spacing.
%	\doublespacing
%	\onehalfspacing
	
\begin{center}
	{\Large \textbf{Advising Meeting }}\\
	\vspace{6pt}
	Mykel Loren Brinkerhoff
\end{center}
%\maketitle
%\maketitleinst
\thispagestyle{fancy}

%------------------------------------
\section{Outline}
%------------------------------------

\begin{itemize}
	\item Discussing \citet{engelsScandinavianNegativeIndefinites2012} and \citet{foxCyclicLinearizationSyntactic2005}
	\item Next Steps
\end{itemize}

%------------------------------------
\section{\cite{engelsScandinavianNegativeIndefinites2012} and \cite{foxCyclicLinearizationSyntactic2005}}
%------------------------------------

\ex. According to \citet{engelsScandinavianNegativeIndefinites2012} NegS is able to be accounted for through \posscitet{foxCyclicLinearizationSyntactic2005} cyclic linearization. 

\ex. Cyclic linearization has as its basic premise the idea that linearization occurs in each \textit{spell-out domain} which \citet{foxCyclicLinearizationSyntactic2005} equate with phases.
\a. \citeauthor{foxCyclicLinearizationSyntactic2005}, following standard conventions, claim that the spell-out domains are at DP, VP, and CP.

\ex. At each of these spell-out domains, the order that the syntactic elements are found in is fixed through the use of order preservation rules. 
\a. These rules are assumed to always be in force and are added to at each spell-out domain. 

\ex. As to how this relates to OS and NegS it is claimed that they are subject to two different effects and how the order preservation rules are formulated.
\a. In the case of OS, they are subject to Holmberg's Gerneralization which states:\\
Object Shift cannot apply across a phonologically visible category, which asymmetrically c-commands the object position except adjuncts.
\b. NegS is claimed to adhere to something that is the mirror image of Holmberg's Generalization called the “Inverse Holmberg Effect” \citep{engelsScandinavianNegativeIndefinites2012} in those situations where NegS occurs across a verb \textit{in-situ}.

\ex. The “Inverse Holmberg Effect” states:\\
Movement of a VP-internal element is permitted only when V-raising does not take place \citep[30]{foxCyclicLinearizationSyntactic2005}.

\ex. When OS and \textit{string-vacuous} NegS occur they both adhere to the same ordering relations: \\
\a.
\I[CP S \rnode{b1}V … \I[NegP \rnode{A1}O adv \I[VP \rnode{b2}t\textsubscript{v} \rnode{A2}t\textsubscript{o} ]]]
\psset{linearc=2pt} 
\ncbar[angle=90]{->}{A2}{A1}  
\ncbar[angle=-90,linestyle=dashed]{->}{b2}{b1} 
\\
\b. VP Ordering: \textbf{V>O}\\
CP Ordering: S>V \textbf{V>O} O>adv adv>VP
\z.

\ex. Because the ordering relations are maintained across the derivation there is no violation, 

\ex. However when the verb  or something else remains in-situ OS is blocked from occurring because there is a rule contradiction
\a. *\I[CP S \rnode{b1}V … \I[NegP \rnode{A1}O adv \I[VP \rnode{b2}t XP \textsubscript{v} \rnode{A2}t\textsubscript{o} ]]]
\psset{linearc=2pt} 
\ncbar[angle=90]{->}{A2}{A1}  
\ncbar[angle=-90,linestyle=dashed]{->}{b2}{b1} 
\\
\b. VP Ordering: V>XP, \textbf{XP>O} \\
CP Ordering: S>V, V>O, O>adv, adv>VP → \textbf{O>XP}
\z.

\ex. There two effects are associated with two different preservation rules at VP. 
\a.  OS and string-vacuous NegS: \textbf{V>O}. 
\b. Verb \textit{in-situ} NegS: \textbf{O>V}
\z.

\ex. These two orders are arrived at by whether or not the object has moved before the spell-out domain is closed. 

\ex. We can see these differences here:\\
\a.
\I[CP S aux … \I[NegP \rnode{A1}O adv \I[VP \rnode{A2}t\textsubscript{o}  V \rnode{A3}t\textsubscript{o} ]]]
\psset{linearc=2pt} 
\ncbar[angle=90]{->}{A3}{A2}  
\ncbar[angle=90]{->}{A2}{A1}
\\
\b. VP Ordering: \textbf{O>V}\\
CP Ordering: S>V, aux>O, O>adv, adv>VP → \textbf{O>V}
\z.

\ex. However, this seems very ad-hoc and would suggest that there must be some sort of feature that motivates the movement of the Object to spec,VP. This is not explained in \citet{engelsScandinavianNegativeIndefinites2012}. 

\ex. If \citet{engelsScandinavianNegativeIndefinites2012} is correct then there must be some sort of motivation for the movement to spec,VP besides as a requirement for explaining why NegS is allowed to occur across verbs \textit{in-situ}.

\ex. I think this is a flaw in the analysis and needs to be spelled out if this analysis is to succeed.


%------------------------------------
\section{Next steps}
%------------------------------------

\ex. Moving forward I think trying to provide the motivation for why NegS first moves to spec,VP would be a good first step

\ex. I think that at this point moving forward with designing an experiment to test which NegS is allowed and if there is a preference for pronominals over full DPs in the different varieties of Scandinavian should be my top priority
 
%\ex. “The tale of two displacements: OS and NegS in Mainland Scandinavian” 

%------------------------------------
%BIBLIOGRAPHY
%------------------------------------

%\singlespacing
%\nocite{*}
\printbibliography

%------------------------------------
%SNIPPETS
%------------------------------------
%\ex. 
%\jtree[xunit=2.6em,yunit=1em] 
%\def\\{[labelgapb=-1.2ex]}%\@1 
%\everymath={\rm}% 
%\! = {CP}
%:({XP}\\{$\scriptstyle [wh]$}@A1 ) {$C’$}
%:({$Cˆ0$}\\{$\scriptstyle [wh,Q]$})
%{$\langle\, TP\,\rangle$}
%<tri>{\dots\quad\rnode[b]{A2}{\it t}\quad\dots}.
%%\nccurve[angleA=-150,angleB=-90,ncurv=1]{->}{A2}{A1} 
%\ncbar[angleA=-90,angleB=-90,armA=1em, armB=1em,linearc=.6ex]{->}{A2}{A1}
%\endjtree
\end{document} 