% !TEX TS-program = xelatex
% !TEX encoding = UTF-8 Unicode		
\documentclass[12pt, letterpaper]{article}

%%BIBLIOGRAPHY- This uses biber/biblatex to generate bibliographies according to the 
%%Unified Style Sheet for Linguistics
\usepackage[main=american, german]{babel}% Recommended
\usepackage{csquotes}% Recommended
\usepackage[backend=biber,
		style=unified,
		maxcitenames=3,
		maxbibnames=99,
		natbib,
		url=false]{biblatex}
\addbibresource{QP1.bib}
\setcounter{biburlnumpenalty}{100}  % allow URL breaks at numbers
%\setcounter{biburlucpenalty}{100}   % allow URL breaks at uppercase letters
%\setcounter{biburllcpenalty}{100}   % allow URL breaks at lowercase letters

%%TYPOLOGY
%\usepackage{fourier} % This allows for the use of certain wingdings like bombs, frowns, etc.
\usepackage[svgnames]{xcolor} % Specify colors by their 'svgnames', for a full list of all colors available see here: http://www.latextemplates.com/svgnames-colors
%\usepackage[compact]{titlesec}
%\titleformat{\section}[runin]{\normalfont\bfseries}{\thesection.}{.5em}{}[.]
%\titleformat{\subsection}[runin]{\normalfont\scshape}{\thesubsection}{.5em}{}[.]
\usepackage[hmargin=1in,vmargin=1in,headheight=15pt]{geometry}  %Margins          
\usepackage{graphicx}	%Inserting graphics, pictures, images 		
\usepackage{stackengine} %Package to allow text above or below other text, Also helpful for HG weights 
\usepackage{amssymb} %Math symbols
\usepackage{amsmath}
\usepackage{setspace} %Linespacing
\usepackage{multicol} %Multicolumn text
\usepackage{enumitem} %Allows for continuous numbering of lists over examples, etc.
\usepackage{multirow} %Useful for combining cells in tablesbrew 
\usepackage{hanging}
\usepackage{fancyhdr} %Allows for the 
\pagestyle{fancy}
\fancyhead[L]{\textit{QP Advising}} 
\fancyhead[R]{\textit{\today}} 
\fancyfoot[L,R]{} 
\fancyfoot[C]{\thepage} 
\renewcommand{\headrulewidth}{0.4pt}
% \usepackage{fourier-orns} %More useful symbols like bombs and jolly-roger, mostly for OT
\usepackage[colorlinks,allcolors={black},urlcolor={blue}]{hyperref} %allows for hyperlinks and pdf bookmarks
%\usepackage{url} %allows for urls
%\def\UrlBreaks{\do\/\do-} %allows for urls to be broken up
\usepackage[normalem]{ulem} %strike out text. Handy for syntax

%%FONTS
\usepackage{fontspec} %Selection of fonts must be ran in XeLaTeX
\setmainfont{Linux Libertine O}
\setsansfont{Linux Biolinum O}

%%PACKAGES FOR LINGUISTICS
% \usepackage{OTtablx} %Generating tableaux with using TIPA
\usepackage[noipa]{OTtablx} % Use this one generating tableaux without using TIPA
% \usepackage[notipa]{ot-tableau} % Another tableau drawing packing use for posters.
\usepackage{linguex} % Linguistic examples
\usepackage{tikz} % Drawing Hasse diagrams
% \usepackage{pst-asr} % Drawing autosegmental features
\usepackage{pstricks} % required for pst-asr, OTtablx, pst-jtree.
% \usepackage{pst-jtree} 	% Syntax tree drawing software
\usepackage{tikz-qtree}	% Another syntax tree drawing software. Uses bracket notation.
% \usepackage{forest}	% Another syntax tree drawing software. Uses bracket notation.
% \usepackage{lingmacros} % Various linguistic macros. Does not work with linguex.
% \usepackage{covington} % Another linguistic examples package.
% \usepackage{gb4e} % Another linguistic examples package. Works with linguex very nicely.
% \usepackage{langsci-avm} % Package for drawring attribute-value matricies
\newcommand{\rcommentg}[1]{\hfill\raisebox{1.9\baselineskip}[0pt][0pt]{#1}} %This allows for hfill when doing glosses

%%TITLE INFORMATION
\title{Brinkerhoff QP Advising}
\author{Mykel Loren Brinkerhoff}
\date{\today}


\begin{document}
	
%%If using linguex, need the following commands to get correct LSA style spacing
%% these have to be after  \begin{document}
	\setlength{\Extopsep}{6pt}
	\setlength{\Exlabelsep}{9pt}		%effect of 0.4in indent from left text edge
%%
	
%% Line spacing setting. Comment out the line spacing you do not need. Comment out all if you want single spacing.
%	\doublespacing
%	\onehalfspacing
\singlespacing
	
\begin{center}
	{\Large \textbf{Advising Meeting}}\\
	\vspace{6pt}
	Mykel Loren Brinkerhoff
\end{center}
%\maketitle
%\maketitleinst
\thispagestyle{fancy}

%------------------------------------
\section*{Outline} \label{sec:outline}
%------------------------------------
\begin{itemize}
	\item Discussion of Swedish Negative Indefinites
\end{itemize}

% Next week:
% - Three expressions full determiners and pronouns what is going on where and how can I tell the difference between 
% - talking about different analyses and their different structures and their interaction with the prosody.
% - are they similar syntactic and morphologically
% 	- reference has to be in the discourse for none. Can we see this for the paper.

%------------------------------------
\section*{Structure for Swedish NIs} \label{sec:structure}
%------------------------------------


%------------------------------------
\subsection*{Swedish expressions} \label{sec:diagnositics}
%------------------------------------

\ex. There are three different expressions in Swedish that are identified as being associated with negative indefinites. These are divided into two pronouns and one determiner.
	\a. Pronouns
		\a. \textit{ingenting} 'nothing'
		\b. \textit{ingen} 'no one, nobody, none'
		\z.
	\b. Determiner
		\a. \textit{ingen}	'no'
		\z. 

\ex. We see that one of the clearest diagnostics between the pronouns and determiners is what sort of material they are allowed to appear in front of.

\ex. In the case of the determiner \textit{ingen} it is only allowed to appear immediately proceeding nouns and adjectives. 
	\ag. Jag har \textit{ingen} \textit{röd} \textit{bil} sett\\
	I have no red car seen\\
	'I haven't seen any car'
	\z. 

\ex. On the other hand, the pronouns are not seen in this exact same distribution. They are allowed to appear before determiners, adverbs, nouns\footnotetext{This is only true if the noun is part of a ditransitive construction. Additionally, I am not entirely sure if Swedish allows nouns to appear without a determiner. I will look into this to double check.}, and prepositions. 
	\ag. Jag har \textit{ingenting} sett.\\
	I have nothing seen\\
	'I haven't seen anything'
	\z. 

\ex. Using this will help in determining what type of element I have in each of the sentences.


%------------------------------------
\subsection*{The syntactic and prosodic structures of NIs} \label{sec:prosodic}
%------------------------------------

\ex. For each of the expressions discussed in the previous section we can assume that there is only two syntactic structures that they potentially could correspond to.
	\begin{multicols}{2}
	\a. Determiners\\
	\begin{tikzpicture}
	\tikzset{every tree node/.style={align=center,anchor=north}} 
	\Tree [.DP D\\\textit{ingen} [.NP\\… ]	]
	\end{tikzpicture}
	\b. Pronouns\\
	\begin{tikzpicture}
	\tikzset{every tree node/.style={align=center,anchor=north}} 
	\Tree [.DP\\\textit{ingenting} ]
	\end{tikzpicture}
	\z.
	\end{multicols}

\ex. The two sets of syntactic structures are in line with most accounts that pronouns are simple heads of DPs with no NP compliment whereas determiners are the heads of DPs which do have a NP compliment.

\ex. For the prosodic structure that is assumed for the above expressions there are several reliable diagnostics proposed by \cite{myrbergProsodicHierarchySwedish2015}.

\ex. According to \cite{myrbergProsodicHierarchySwedish2015}, because we observe tonal accents on the NI pronouns we must assume that they are both minimal and maximal prosodic words. 
	\a. Prosodic structure for pronouns\\
	\begin{tikzpicture} 
	\tikzset{every tree node/.style={align=center,anchor=north}} 
	\Tree [.$\omega_{max}$ [.$\omega_{min}$\\\textit{ingen} ] ] 
	\end{tikzpicture}

\ex. Under their analysis the domain for tonal accent is that of the maximal prosodic word, whereas the domain for stress and syllabification is the minimal prosodic word. 
	\a. Evidence for this comes from the incorporation of pronouns into words. In those cases, we see that prosodically weak pronouns that begin with a [h] word initially lose this segment under syllabification and whatever tonal distinction they had is erased and replaced by the tonal pattern of the host.	
	\b. This same behavior is also observed in compounds that syllabification will eliminate certain segments and that what ever tonal accents the words had original are replaced by the compound accent. 

\ex. This compound accent is also used when the NI determiner is present it forms it own minimal $\omega$ but combines with the noun to form a maximal $\omega$.
	\a. Prosodic structure for determiners.\\
	\begin{tikzpicture} 
	\tikzset{every tree node/.style={align=center,anchor=north}} 
	\Tree [.$\omega_{max}$ [.$\omega_{min}$\\\textit{inga}\\no ] [.$\omega_{min}$\\\textit{böck}\\books ] ] 
	\end{tikzpicture}


% \begin{tikzpicture} 
% \tikzset{every tree node/.style={align=center,anchor=north}} 
% \Tree [.S 	[.NP Det\\the N\\cat ] 
% 			[.VP V\\sat [.PP P\\on 
% 						[.NP Det\\the N\\mat ] ] ] ] 
% \end{tikzpicture}

%------------------------------------
\section*{Where next} \label{sec:next}
%------------------------------------
\begin{itemize}
	\item Develop an analysis of the syntactic and morphological similarities or lack thereof. 
	\item Continue to process the data. 
		\begin{itemize}
			\item I received a lot of really good feedback from the first Pre-290 presentation.
			\item I have several ideas of how to more effectively process the data quickly and still maintaining some amount of accuracy. 
		\end{itemize}
	\item Also received some questions concerning NegShifts interaction with other adverbs like \textit{aldri} 'never' and the ordering of NegShift with respect to shifted object pronouns. 
	\item This would involve reading \cite{broekhuisUnificationObjectShift2020} as he has some good discussion on object shift and object scrambling. Which I would like to discuss next time if possible.
\end{itemize}

%------------------------------------
%BIBLIOGRAPHY
%------------------------------------

%\singlespacing
%\nocite{*}
\printbibliography

\end{document} 