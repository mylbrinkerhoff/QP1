% !TEX TS-program = xelatex
% !TEX encoding = UTF-8 Unicode		

\documentclass[12pt, letterpaper]{article}

%%BIBLIOGRAPHY- This uses biber/biblatex to generate bibliographies according to the 
%%Unified Style Sheet for Linguistics
\usepackage[main=american, german]{babel}% Recommended
\usepackage{csquotes}% Recommended
\usepackage[backend=biber,
		style=unified,
		maxcitenames=3,
		maxbibnames=99,
		natbib,
		url=false]{biblatex}
\addbibresource{link2.bib}	%	Use when on Laptop
% \addbibresource{link.bib} %	Use when on desktop
\setcounter{biburlnumpenalty}{100}  % allow URL breaks at numbers
%\setcounter{biburlucpenalty}{100}   % allow URL breaks at uppercase letters
%\setcounter{biburllcpenalty}{100}   % allow URL breaks at lowercase letters

%%TYPOLOGY
\usepackage[svgnames]{xcolor} % Specify colors by their 'svgnames', for a full list of all colors available see here: http://www.latextemplates.com/svgnames-colors
%\usepackage[compact]{titlesec}
%\titleformat{\section}[runin]{\normalfont\bfseries}{\thesection.}{.5em}{}[.]
%\titleformat{\subsection}[runin]{\normalfont\scshape}{\thesubsection}{.5em}{}[.]
\usepackage[hmargin=1in,vmargin=1in]{geometry}  %Margins          
\usepackage{graphicx}	%Inserting graphics, pictures, images 		
\usepackage{stackengine} %Package to allow text above or below other text, Also helpful for HG weights 
\usepackage{fontspec} %Selection of fonts must be ran in XeLaTeX
\usepackage{amssymb} %Math symbols
\usepackage{amsmath} % Mathematical enhancements for LaTeX
\usepackage{setspace} %Linespacing
\usepackage{multicol} %Multicolumn text
\usepackage{enumitem} %Allows for continuous numbering of lists over examples, etc.
\usepackage{multirow} %Useful for combining cells in tablesbrew 
\usepackage{hanging}
\usepackage{fancyhdr} %Allows for the 
\pagestyle{fancy}
\fancyhead[L]{\textit{QP Advising}} 
\fancyhead[R]{\textit{\today}} 
\fancyfoot[L,R]{} 
\fancyfoot[C]{\thepage} 
\renewcommand{\headrulewidth}{0.4pt}
\setlength{\headheight}{14.5pt} % ...at least 14.49998pt
% \usepackage{fourier} % This allows for the use of certain wingdings like bombs, frowns, etc.
% \usepackage{fourier-orns} %More useful symbols like bombs and jolly-roger, mostly for OT
\usepackage[colorlinks,allcolors={black},urlcolor={blue}]{hyperref} %allows for hyperlinks and pdf bookmarks
% \usepackage{url} %allows for urls
% \def\UrlBreaks{\do\/\do-} %allows for urls to be broken up
\usepackage[normalem]{ulem} %strike out text. Handy for syntax

%%FONTS
\setmainfont{Linux Libertine O}
\setsansfont{Linux Biolinum O}
% \setmonofont{DejaVu Sans Mono}

%%PACKAGES FOR LINGUISTICS
%\usepackage{OTtablx} %Generating tableaux with using TIPA
\usepackage[noipa]{OTtablx} % Use this one generating tableaux without using TIPA
%\usepackage[notipa]{ot-tableau} % Another tableau drawing packing use for posters.
% \usepackage{linguex} % Linguistic examples
\usepackage{langsci-gb4e} % Language Science Press' modification of gb4e
% \usepackage{langsci-avm} % Language Science Press' AVM package
\usepackage{tikz} % Drawing Hasse diagrams
%\usepackage{pst-asr} % Drawing autosegmental features
\usepackage{pstricks} % required for pst-asr, OTtablx, pst-jtree.
% \usepackage{pst-jtree} 	% Syntax tree draawing software
\usepackage{tikz-qtree}	% Another syntax tree drawing software. Uses bracket notation.
% \usepackage{forest}	% Another syntax tree drawing software. Uses bracket notation.
%\usepackage{lingmacros} % Various linguistic macros. Does not work with linguex.
%\usepackage{covington} % Another linguistic examples package.

%%TITLE INFORMATION
\title{20201030 QP Advising}
\author{Mykel Loren Brinkerhoff}
\date{\today}


\begin{document}
	
%%If using linguex, need the following commands to get correct LSA style spacing
%% these have to be after  \begin{document}
	% \setlength{\Extopsep}{6pt}
	% \setlength{\Exlabelsep}{9pt}		%effect of 0.4in indent from left text edge
%%
	
%% Line spacing setting. Comment out the line spacing you do not need. Comment out all if you want single spacing.
%	\doublespacing
%	\onehalfspacing
	
\begin{center}
	{\Large \textbf{Advising Meeting}}\\
	\vspace{6pt}
	Mykel Loren Brinkerhoff
\end{center}
%\maketitle
%\maketitleinst
\thispagestyle{fancy}

%------------------------------------
\section*{Outline of Handout} \label{sec:OUTLINE}
%------------------------------------
\begin{itemize}
	\item Discussion of \cite{broekhuisUnificationObjectShift2020} in §\ref{sec:BROEKHUIS}
	\item Discussion of Scandinavian Pronoun structure in §\ref{sec:PRONOUNS}
	\item Next steps in §\ref{sec:NEXT}
\end{itemize}

%------------------------------------
\section{Broekhuis 2020} \label{sec:BROEKHUIS}
%------------------------------------

\ea \cite{broekhuisUnificationObjectShift2020} explores the possibility that object shift and object scrambling are the same phenomenon and concludes that this is in fact the case.
\ex However, \citeauthor[417f]{broekhuisUnificationObjectShift2020} does point out that weak pronominal object shift behaves differently than full DP objects in what loci there are allowed to inhabit. In the case of weak pronominals they are required to appear outside of the \textit{v}P if there is no intervening phonological material (i.e., Holmberg's Generalization \cite{holmbergWordOrderSyntactic1986,holmbergRemarksHolmbergGeneralization1999}).
\ex \citeauthor{broekhuisUnificationObjectShift2020} does have some interesting discussion about the interaction of NegShift and pronominal OS. 
\ex Citing examples from \citet[163ff]{christensenInterfacesNegationSyntax2005}, Broekhuis shows this pair of examples:
	\ea \label{ex:NegShift}
	\gll Jeg har <ingen bøger> lånt hende <*ingen bøger>.\\
	I have no books lent her\\
	\glt `I haven't lent her any books
	\ex \label{ex:NegOS}
	\gll Jeg lånte henda fraktisch ingen bøger.\\
	I lent her actually no books\\
	\glt `I didn't actually lend her any books.'
	\z
\ex In \ref{ex:NegShift}, we see that when we have a negative object that it shifts to a position higher than the \textit{v}P if it were to remain in-situ as it would be ungrammatical and would require the use of \textit{ikke} `not' and the NPI \textit{nogen}.
	\ea
	\gll Jeg har \textit{ikke} lånt hende \textit{nogen} bøger.\\
	I have not lent her any books.\\
	\glt `I haven't lent her any books.'
	\z
\ex However, when the main verb has raised to C⁰ as in \ref{ex:NegOS} then the weak pronominal moves to a position higher than the adverb \textit{fraktisch} `actually'. The negative object is not able to move to the similar position that is higher than the adverb. Additionally, this results in OS > NegShift and according to \citeauthor{broekhuisUnificationObjectShift2020} this is a universal fact.
\ex This does help us see that that even though these two phenomena appear to be similar they are in fact slightly different, due to the differences in the where the different movement operations' target is.
\z 

%------------------------------------
\section{Scandinavian pronouns} \label{sec:PRONOUNS}
%------------------------------------

\ea There two different approaches that we can take when accounting for the syntactic structure of the negative indefinite pronouns in Swedish. The two positions are: (a) the pronoun is the head of a DP on its own; or (b) the pronoun resides in D⁰ and takes a null NP complete.\footnote{This second option could also assume that the pronoun originated in NP and moved to D⁰ prior to spell-out of the DP phase.}
	\ea DP with no complement\\
	\begin{tikzpicture} 
	\tikzset{every tree node/.style={align=center,anchor=north}} 
	\Tree [.DP\\\emph{pronoun} ] 
	\end{tikzpicture}  
	\ex DP with null NP complement\label{ex:DN} \\
	\begin{tikzpicture} 
	\tikzset{every tree node/.style={align=center,anchor=north}} 
	\Tree [.DP D\\\emph{pronoun} [.NP\\$\varnothing$ ] ]
	\end{tikzpicture}
	\z
\ex The reason this question is 
\ex Evidence for one structure over the other comes from whether or not modification of the pronoun is allowed. 
	\ea I take modification to mean any addition to the syntactic structure beyond a bare head regardless of whether it is an adjunct or a compliment. 
	\z 
\ex We know that cross linguistically determiners always select an NP compliment and does not seem to select anything other than NPs and are resistant to any type of modification, with the exception of genitive constructions. 
\ex However, this is not the case with NPs which are able to be modified by a wide variety of syntactic elements including adjectives, prepositional phrases, relative clauses, and subordinate clauses. 
\ex Because of this difference in behavior between determiners and nouns, we will be able to determine which of the two syntactic structures is the correct one for the Swedish negative indefinite pronouns if modification of these pronouns is present.
\ex If pronouns are modified in any way through the addition of one these syntactic elements that are commonly associated with NPs, then it bares to reason that these syntactic elements are attaching to something that is an NP.
\ex Data will be drawn from across all of the Scandinavian languages because of the close similarity between the mainland Scandinavian languages syntactically.
\ex We observe in Danish that their negative indefinite pronouns are able to be modified with PPs and CPs as shown in \citet[218ff]{allanDanishComprehensiveGrammar1995}.
	\ea Prepositional phrases
		\ea 
		\gll Det er \textit{intet} [\textsubscript{PP} i vej-an ].\\
			It is nothing ~ in way-\textsc{det}\\
		\glt `There is nothing wrong'
		\ex 
		\gll Jeg kender \textit{ingen} [\textsubscript{PP} her i byen ].\\
			I know no-one ~ here in town\\
		\glt `I know no one in this town.'
		\z
	\ex Complimentizer phrases
		\ea 
		\gll Der er \textit{ingen}, [\textsubscript{CP} der har set ham ].\\
		there is no-one, ~ who has seen him\\
		\glt `Nobody has seen him'
		\ex
		\gll Der er \textit{intet} [\textsubscript{CP} at være bange for ].\\
		There is nothing ~ to be afraid of\\
		\glt `There is nothing to be afraid of'
		\z 
	\z
\ex This same behavior is also observed in Swedish \citep[197ff]{holmesSwedishComprehensiveGrammar2013}.
	\ea Prepositional phrases
		\ea \label{ex:inget}
		\gll Han äger \textit{inget} [\textsubscript{PP} av värde ].\\
			He owns nothing ~ of value\\
		\glt `He owns nothing of value.'
		\z 
	\ex Complimentizer phrases
		\ea
		\gll Jag såg \textit{ingen} [\textsubscript{CP} jag {kände igen} ].\\
		I saw no-one ~ I recognize\\
		\glt `I saw no one that I recognize.'
		\ex 
		\gll Jag har \textit{ingenting} [\textsubscript{CP} att säga ].\\
		I have nothing ~ to say\\
		\glt `I have nothing to say.'
		\z 
	\z
\ex Example \ref{ex:inget} shows the use of \textit{inget} instead of \textit{ingenting} which according to \citet{holmesSwedishComprehensiveGrammar2013} are fully interchangeable with each other.
\ex \citet[65f]{christensenInterfacesNegationSyntax2005} describes some interesting cases involving the weight of the NI which is quoted by \citet{penkaNegativeIndefinites2011}.
\ex Christensen shows that when a negative indefinite is sufficiently large movement is band from occurring in Danish.
	\ea
	\gll Jeg har \textit{intet} hørt.\\
	I have nothing heard\\
	\glt  `I havn't heard anything.'
	\ex 
	\gll Jeg har [\textit{intet} \textit{nyt}] hørt.\\
	I have nothing new heard\\
	\glt `I haven't heard anything new'
	\ex[*] {
	\gll Jeg har [\textit{intet} \textit{nyt} \textit{i} \textit{sagen}] hørt.\\
	I have nothing new in case-\textsc{det} heard\\
	\glt `I haven't heard anything new about the case.'
	}
	\ex[*] {
	\gll Jeg har [\textit{intet} \textit{nyt} \textit{i} \textit{sagen} \textit{om} \textit{de} \textit{stjålne} \textit{malerier}] hørt.\\
	I have nothing new in case-\textsc{det} about the stolen paintings heard\\
	}
	\z
\ex In those instances where the NI is too large one potential repair is to strand the prepositional phrase while moving just the pronoun.
	\ea Jeg har \textit{intet}\textsubscript{i} hørt t\textsubscript{i} [\textsubscript{PP} i sagen om de stjålne malerier ].\\
	\z   
\ex Based on this behavior of allowing modification of the pronoun, it is safe to assume that the syntactic structure is the one in \ref{ex:DN}.
\z

%------------------------------------
\section{Next steps} \label{sec:NEXT}
%------------------------------------
\begin{itemize}
	\item Continue looking into the difference in between the different negative indefinites.
	\item Work on sorting the corpus data.
\end{itemize}


%------------------------------------
%BIBLIOGRAPHY
%------------------------------------

%\singlespacing
%\nocite{*}
\printbibliography

\end{document}