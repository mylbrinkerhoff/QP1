@article{adgerPronounsPostposePF2007,
  title = {Pronouns {{Postpose}} at {{PF}}},
  author = {Adger, David},
  date = {2007-03},
  journaltitle = {Linguistic Inquiry},
  shortjournal = {Linguistic Inquiry},
  volume = {38},
  pages = {343--349},
  issn = {0024-3892, 1530-9150},
  doi = {10.1162/ling.2007.38.2.343},
  url = {http://www.mitpressjournals.org/doi/10.1162/ling.2007.38.2.343},
  urldate = {2020-02-12},
  file = {/Users/mybrink/Zotero/storage/H576N7XQ/Adger_2007_Pronouns Postpose at PF.pdf},
  langid = {english},
  number = {2}
}

@book{andersonAspectsTheoryClitics2005,
  title = {Aspects of the Theory of Clitics},
  author = {Anderson, Stephen R.},
  date = {2005},
  publisher = {{Oxford University Press}},
  location = {{New York}},
  doi = {10.1093/acprof:oso/9780199279906.001.0001},
  isbn = {978-0-19-927990-6 978-0-19-927991-3},
  keywords = {Clitics,Grammar; Comparative and general},
  pagetotal = {317},
  series = {Oxford Studies in Theoretical Linguistics}
}

@article{anderssenAcquisitionWordOrder2018,
  title = {The Acquisition of Word Order in {{L2 Norwegian}}: {{The}} Case of Subject and Object Shift},
  shorttitle = {The Acquisition of Word Order in {{L2 Norwegian}}},
  author = {Anderssen, Merete and Bentzen, Kristine and Busterud, Guro and Dahl, Anne and Lundquist, Björn and Westergaard, Marit},
  date = {2018-12},
  journaltitle = {Nordic Journal of Linguistics},
  shortjournal = {Nord J Linguist},
  volume = {41},
  pages = {247--274},
  issn = {0332-5865, 1502-4717},
  doi = {10.1017/S0332586518000203},
  url = {https://www.cambridge.org/core/product/identifier/S0332586518000203/type/journal_article},
  urldate = {2020-02-07},
  abstract = {This article reports on a syntactic acceptability judgement study of 59 adult L2/Ln learners of Norwegian and a group of native controls, studying subject and object shift. These constructions involve movement of (mainly) pronominal subjects or objects across negation/adverbs. Both subject shift and object shift display considerable micro-variation in terms of syntax and information structure, dependent on factors such as nominal type (pronoun vs. full DP), function (subject vs. object), and information status (given vs. new/focused). Previous studies have shown that Norwegian children have an early preference for the unshifted position in both constructions, but that they acquire subject shift relatively early (before age 3). Object shift, on the other hand, is typically not in place until after age 6–7. Importantly, children are conservative learners, and never shift elements that should not move in the adult language. The results of the current study show that L2/Ln learners do not make all the fine distinctions that children make, in that they have a clear preference for all subjects in shifted position and all objects in unshifted position, although some distinctions fall into place with increased proficiency. Importantly, unlike children, the L2/Ln learners are not conservative learners; rather, they over-accept syntactic movement in several cases. The equivalent to this in language production would be to apply syntactic movement where it is not attested in the target language, which would be the opposite behaviour to that observed in L1 children.},
  file = {/Users/mybrink/Zotero/storage/WLVT3QEJ/Anderssen et al_2018_The acquisition of word order in L2 Norwegian.pdf},
  langid = {english},
  number = {3}
}

@article{anderssenTopicalityComplexityAcquisition2012,
  ids = {anderssenTopicalityComplexityAcquisition2012a},
  title = {Topicality and {{Complexity}} in the {{Acquisition}} of {{Norwegian Object Shift}}},
  author = {Anderssen, Merete and Bentzen, Kristine and Rodina, Yulia},
  date = {2012-01},
  journaltitle = {Language Acquisition},
  shortjournal = {Language Acquisition},
  volume = {19},
  pages = {39--72},
  issn = {1048-9223, 1532-7817},
  doi = {10.1080/10489223.2012.633844},
  url = {http://www.tandfonline.com/doi/abs/10.1080/10489223.2012.633844},
  urldate = {2020-01-29},
  file = {/Users/mybrink/Zotero/storage/63WVRDEJ/Anderssen et al_2012_Topicality and Complexity in the Acquisition of Norwegian Object Shift.pdf;/Users/mybrink/Zotero/storage/FPDQTEGK/Anderssen et al_2012_Topicality and Complexity in the Acquisition of Norwegian Object Shift.pdf},
  langid = {english},
  number = {1}
}

@article{bennettLightestRightApparently2016,
  title = {Lightest to the {{Right}}: {{An Apparently Anomalous Displacement}} in {{Irish}}},
  shorttitle = {Lightest to the {{Right}}},
  author = {Bennett, Ryan and Elfner, Emily and McCloskey, James},
  date = {2016-04},
  journaltitle = {Linguistic Inquiry},
  shortjournal = {Linguistic Inquiry},
  volume = {47},
  pages = {169--234},
  issn = {0024-3892, 1530-9150},
  doi = {10.1162/LING_a_00209},
  url = {http://www.mitpressjournals.org/doi/10.1162/LING_a_00209},
  urldate = {2020-02-07},
  file = {/Users/mybrink/Zotero/storage/QD7NN6H3/Bennett et al_2016_Lightest to the Right.pdf},
  langid = {english},
  number = {2}
}

@article{bentzenFormPositionPronominal2019,
  title = {The Form and Position of Pronominal Objects with Non-Nominal Antecedents in {{Scandinavian}} and {{German}}},
  author = {Bentzen, Kristine and Anderssen, Merete},
  date = {2019-07-01},
  journaltitle = {The Journal of Comparative Germanic Linguistics},
  shortjournal = {J Comp German Linguistics},
  volume = {22},
  pages = {169--188},
  issn = {1572-8552},
  doi = {10.1007/s10828-019-09105-w},
  url = {https://doi.org/10.1007/s10828-019-09105-w},
  urldate = {2020-02-07},
  abstract = {The present paper discusses a possible correlation between the placement of pronominal objects with non-nominal antecedents in Norwegian, and the use of the pronouns es ‘it’ and das ‘that’ in German. For Norwegian object shift (OS), it has been shown that while pronominal objects with non-nominal antecedents generally do not shift, this is not the case when these elements take on the discourse function of continuing topics. In this paper, we show that a very similar pattern can be observed in German. However, this is not related to whether object pronouns scramble or not, but rather to which pronominal form is used to refer back to the clausal antecedent. In German, das is generally used to refer back to non-nominal antecedents, however, es is also sometimes an option. In this study, we find parallels between the use of OS and es, on the one hand, and lack of OS and das, on the other, and propose that the former is preferred when the proposition the proform refers back to is part of the common ground in the discourse. This ties in nicely with previous research on Norwegian OS, as in order for a proposition to constitute a continuing topic in the discourse, it has to be established as part of the interlocutors’ common ground.},
  file = {/Users/mybrink/Zotero/storage/6YXT93VL/Bentzen_Anderssen_2019_The form and position of pronominal objects with non-nominal antecedents in.pdf},
  keywords = {Clausal antecedents,Common ground,German,Norwegian,Object shift,Proforms,Scrambling},
  langid = {english},
  number = {2}
}

@article{bentzenObjectShiftSpoken2013,
  title = {Object {{Shift}} in Spoken {{Mainland Scandinavian}}: {{A}} Corpus Study of {{Danish}}, {{Norwegian}}, and {{Swedish}}},
  shorttitle = {Object {{Shift}} in Spoken {{Mainland Scandinavian}}},
  author = {Bentzen, Kristine and Anderssen, Merete and Waldmann, Christian},
  date = {2013-10},
  journaltitle = {Nordic Journal of Linguistics},
  shortjournal = {Nord J Linguist},
  volume = {36},
  pages = {115--151},
  issn = {0332-5865, 1502-4717},
  doi = {10.1017/S0332586513000218},
  url = {https://www.cambridge.org/core/product/identifier/S0332586513000218/type/journal_article},
  urldate = {2020-02-07},
  abstract = {Recent work on Object Shift (OS) suggests that this is not as uniform an operation as traditionally assumed. In this paper, we examine OS in the spontaneous speech of adults in large Danish, Norwegian and Swedish child language corpora in order to explore variation with respect to OS across these three languages. We evaluate our results against three recent strands of accounts of OS, namely a prosodic/phonological account, an account in terms of cognitive status, and an account in terms of information structure. Our investigation shows that there is both within-language and across-language variation in the application of OS, and that the three accounts can explain some of our data. However, all accounts are faced with challenges, especially when explaining exceptional cases.},
  file = {/Users/mybrink/Zotero/storage/CU7RBFSP/Bentzen et al_2013_Object Shift in spoken Mainland Scandinavian.pdf},
  langid = {english},
  number = {2}
}

@incollection{cardinalettiTypologyStructuralDeficiency1999,
  ids = {cardinalettiTypologyStructuralDeficiency1999a},
  title = {The Typology of Structural Deficiency: {{A}} Case Study of the Three Classes of Pronouns},
  shorttitle = {The Typology of Structural Deficiency},
  booktitle = {Eurotyp},
  author = {Cardinaletti, Anna and Starke, Michal},
  editor = {van Riemsdijk, Henk},
  date = {1999-01-31},
  publisher = {{DE GRUYTER MOUTON}},
  location = {{Berlin, New York}},
  doi = {10.1515/9783110804010.145},
  url = {https://www.degruyter.com/view/books/9783110804010/9783110804010.145/9783110804010.145.xml},
  urldate = {2020-02-18},
  file = {/Users/mybrink/Zotero/storage/P9UW9G7X/Cardinaletti_Starke_1999_The typology of structural deficiency.pdf},
  isbn = {978-3-11-080401-0}
}

@incollection{chomskyDerivationPhase2001,
  title = {Derivation by Phase},
  booktitle = {Ken {{Hale}}: {{A}} Life in Language},
  author = {Chomsky, Noam},
  editor = {Kenstowicz, Michael},
  date = {2001},
  pages = {1--52},
  publisher = {{MIT Press}},
  location = {{Cambridge, MA}},
  file = {/Users/mybrink/Zotero/storage/E8GVWD5W/Chomsky_2001_Derivation by phase.pdf}
}

@incollection{chomskyMinimalistProgramLinguistic1993,
  title = {A Minimalist Program for Linguistic Theory},
  booktitle = {The View from {{Building}} 20: {{Essays}} in Linguistics in Honor of {{Sylvain Bromberger}}},
  author = {Chomsky, Noam},
  editor = {Hale, Kenneth and Keyser, Samuel Jay},
  date = {1993},
  pages = {1--52},
  publisher = {{MIT Press}},
  location = {{Cambridge, MA}}
}

@article{christensenNegshiftLicensingRepair2008,
  title = {Neg-Shift, Licensing, and Repair Strategies},
  author = {Christensen, Ken Ramshj},
  date = {2008-08},
  journaltitle = {Studia Linguistica},
  volume = {62},
  pages = {182--223},
  issn = {00393193, 14679582},
  doi = {10.1111/j.1467-9582.2008.00146.x},
  url = {http://doi.wiley.com/10.1111/j.1467-9582.2008.00146.x},
  urldate = {2020-02-21},
  file = {/Users/mybrink/Zotero/storage/FZLUF56B/Christensen_2008_Neg-shift, licensing, and repair strategies.pdf},
  langid = {english},
  number = {2}
}

@article{christensenNorwegianIngenCase1986,
  title = {Norwegian Ingen: A Case of Post-Syntactic Lexicalization.},
  author = {Christensen, Kirsti Koch},
  date = {1986},
  journaltitle = {Scandinavian Syntax},
  pages = {21--35},
  file = {/Users/mybrink/Zotero/storage/EF3KSRV3/Christensen_1986_Norwegian ingen.pdf},
  keywords = {⛔ No DOI found}
}


@article{eideSwedishCulturomicsGigaword2016,
	title = {The {{Swedish Culturomics Gigaword Corpus}}: {{A One Billion Word Swedish Reference Dataset}} for {{NLP}}},
	author = {Eide, Stian Rødven and Tahmasebi, Nina and Borin, Lars},
	date = {2016},
	pages = {5},
	url = {https://spraakbanken.gu.se/en/resources/gigaword},
	abstract = {In this paper we present a dataset of contemporary Swedish containing one billion words. The dataset consists of a wide range of sources, all annotated using a state-of-the-art corpus annotation pipeline, and is intended to be a static and clearly versioned dataset. This will facilitate reproducibility of experiments across institutions and make it easier to compare NLP algorithms on contemporary Swedish. The dataset contains sentences from 1950 to 2015 and has been carefully designed to feature a good mix of genres balanced over each included decade. The sources include literary, journalistic, academic and legal texts, as well as blogs and web forum entries.},
	file = {/Users/mykelbrinkerhoff/Zotero/storage/QFUJD8LF/Eide et al_2016_The Swedish Culturomics Gigaword Corpus.pdf},
	keywords = {⛔ No DOI found},
	langid = {english}
}


@article{engelsMicrovariationObjectPositions2011,
  ids = {engelsMicrovariationObjectPositions2011a},
  title = {Microvariation in Object Positions: {{Negative Shift}} in {{Scandinavian}}},
  shorttitle = {Microvariation in Object Positions},
  author = {Engels, Eva},
  date = {2011-10},
  journaltitle = {Nordic Journal of Linguistics},
  shortjournal = {Nord J Linguist},
  volume = {34},
  pages = {133--155},
  issn = {0332-5865, 1502-4717},
  doi = {10.1017/S033258651100014X},
  url = {https://www.cambridge.org/core/product/identifier/S033258651100014X/type/journal_article},
  urldate = {2020-01-31},
  abstract = {In the Scandinavian languages, sentential negation must be licensed in Spec–head relation in the IP-domain, necessitating leftward movement of negative objects, Negative Shift (NegS). While string-vacuous NegS is possible in all Scandinavian varieties, there is a fair amount of cross-linguistic variation in non-string-vacuous NegS. In particular, the varieties contrast in which constituents can be crossed by NegS and whether or not crossing of a certain constituent requires the presence of an intervening verb. The paper presents the complex variation as to the distribution of negative objects in Scandinavian, using data from different sources, and outlines an analysis within Fox \& Pesetsky’s (2003, 2005a, b) cyclic linearization model, which accounts for this variation by differences in the availability of the intermediate positions non-string-vacuous movement is forced to proceed through.},
  file = {/Users/mybrink/Zotero/storage/7YLKD66E/Engels_2011_Microvariation in object positions.pdf;/Users/mybrink/Zotero/storage/RYWCHMT9/Engels_2011_Microvariation in object positions.pdf},
  langid = {english},
  number = {2}
}

@article{engelsScandinavianNegativeIndefinites2012,
  title = {Scandinavian {{Negative Indefinites}} and {{Cyclic Linearization}}: {{Scandinavian Negative Indefinites}} and {{Cyclic Linearization}}},
  shorttitle = {Scandinavian {{Negative Indefinites}} and {{Cyclic Linearization}}},
  author = {Engels, Eva},
  date = {2012-06},
  journaltitle = {Syntax},
  volume = {15},
  pages = {109--141},
  issn = {13680005},
  doi = {10.1111/j.1467-9612.2011.00161.x},
  url = {http://doi.wiley.com/10.1111/j.1467-9612.2011.00161.x},
  urldate = {2020-01-29},
  file = {/Users/mybrink/Zotero/storage/XDRQZ52P/Engels_2012_Scandinavian Negative Indefinites and Cyclic Linearization.pdf},
  langid = {english},
  number = {2}
}

@incollection{erteschik-shirScandinavianObjectShift2017,
  title = {Scandinavian {{Object Shift Is Phonology}}},
  booktitle = {Order and Structure in Syntax {{I}}: {{Word}} Order and Syntactic Structure},
  author = {Erteschik-Shir, Nomi and Josefsson, Gunlög},
  editor = {Bailey, Laura R. and Sheehan, Michelle},
  date = {2017-12-18},
  pages = {99--115},
  publisher = {{Language Science Press}},
  location = {{Berlin}},
  doi = {10.5281/ZENODO.1117700},
  url = {https://zenodo.org/record/1117700},
  urldate = {2020-02-21},
  abstract = {The problem addressed in this paper is a case of word order microvariation in Mainland Scandinavian: optional vs. obligatory Object Shift (OS). Following standard assumptions (see \textbackslash{}citealt\{Selkirk1996\}), weak object pronouns are assumed to be affixal clitics at PF which do not themselves have the status of prosodic words. Since adverbs (including negation), are unsuitable as hosts, weak object pronouns may undergo OS, in other words precede adverbs, ending up encliticized onto the preceding verb or subject. In standard Danish, OS is obligatory; the order adverb+weak pronoun is blocked. However, in Swedish, OS is optional, as is the case for some Danish dialects, spoken in the southeastern island area. In our paper we explain the distribution of optional vs. obligatory OS by the phonological properties of the two varieties. What “optional OS” in Swedish and varieties of Danish have in common is the occurrence of a tonal accent, which creates a larger phonological unit than the minimal prosodic word, a Tonal Unit. We propose that the mechanism that allows a weak pronoun to remain in the canonical position in Swedish and the southeastern island dialects in Danish, is the availability of tonal accent. The tonal accent enables the inclusion of the pronoun in such a unit. Standard Danish, on the other hand, lacks tonal accent altogether which is why OS is obligatory in this dialect.},
  file = {/Users/mybrink/Zotero/storage/XHMZ5GAU/Erteschik-Shir_Josefsson_2017_Scandinavian Object Shift Is Phonology.pdf},
  number = {1},
  series = {Open {{Generative Syntax}}}
}

@article{erteschik-shirSoundPatternsSyntax2005,
  title = {Sound {{Patterns}} of {{Syntax}}: {{Object Shift}}},
  shorttitle = {Sound {{Patterns}} of {{Syntax}}},
  author = {Erteschik-Shir, Nomi},
  date = {2005},
  journaltitle = {Theoretical Linguistics},
  volume = {31},
  pages = {47--93},
  issn = {1613-4060},
  doi = {10.1515/thli.2005.31.1-2.47},
  url = {https://www.degruyter.com/view/j/thli.2005.31.issue-1-2/thli.2005.31.1-2.47/thli.2005.31.1-2.47.xml?format=INT},
  urldate = {2020-02-07},
  abstract = {In this paper, I explore what a purely phonological account of object shift (OS) involves and what research questions it leads to, in particular what it means for word order to be phonologically motivated and what morpho-phonological primitives are involved. I pursue the possibility that what licenses OS of full DPs in Icelandic is phonological properties, not found in other Scandinavian languages, together with overt case marking. Although the position of the object is determined phonologically, the architecture I propose, in which topic and focus play a central role, allows for an account of the semantic effect associated with OS in Icelandic.},
  file = {/Users/mybrink/Zotero/storage/2N2X7S86/Erteschik-Shir_2005_Sound Patterns of Syntax.pdf},
  number = {1-2}
}

@collection{erteschik-shirSoundPatternsSyntax2010,
  ids = {erteschik-shirSoundPatternsSyntax2010a,erteschik-shirSoundPatternsSyntax2010b,erteschik-shirSoundPatternsSyntax2010c},
  title = {The Sound Patterns of Syntax},
  editor = {Erteschik-Shir, Nomi and Rochman, Lisa},
  date = {2010},
  publisher = {{Oxford University Press}},
  location = {{Oxford ; New York}},
  isbn = {978-0-19-955686-1 978-0-19-955687-8},
  keywords = {Grammar; Comparative and general,Phonology,Syntax},
  note = {OCLC: ocn437305859},
  number = {29},
  pagetotal = {385},
  series = {Oxford Linguistics}
}

@unpublished{erteschik-shirVariationMainlandScandinavian2019,
  title = {Variation in {{Mainland Scandinavian Object Shift}}: {{A Prosodic Analysis}}.},
  author = {Erteschik-Shir, Nomi and Josefsson, Gunlög and Köhnlein, Björn},
  date = {2019},
  file = {/Users/mybrink/Zotero/storage/EAFM9JYM/Erteschik-Shir et al_2019_Variation in Mainland Scandinavian Object Shift.pdf},
  howpublished = {lingbuzz/003688},
  type = {lingbuzz/003688}
}

@article{foxCyclicLinearizationSyntactic2005,
  title = {Cyclic {{Linearization}} of {{Syntactic Structure}}},
  author = {Fox, Danny and Pesetsky, David},
  date = {2005},
  journaltitle = {Theoretical Linguistics},
  volume = {31},
  pages = {1--45},
  issn = {1613-4060},
  doi = {10.1515/thli.2005.31.1-2.1},
  url = {https://www.degruyter.com/view/j/thli.2005.31.issue-1-2/thli.2005.31.1-2.1/thli.2005.31.1-2.1.xml?format=INT},
  urldate = {2020-02-07},
  abstract = {This paper proposes an architecture for the mapping between syntax and phonology – in particular, that aspect of phonology that determines the linear ordering of words. We propose that linearization is restricted in two key ways. (1) the relative ordering of words is fixed at the end of each phase, or ‘‘Spell-out domain’’; and (2) ordering established in an earlier phase may not be revised or contradicted in a later phase. As a consequence, overt extraction out of a phase P may apply only if the result leaves unchanged the precedence relations established in P. We argue first that this architecture (‘‘cyclic linearization’’) gives us a means of understanding the reasons for successive-cyclic movement. We then turn our attention to more specific predictions of the proposal: in particular, the effects of Holmberg’s Generalization on Scandinavian Object Shift; and also the Inverse Holmberg Effects found in Scandinavian ‘‘Quantifier Movement’’ constructions (Rögnvaldsson (1987); Jónsson (1996); Svenonius (2000)) and in Korean scrambling configurations (Ko (2003, 2004)). The cyclic linearization proposal makes predictions that cross-cut the details of particular syntactic configurations. For example, whether an apparent case of verb fronting results from V-to-C movement or from ‘‘remnant movement’’ of a VP whose complements have been removed by other processes, the verb should still be required to precede its complements after fronting if it preceded them before fronting according to an ordering established at an earlier phase. We argue that ‘‘cross-construction’’ consistency of this sort is in fact found.},
  file = {/Users/mybrink/Zotero/storage/NTAPVWS9/Fox_Pesetsky_2005_Cyclic Linearization of Syntactic Structure.pdf},
  number = {1-2}
}

@article{holmbergMovementDoubleObject2019,
  title = {Movement from the {{Double Object Construction Is Not Fully Symmetrical}}},
  author = {Holmberg, Anders and Sheehan, Michelle and van der Wal, Jenneke},
  date = {2019-10},
  journaltitle = {Linguistic Inquiry},
  shortjournal = {Linguistic Inquiry},
  volume = {50},
  pages = {677--722},
  issn = {0024-3892, 1530-9150},
  doi = {10.1162/ling_a_00322},
  url = {https://www.mitpressjournals.org/doi/abs/10.1162/ling_a_00322},
  urldate = {2020-02-20},
  abstract = {A movement asymmetry arises in some languages that are otherwise symmetrical for both A- and Ā-movement in the double object construction, including Norwegian, North-West British English, and a range of Bantu languages including Zulu and Lubukusu: a Theme object can be Ā-moved out of a Recipient (Goal) passive, but not vice versa. Our explanation of this asymmetry is based on phase theory— more specifically, a stricter version of the Phase Impenetrability Condition proposed by Chomsky (2001) . The effect is that, in a Theme passive, a Recipient object destined for the C-domain gets trapped within the lower V-related phase by movement of the Theme. The same effect is observed in Italian, a language in which only Theme passives are possible. A similar effect is also found in some Bantu languages in connection with object marking/agreement: object agreement with the Theme in a Recipient passive is possible, but not vice versa. We show that this, too, can be understood within the theory that we articulate.},
  file = {/Users/mybrink/Zotero/storage/2QXZHK73/Holmberg et al_2019_Movement from the Double Object Construction Is Not Fully Symmetrical.pdf},
  langid = {english},
  number = {4},
  options = {useprefix=true}
}

@article{holmbergRemarksHolmbergGeneralization1999,
  title = {Remarks on {{Holmberg}}’s {{Generalization}}},
  author = {Holmberg, Anders},
  date = {1999-04},
  journaltitle = {Studia Linguistica},
  shortjournal = {Studia Linguistica},
  volume = {53},
  pages = {1--39},
  issn = {0039-3193, 1467-9582},
  doi = {10.1111/1467-9582.00038},
  url = {https://onlinelibrary.wiley.com/doi/abs/10.1111/1467-9582.00038},
  urldate = {2020-01-29},
  file = {/Users/mybrink/Zotero/storage/FCAR57LR/Holmberg_1999_Remarks on Holmberg’s Generalization.pdf},
  langid = {english},
  number = {1}
}

@book{holmbergRoleInflectionScandinavian1995,
  title = {The Role of Inflection in {{Scandinavian}} Syntax},
  author = {Holmberg, Anders and Platzack, Christer},
  date = {1995},
  publisher = {{Oxford University Press}},
  location = {{New York}},
  file = {/Users/mybrink/Zotero/storage/UKBIWR3X/Holmberg_Platzack_1995_The role of inflection in Scandinavian syntax.pdf},
  isbn = {978-0-19-506745-3 978-0-19-506746-0},
  keywords = {Inflection,Scandinavian languages,Syntax},
  pagetotal = {253},
  series = {Oxford Studies in Comparative Syntax}
}

@thesis{holmbergWordOrderSyntactic1986,
  title = {Word Order and Syntactic Features in the {{Scandinavian}} Languages and {{English}}.},
  author = {Holmberg, Anders},
  date = {1986},
  institution = {{University of Stockholm}},
  location = {{Stockholm, Sweden}},
  file = {/Users/mybrink/Zotero/storage/PGSMIEJJ/Holmberg_1986_Word order and syntactic features in the Scandinavian languages and English.pdf},
  type = {Doctoral dissertation}
}

@book{kayneAntisymmetrySyntax1994,
  title = {The Antisymmetry of Syntax},
  author = {Kayne, Richard S.},
  date = {1994},
  publisher = {{MIT Press}},
  location = {{Cambridge, MA}},
  isbn = {978-0-262-61107-7 978-0-262-11194-2},
  note = {OCLC: 256737033},
  number = {25},
  series = {Linguistic Inquiry Monographs}
}

@book{kristoffersenPhonologyNorwegian2007,
  title = {The Phonology of {{Norwegian}}},
  author = {Kristoffersen, Gjert},
  date = {2007},
  publisher = {{Oxford University Press}},
  location = {{Oxford}},
  file = {/Users/mybrink/Zotero/storage/2Y6NTM3L/Kristoffersen_2007_The phonology of Norwegian.pdf},
  isbn = {978-0-19-922932-1 978-0-19-823765-5},
  langid = {english},
  note = {OCLC: 137221452}
}

@article{legateInterfacePropertiesPhase2003,
  title = {Some {{Interface Properties}} of the {{Phase}}},
  author = {Legate, Julie Anne},
  date = {2003-07},
  journaltitle = {Linguistic Inquiry},
  shortjournal = {Linguistic Inquiry},
  volume = {34},
  pages = {506--515},
  issn = {0024-3892, 1530-9150},
  doi = {10.1162/ling.2003.34.3.506},
  url = {http://www.mitpressjournals.org/doi/10.1162/ling.2003.34.3.506},
  urldate = {2020-02-04},
  file = {/Users/mybrink/Zotero/storage/TFSUZ5QZ/Legate_2003_Some Interface Properties of the Phase.pdf},
  langid = {english},
  number = {3}
}

@collection{lippusNordicProsodyProceedings2013,
  title = {Nordic {{Prosody Proceedings}} of the {{XIth Conference}}, {{Tartu}} 2012},
  editor = {Lippus, Pärtel and Asu, Eva Liina},
  date = {2013},
  publisher = {{Peter Lang GmbH, Internationaler Verlag der Wissenschaften}},
  location = {{Frankfurt}},
  url = {http://nbn-resolving.de/urn:nbn:de:101:1-2014041310430},
  urldate = {2020-02-24},
  isbn = {978-3-653-03047-1},
  langid = {english},
  note = {OCLC: 879508679}
}

@article{medeirosHawaiianVPremnantMovement2013,
  title = {Hawaiian {{VP}}-Remnant Movement: {{A}} Cyclic Linearization Approach},
  shorttitle = {Hawaiian {{VP}}-Remnant Movement},
  author = {Medeiros, David J.},
  date = {2013-04},
  journaltitle = {Lingua},
  shortjournal = {Lingua},
  volume = {127},
  pages = {72--97},
  issn = {00243841},
  doi = {10.1016/j.lingua.2013.01.008},
  url = {https://linkinghub.elsevier.com/retrieve/pii/S0024384113000296},
  urldate = {2020-01-29},
  file = {/Users/mybrink/Zotero/storage/L4AIKJQ7/Medeiros_2013_Hawaiian VP-remnant movement.pdf},
  langid = {english}
}

@article{mikkelsenProsodyFocusObject2011,
  title = {On {{Prosody}} and {{Focus}} in {{Object Shift}}: {{Prosody}} and {{Focus}} in {{Object Shift}}},
  shorttitle = {On {{Prosody}} and {{Focus}} in {{Object Shift}}},
  author = {Mikkelsen, Line},
  date = {2011-09},
  journaltitle = {Syntax},
  volume = {14},
  pages = {230--264},
  issn = {13680005},
  doi = {10.1111/j.1467-9612.2011.00152.x},
  url = {http://doi.wiley.com/10.1111/j.1467-9612.2011.00152.x},
  urldate = {2020-01-29},
  file = {/Users/mybrink/Zotero/storage/CMJ4HZ36/Mikkelsen_2011_On Prosody and Focus in Object Shift_annotated.pdf;/Users/mybrink/Zotero/storage/NHRQTPZV/Mikkelsen_2011_On Prosody and Focus in Object Shift.pdf},
  langid = {english},
  number = {3}
}

@article{moren-duolljaProsodySwedishUnderived2013,
  title = {The Prosody of {{Swedish}} Underived Nouns: {{No}} Lexical Tones Required},
  shorttitle = {The Prosody of {{Swedish}} Underived Nouns},
  author = {Morén-Duolljá, Bruce},
  date = {2013-02-15},
  journaltitle = {Nordlyd},
  shortjournal = {NLY},
  volume = {40},
  pages = {196},
  issn = {1503-8599},
  doi = {10.7557/12.2506},
  url = {https://www.ub.uit.no/baser/septentrio/index.php/nordlyd/article/view/2506},
  urldate = {2020-03-06},
  abstract = {This paper provides a detailed representational analysis of the morpho-prosodic system of underived nouns in a dialect of Swedish.\&nbsp; It shows that the morphology, stress and tonal patterns are not as complex as they first appear once the data are looked at in sufficient detail.\&nbsp; Further, it shows that the renowned Swedish "lexical pitch accent" is not the result of lexical tones/tonemes.\&nbsp; Rather, Swedish is like all other languages and uses tones to mark the edges of prosodic constituents on the surface. "Accent 2" occurs when tones mark the edge of a structural uneven trochee (i.e. recursive foot) and "accent 1" occurs elsewhere. This analysis is counter all other treatments of North Germanic tones and denies the almost unquestioned assumption that there is an underlying tone specification on roots and/or affixes in many North Germanic varieties. At the same time, it unifies the intuitions behind the three previous approaches found in the literature.},
  file = {/Users/mybrink/Zotero/storage/N2RFZ9FD/Morén-Duolljá_2013_The prosody of Swedish underived nouns.pdf},
  number = {1}
}

@book{mullerDanishHeadDrivenPhraseInpreparation,
  title = {Danish in {{Head}}-{{Driven Phrase Structure Grammar}}},
  author = {Müller, Stefan and Ørsnes, Bjarne},
  year = {In preparation},
  publisher = {{Language Science Press}},
  location = {{Berlin}}
}

@incollection{mullerHPSGAnalysisObject2013,
  title = {Towards an {{HPSG Analysis}} of {{Object Shift}} in {{Danish}}},
  booktitle = {Formal {{Grammar}}},
  author = {Müller, Stefan and Ørsnes, Bjarne},
  editor = {Morrill, Glyn and Nederhof, Mark-Jan},
  date = {2013},
  volume = {8036},
  pages = {69--89},
  publisher = {{Springer Berlin Heidelberg}},
  location = {{Berlin, Heidelberg}},
  doi = {10.1007/978-3-642-39998-5_5},
  url = {http://link.springer.com/10.1007/978-3-642-39998-5_5},
  urldate = {2020-01-29},
  editorb = {Hutchison, David and Kanade, Takeo and Kittler, Josef and Kleinberg, Jon M. and Mattern, Friedemann and Mitchell, John C. and Naor, Moni and Nierstrasz, Oscar and Pandu Rangan, C. and Steffen, Bernhard and Sudan, Madhu and Terzopoulos, Demetri and Tygar, Doug and Vardi, Moshe Y. and Weikum, Gerhard},
  editorbtype = {redactor},
  file = {/Users/mybrink/Zotero/storage/IH5Q6ANF/Müller_Ørsnes_2013_Towards an HPSG Analysis of Object Shift in Danish.pdf},
  isbn = {978-3-642-39997-8 978-3-642-39998-5}
}

@article{myrbergProsodicHierarchySwedish2015,
  title = {The Prosodic Hierarchy of {{Swedish}}},
  author = {Myrberg, Sara and Riad, Tomas},
  date = {2015-10},
  journaltitle = {Nordic Journal of Linguistics},
  shortjournal = {Nord J Linguist},
  volume = {38},
  pages = {115--147},
  issn = {0332-5865, 1502-4717},
  doi = {10.1017/S0332586515000177},
  url = {https://www.cambridge.org/core/product/identifier/S0332586515000177/type/journal_article},
  urldate = {2020-02-24},
  abstract = {We give an overview of the phonological properties and processes that define the categories of the prosodic hierarchy in Swedish: the
              prosodic word
              (ω), the
              prosodic phrase
              (φ) and the
              intonation phrase
              (ι). The separation of two types of tonal prominence,
              big accents
              versus
              small accents
              (previously called
              focal
              and
              word accent
              , e.g. Bruce 1977, 2007), is crucial for our analysis. The ω in Swedish needs to be structured on two levels, which we refer to as the minimal ω and the maximal ω, respectively. The minimal ω contains one stress, whereas the maximal ω contains one accent. We argue for a separate category φ that governs the distribution of big accents within clauses. The ι governs the distribution of clause-related edge phenomena like the
              initiality accent
              and right-edge boundary tones as well as the distribution of
              nuclear big accents
              .},
  file = {/Users/mybrink/Zotero/storage/3SHLWHNI/Myrberg_Riad_2015_The prosodic hierarchy of Swedish.pdf},
  langid = {english},
  number = {2}
}

@incollection{penkaDistributionalRestrictionsScandinavian2011,
  title = {Distributional {{Restrictions}} in {{Scandinavian}}},
  booktitle = {Negative Indefinites},
  author = {Penka, Doris},
  date = {2011},
  pages = {163},
  publisher = {{Oxford University Press}},
  abstract = {In the Scandinavian languages, negative indefinites show a restriction in their distribution to the effect that they have to be adjacent to a position a negative marker can occupy. This chapter shows that this distributional restriction follows from the analysis proposed for German in Chapter 3. It discusses other approaches to negative indefinites in Scandinavian and shows that these are insufficient in the light of split readings that also arise in these languages. The chapter concludes with remarks on the resulting cross-linguistic analysis of negative indefinites and its possible extension to other languages, in particular English.},
  file = {/Users/mybrink/Zotero/storage/K2EYBFSB/Penka_2011_Distributional Restrictions in Scandinavian.pdf},
  langid = {english}
}

@book{penkaNegativeIndefinites2011,
  ids = {penkaNegativeIndefinites2010},
  title = {Negative Indefinites},
  author = {Penka, Doris},
  date = {2011},
  publisher = {{Oxford University Press}},
  location = {{Oxford ; New York}},
  doi = {10.1093/acprof:oso/9780199567263.001.0001},
  file = {/Users/mybrink/Zotero/storage/EFSDDZQV/Penka_2011_Negative indefinites.pdf},
  isbn = {978-0-19-956726-3 978-0-19-956727-0},
  keywords = {Grammar; Comparative and general,Negatives,Polarity (Linguistics),Semantics,Syntax},
  number = {no. 32},
  pagetotal = {264},
  series = {Oxford Studies in Theoretical Linguistics}
}

@book{riadPhonologySwedish2014,
  ids = {riadPhonologySwedish2014a},
  title = {The Phonology of {{Swedish}}},
  author = {Riad, Tomas},
  date = {2014},
  publisher = {{Oxford University Press}},
  location = {{Oxford, United Kingdom}},
  isbn = {978-0-19-954357-1},
  keywords = {Phonology,Swedish language},
  note = {OCLC: ocn866836612},
  pagetotal = {338},
  series = {The Phonology of the World's Languages}
}

@article{rognvaldssonOVWORDORDER,
  title = {{{OV WORD ORDER IN ICELANDIC}}},
  author = {Rögnvaldsson, Eiríkur},
  pages = {12},
  file = {/Users/mybrink/Zotero/storage/ABMYCQGE/Rögnvaldsson_OV WORD ORDER IN ICELANDIC.pdf},
  keywords = {⛔ No DOI found},
  langid = {english}
}

@article{selkirkClauseIntonationalPhrase2009,
  title = {On {{Clause}} and Intonational Phrase in {{Japanese}}: {{The}} Syntactic Grounding of Prosodic Constituent Structure},
  author = {Selkirk, Elisabeth},
  date = {2009},
  journaltitle = {Gengo Kenkyu},
  volume = {2009},
  pages = {35--73},
  keywords = {⛔ No DOI found},
  number = {136}
}

@incollection{selkirkSyntaxPhonologyInterface2011,
  title = {The {{Syntax}}-{{Phonology Interface}}},
  booktitle = {The {{Handbook}} of {{Phonological Theory}}},
  author = {Selkirk, Elisabeth},
  date = {2011},
  pages = {435--484},
  publisher = {{Wiley-Blackwell}},
  location = {{Oxford, UK}},
  doi = {10.1002/9781444343069.ch14},
  url = {http://doi.wiley.com/10.1002/9781444343069.ch14},
  file = {/Users/mykelbrinkerhoff/Zotero/Bibliography/2011/SelkirkE/SelkirkE_2011_The Syntax-Phonology Interface.pdf},
  isbn = {978-1-4051-5768-1}
}

@inproceedings{sellsNegationSwedishWhere2000,
  title = {Negation in {{Swedish}}: {{Where It}}’s {{Not At}}},
  booktitle = {Proceedings of the {{LFG00 Conference}}},
  author = {Sells, Peter},
  editor = {Butt, Miriam and King, Tracy Holloway},
  date = {2000},
  pages = {244--263},
  publisher = {{CSLI Publications}},
  location = {{University of California, Berkeley}},
  file = {/Users/mybrink/Zotero/storage/SXNEJGKD/Sells_Negation in Swedish.pdf},
  keywords = {⛔ No DOI found},
  langid = {english}
}

@incollection{sundquistObjectShiftHolmberg2002,
  title = {Object {{Shift}} and {{Holmberg}}’s {{Generalization}} in the {{History}} of {{Norwegian}}},
  booktitle = {Syntactic {{Effects}} of {{Morphological Change}}},
  author = {Sundquist, John D.},
  editor = {Lightfoot, David W.},
  date = {2002-06-27},
  pages = {326--349},
  publisher = {{Oxford University Press}},
  doi = {10.1093/acprof:oso/9780199250691.003.0019},
  url = {http://www.oxfordscholarship.com/view/10.1093/acprof:oso/9780199250691.001.0001/acprof-9780199250691-chapter-19},
  urldate = {2020-01-29},
  file = {/Users/mybrink/Zotero/storage/RC66AY7B/Sundquist_2002_Object Shift and Holmberg’s Generalization in the History of Norwegian.pdf},
  isbn = {978-0-19-925069-1}
}

@article{svenoniusHowPhonologicalObject2005,
  title = {How {{Phonological}} Is {{Object Shift}}?},
  author = {Svenonius, Peter},
  date = {2005},
  journaltitle = {Theoretical Linguistics},
  volume = {31},
  pages = {215--227},
  issn = {1613-4060},
  doi = {10.1515/thli.2005.31.1-2.215},
  url = {https://www.degruyter.com/view/j/thli.2005.31.issue-1-2/thli.2005.31.1-2.215/thli.2005.31.1-2.215.xml?format=INT},
  urldate = {2020-02-07},
  abstract = {The papers here represent very different analyses of Object Shift (OS), and take very different positions on the interaction of phonology and syntax, an area in which there are many unsettled questions.},
  file = {/Users/mybrink/Zotero/storage/98JXE3YA/Svenonius_2005_How Phonological is Object Shift.pdf},
  keywords = {_tablet},
  number = {1-2}
}

@article{svenoniusStrainsNegationNorwegian2002,
  title = {Strains of Negation in {{Norwegian}}},
  author = {Svenonius, Peter},
  date = {2002},
  journaltitle = {Working papers in Scandinavian Syntax},
  volume = {69},
  pages = {121--146},
  file = {/Users/mybrink/Zotero/storage/9UFXA6ZK/Svenonius_2002_Strains of negation in Norwegian.pdf},
  keywords = {⛔ No DOI found}
}

@article{thrainssonFullNPObject2013,
  title = {Full {{NP Object Shift}}: {{The Old Norse Puzzle}} and the {{Faroese Puzzle}} Revisited},
  shorttitle = {Full {{NP Object Shift}}},
  author = {Thráinsson, Höskuldur},
  date = {2013-10},
  journaltitle = {Nordic Journal of Linguistics},
  shortjournal = {Nord J Linguist},
  volume = {36},
  pages = {153--186},
  issn = {0332-5865, 1502-4717},
  doi = {10.1017/S033258651300022X},
  url = {https://www.cambridge.org/core/product/identifier/S033258651300022X/type/journal_article},
  urldate = {2020-01-29},
  abstract = {This paper argues that there is no reason to believe that full NP Object Shift (NPOS) was not found in Old Norse (Old Icelandic) nor that it is more common in Modern Icelandic than in earlier stages of the language. In addition, it is claimed that NPOS is also an option in Modern Faroese, contrary to common belief, although it is much more restricted in Faroese than in Icelandic. These results demonstrate the usefulness of systematic corpus studies while at the same time reminding us of their limits. In addition, they shed a new light on the status of Faroese among the Scandinavian languages and on the nature of intra-speaker variation and grammar competition.},
  file = {/Users/mybrink/Zotero/storage/LY7LI9DQ/Thráinsson_2013_Full NP Object Shift.pdf},
  langid = {english},
  number = {2}
}

@incollection{thrainssonObjectShiftScrambling2001,
  ids = {thrinssonObjectShiftScrambling2001},
  title = {Object {{Shift}} and {{Scrambling}}},
  booktitle = {The {{Handbook}} of {{Contemporary Syntactic Theory}}},
  author = {Thráinsson, Höskuldur},
  date = {2001-01-01},
  pages = {55},
  publisher = {{Blackwell Publishers Ltd}},
  location = {{Oxford, UK}},
  doi = {10.1002/9780470756416.ch6},
  url = {http://doi.wiley.com/10.1002/9780470756416.ch6},
  file = {/Users/mybrink/Zotero/storage/MYXDK3LS/Thráinsson_2001_Object Shift and Scrambling.pdf},
  isbn = {978-0-470-75641-6 978-0-631-20507-4},
  langid = {english}
}

@book{thrainssonSyntaxIcelandic2010,
  ids = {hoskuldurthrainssonSyntaxIcelandic2007,hoskuldurthrainssonSyntaxIcelandic2007a},
  title = {The syntax of Icelandic},
  author = {Thráinsson, Höskuldur},
  date = {2010},
  edition = {1. paperback ed},
  publisher = {{Cambridge Univ. Press}},
  location = {{Cambridge}},
  file = {/Users/mybrink/Zotero/storage/L2LTGFIM/Thráinsson_2010_The syntax of Icelandic.pdf},
  isbn = {978-0-521-59790-6 978-0-521-59190-4},
  keywords = {Grammar; Comparative and general,Icelandic language,Syntax},
  langid = {eng ice},
  note = {OCLC: 820488883},
  pagetotal = {563},
  series = {Cambridge syntax guides}
}

@incollection{viknerObjectShiftScandinavian2017,
  title = {Object {{Shift}} in {{Scandinavian}}},
  booktitle = {The {{Wiley Blackwell Companion}} to {{Syntax}}, {{Second Edition}}},
  author = {Vikner, Sten},
  editor = {Everaert, Martin and van Riemsdijk, Henk C.},
  date = {2017-11-24},
  pages = {1--60},
  publisher = {{John Wiley \& Sons, Inc.}},
  location = {{Hoboken, NJ, USA}},
  doi = {10.1002/9781118358733.wbsyncom114},
  url = {http://doi.wiley.com/10.1002/9781118358733.wbsyncom114},
  urldate = {2020-01-29},
  file = {/Users/mybrink/Zotero/storage/GPETGHBA/Vikner_2017_Object Shift in Scandinavian.pdf},
  isbn = {978-1-118-35873-3},
  langid = {english},
  options = {useprefix=true}
}


